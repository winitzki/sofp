\batchmode
\makeatletter
\def\input@path{{/Users/sergei.winitzki/Code/talks/ftt-fp/}}
\makeatother
\documentclass[english,,russian]{beamer}
\usepackage[T2A,T1]{fontenc}
\usepackage[utf8]{inputenc}
\setcounter{secnumdepth}{3}
\setcounter{tocdepth}{3}
\usepackage{babel}
\usepackage{tipa}
\usepackage{tipx}
\usepackage{amsmath}
\ifx\hypersetup\undefined
  \AtBeginDocument{%
    \hypersetup{unicode=true,pdfusetitle,
 bookmarks=true,bookmarksnumbered=false,bookmarksopen=false,
 breaklinks=false,pdfborder={0 0 1},backref=false,colorlinks=true}
  }
\else
  \hypersetup{unicode=true,pdfusetitle,
 bookmarks=true,bookmarksnumbered=false,bookmarksopen=false,
 breaklinks=false,pdfborder={0 0 1},backref=false,colorlinks=true}
\fi

\makeatletter

%%%%%%%%%%%%%%%%%%%%%%%%%%%%%% LyX specific LaTeX commands.
\DeclareRobustCommand{\cyrtext}{%
  \fontencoding{T2A}\selectfont\def\encodingdefault{T2A}}
\DeclareRobustCommand{\textcyr}[1]{\leavevmode{\cyrtext #1}}


%%%%%%%%%%%%%%%%%%%%%%%%%%%%%% Textclass specific LaTeX commands.
% this default might be overridden by plain title style
\newcommand\makebeamertitle{\frame{\maketitle}}%
% (ERT) argument for the TOC
\AtBeginDocument{%
  \let\origtableofcontents=\tableofcontents
  \def\tableofcontents{\@ifnextchar[{\origtableofcontents}{\gobbletableofcontents}}
  \def\gobbletableofcontents#1{\origtableofcontents}
}

%%%%%%%%%%%%%%%%%%%%%%%%%%%%%% User specified LaTeX commands.
\usetheme[secheader]{Boadilla}
\usecolortheme{seahorse}
\title[Chapter 10: Free type constructions]{Chapter 10: Free type constructions}
%\subtitle{Part 2: Their laws and structure}
\author{Sergei Winitzki}
\date{2018-11-22}
\institute[ABTB]{Academy by the Bay}
\setbeamertemplate{headline}{} % disable headline at top
\setbeamertemplate{navigation symbols}{} % disable navigation bar at bottom
\usepackage[all]{xy}
\usepackage[nocenter]{qtree}
\makeatletter
% Macros to assist LyX with XYpic when using scaling.
\newcommand{\xyScaleX}[1]{%
\makeatletter
\xydef@\xymatrixcolsep@{#1}
\makeatother
} % end of \xyScaleX
\makeatletter
\newcommand{\xyScaleY}[1]{%
\makeatletter
\xydef@\xymatrixrowsep@{#1}
\makeatother
} % end of \xyScaleY
\newcommand{\shui}{\begin{CJK}{UTF8}{gbsn}水\end{CJK}}
\usepackage{CJKutf8} % For occasional Chinese characters. Also, add "russian" to documentclass.

\makeatother

\begin{document}
\frame{\titlepage}
\begin{frame}{Free constructions in mathematics: Example I}
\begin{itemize}
\item Consider the Russian letter ц (ts\`{e}) and the Chinese word \begin{CJK}{UTF8}{gbsn}水\end{CJK}
(shu\textipa{\v i})
\item We want to \emph{multiply} ц by \shui. Multiply how?
\item Say, we want an associative (but noncommutative) product of them
\begin{itemize}
\item So we want to define a \emph{semigroup} that \emph{contains} ц and
\shui
\begin{itemize}
\item while we still know nothing about ц and \shui
\end{itemize}
\end{itemize}
\item Consider the set of all \emph{unevaluated expressions} such as ц$\cdot$\shui$\cdot$\shui$\cdot$ц$\cdot$\shui
\begin{itemize}
\item Here ц$\cdot$\shui~is different from \shui$\cdot$ц but $\left(a\cdot b\right)\cdot c=a\cdot\left(b\cdot c\right)$
\end{itemize}
\item All these expressions form a \textbf{free semigroup} generated by
ц and \shui
\item Example calculation: (\shui$\cdot$\shui)$\cdot$(ц$\cdot$\shui)$\cdot$ц
$=$ \shui$\cdot$\shui$\cdot$ц$\cdot$\shui$\cdot$ц
\end{itemize}
How to represent this as a data type:
\begin{itemize}
\item Tree encoding, as the full expression tree: ((\shui,\shui),(ц,\shui)),ц)
\begin{itemize}
\item Implement the operation $a\cdot b$ as pair constructor (easy and
cheap)
\end{itemize}
\item ``Smart'' encoding, as a reduced structure: List(\shui``Smart'' encoding, as a reduced structure: List(\shui,\shui,ц,\shui,ц)
\begin{itemize}
\item Implement the opration $a\cdot b$ by concatenating the lists (more
expensive)
\end{itemize}
\end{itemize}
\end{frame}

\begin{frame}{Free constructions in mathematics: Example II}
\begin{itemize}
\item Want to define a product operation for $n$-dimensional vectors: $\mathbf{v}_{1}\otimes\mathbf{v}_{2}$
\item The $\otimes$ must be associative and distributive (but not commutative):
\begin{align*}
\mathbf{u}\otimes\left(\mathbf{v}\otimes\mathbf{w}\right) & =\left(\mathbf{u}\otimes\mathbf{v}\right)\otimes\mathbf{w}\\
\mathbf{u}\otimes\left(a_{1}\mathbf{v}_{1}+a_{2}\mathbf{v}_{2}\right) & =a_{1}\left(\mathbf{u}\otimes\mathbf{v}_{1}\right)+a_{2}\left(\mathbf{u}\otimes\mathbf{v}_{2}\right)\\
\left(a_{1}\mathbf{v}_{1}+a_{2}\mathbf{v}_{2}\right) & \otimes\mathbf{u}=a_{1}\left(\mathbf{v}_{1}\otimes\mathbf{u}\right)+a_{2}\left(\mathbf{v}_{2}\otimes\mathbf{u}\right)
\end{align*}
\item We have such a product for 3-dimensional vectors only; ignore that
\item Consider \emph{unevaluated} \emph{expressions} of the form $\mathbf{u}_{1}\otimes\mathbf{v}_{1}+\mathbf{u}_{2}\otimes\mathbf{v}_{2}+...$
\end{itemize}
\end{frame}

\begin{frame}{Exercises}
\begin{enumerate}
\item {\footnotesize{}\vspace{-0.15cm}Show that }{\footnotesize\par}
\end{enumerate}
\end{frame}

\end{document}
