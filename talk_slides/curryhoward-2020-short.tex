%% LyX 2.3.3 created this file.  For more info, see http://www.lyx.org/.
%% Do not edit unless you really know what you are doing.
\documentclass[english]{beamer}
\usepackage[T1]{fontenc}
\usepackage[latin9]{inputenc}
\setcounter{secnumdepth}{3}
\setcounter{tocdepth}{3}
\usepackage{babel}
\usepackage{amsmath}
\ifx\hypersetup\undefined
  \AtBeginDocument{%
    \hypersetup{unicode=true,pdfusetitle,
 bookmarks=true,bookmarksnumbered=false,bookmarksopen=false,
 breaklinks=false,pdfborder={0 0 1},backref=false,colorlinks=true}
  }
\else
  \hypersetup{unicode=true,pdfusetitle,
 bookmarks=true,bookmarksnumbered=false,bookmarksopen=false,
 breaklinks=false,pdfborder={0 0 1},backref=false,colorlinks=true}
\fi

\makeatletter

%%%%%%%%%%%%%%%%%%%%%%%%%%%%%% LyX specific LaTeX commands.
%% Because html converters don't know tabularnewline
\providecommand{\tabularnewline}{\\}

%%%%%%%%%%%%%%%%%%%%%%%%%%%%%% Textclass specific LaTeX commands.
% this default might be overridden by plain title style
\newcommand\makebeamertitle{\frame{\maketitle}}%
% (ERT) argument for the TOC
\AtBeginDocument{%
  \let\origtableofcontents=\tableofcontents
  \def\tableofcontents{\@ifnextchar[{\origtableofcontents}{\gobbletableofcontents}}
  \def\gobbletableofcontents#1{\origtableofcontents}
}
\newenvironment{lyxcode}
  {\par\begin{list}{}{
    \setlength{\rightmargin}{\leftmargin}
    \setlength{\listparindent}{0pt}% needed for AMS classes
    \raggedright
    \setlength{\itemsep}{0pt}
    \setlength{\parsep}{0pt}
    \normalfont\ttfamily}%
   \def\{{\char`\{}
   \def\}{\char`\}}
   \def\textasciitilde{\char`\~}
   \item[]}
  {\end{list}}

%%%%%%%%%%%%%%%%%%%%%%%%%%%%%% User specified LaTeX commands.
\usetheme[secheader]{Boadilla}
\usecolortheme{seahorse}
\title[Curry-Howard correspondence]{A brief tutorial on the Curry-Howard correspondence}
\subtitle{For programmers, with code examples in Scala}
\author{Sergei Winitzki}
\date{September 19, 2020}
\institute[ABTB]{Academy By the Bay}
\setbeamertemplate{headline}{} % disable headline at top
\setbeamertemplate{navigation symbols}{}

% Double-stroked fonts to replace the non-working \mathbb{1}.
\usepackage{bbold}
\DeclareMathAlphabet{\bbnumcustom}{U}{BOONDOX-ds}{m}{n} % Use BOONDOX-ds or bbold.
\newcommand{\custombb}[1]{\bbnumcustom{#1}}
% The LyX document will define a macro \bbnum{#1} that calls \custombb{#1}.

\usepackage{relsize} % make math symbols larger or smaller
\usepackage{stmaryrd} % some extra symbols such as \fatsemi
% Note: using \forwardcompose inside a \text{} will cause a LaTeX error!
\newcommand{\forwardcompose}{\hspace{1.5pt}\ensuremath\mathsmaller{\fatsemi}\hspace{1.5pt}}


% Make underline green.
\definecolor{greenunder}{rgb}{0.1,0.6,0.2}
%\newcommand{\munderline}[1]{{\color{greenunder}\underline{{\color{black}#1}}\color{black}}}
\def\mathunderline#1#2{\color{#1}\underline{{\color{black}#2}}\color{black}}
% The LyX document will define a macro \gunderline{#1} that will use \mathunderline with the color `greenunder`.
%\def\gunderline#1{\mathunderline{greenunder}{#1}} % This is now defined by LyX itself with GUI support.

% Scala syntax highlighting. See https://tex.stackexchange.com/questions/202479/unable-to-define-scala-language-with-listings
%\usepackage[T1]{fontenc}
%\usepackage[utf8]{inputenc}
%\usepackage{beramono}
%\usepackage{listings}
% The listing settings are now supported by LyX in a separate section "Listings".
\usepackage{xcolor}

\definecolor{scalakeyword}{rgb}{0.16,0.07,0.5}
\definecolor{dkgreen}{rgb}{0,0.6,0}
\definecolor{gray}{rgb}{0.5,0.5,0.5}
\definecolor{mauve}{rgb}{0.58,0,0.82}
\definecolor{aqua}{rgb}{0.9,0.96,0.999}
\definecolor{scalatype}{rgb}{0.2,0.3,0.2}

\makeatother

\begin{document}
\frame{\titlepage}
\global\long\def\gunderline#1{\mathunderline{greenunder}{#1}}%
\global\long\def\bef{\forwardcompose}%
\global\long\def\bbnum#1{\custombb{#1}}%

\begin{frame}{What problems does Curry-Howard correspondence solve?}

The CH correspondence is a theory that answers these questions:
\begin{enumerate}
\item Can a program compute a value of type $X$ given values of some types
$A$, $B$, $C$, ...? Example:\vspace{-0.8\baselineskip}
\end{enumerate}
\begin{lyxcode}
\textcolor{blue}{\footnotesize{}def~f{[}A,~B,~C{]}(x:~A~=>~Option{[}B{]},~y:~Either{[}A,~C{]},~z:~C~=>~B):~B~=~\{}{\footnotesize\par}

\textcolor{blue}{\footnotesize{}~~~val~t:~Either{[}B,~C{]}~=~???}\textcolor{darkgray}{\footnotesize{}~//~Can~we~implement~`t`~here?}{\footnotesize\par}

\textcolor{blue}{\footnotesize{}~~~t.map(z).merge}{\footnotesize\par}

\textcolor{blue}{\footnotesize{}\}}{\footnotesize\par}
\end{lyxcode}
\begin{enumerate}
\item []\addtocounter{enumi}{1}\vspace*{-0.5cm}
\item Can we infer the code of a function from its type signature?\\
Examples:\vspace{-0.8\baselineskip}
\end{enumerate}
\begin{lyxcode}
\textcolor{blue}{\footnotesize{}def~f{[}A,~B,~C{]}:~(A~=>~Either{[}B,~C{]})~=>~Either{[}A,~C{]}~=>~Either{[}B,~C{]}}{\footnotesize\par}

\textcolor{blue}{\footnotesize{}def~g{[}A,~B,~C{]}:~(A~=>~Either{[}B,~C{]})~=>~Either{[}A~=>~B,~A~=>~C{]}}{\footnotesize\par}

\textcolor{blue}{\footnotesize{}def~h{[}A,~B{]}:~((((A~=>~B)~=>~A)~=>~A)~=>~B)~=>~B}{\footnotesize\par}
\end{lyxcode}
Theory gives an algorithm for writing code ``guided by the types''
\begin{itemize}
\item The \href{https://github.com/Chymyst/curryhoward}{curryhoward} library
generates Scala code from type signatures
\begin{itemize}
\item Often, there is only one ``useful'' implementation out of many
\begin{itemize}
\item The \texttt{\textcolor{blue}{\footnotesize{}curryhoward}} library
tries to find that implementation
\end{itemize}
\end{itemize}
\end{itemize}
\begin{lyxcode}
\textcolor{blue}{\footnotesize{}def~h{[}A,~B{]}:~((((A~=>~B)~=>~A)~=>~A)~=>~B)~=>~B~~=~~implement}{\footnotesize\par}
\end{lyxcode}
\end{frame}

\begin{frame}{From types to logical propositions I. $\mathcal{CH}$-propositions}

\begin{itemize}
\item How to \emph{prove} that this function is not implementable?\vspace{-0.8\baselineskip}
\end{itemize}
\begin{lyxcode}
\textcolor{blue}{\footnotesize{}def~bad{[}A,~B,~C{]}(x:~A~=>~Option{[}B{]},~y:~Either{[}A,~C{]},~z:~C~=>~B):~B}{\footnotesize\par}
\end{lyxcode}
The idea is to build a system of logical derivation rules and axioms
(a \textbf{logic})

The logic should be able to prove rigorously whether any code expression
in the body of the function \texttt{\textcolor{blue}{\footnotesize{}bad}}
can compute values of type \texttt{\textcolor{blue}{\footnotesize{}B}} 
\begin{itemize}
\item Denote such \emph{propositions} by ${\cal CH}(B)$ -- ``$\mathcal{C}$ode
$\mathcal{H}$as a value of type \texttt{\textcolor{blue}{\footnotesize{}B}}''
\end{itemize}
How to obtain rules for reasoning about $\mathcal{CH}$-propositions?
\begin{itemize}
\item The code of \texttt{\textcolor{blue}{\footnotesize{}bad}} might contain
expressions such as \texttt{\textcolor{blue}{\footnotesize{}y.map(z)}} 
\begin{itemize}
\item This computes a value of type \texttt{\textcolor{blue}{\footnotesize{}Either{[}A,
B{]}}} from values of types \texttt{\textcolor{blue}{\footnotesize{}Either{[}A,
C{]}}} and \texttt{\textcolor{blue}{\footnotesize{}(C => B)}}{\footnotesize\par}
\end{itemize}
\item Code expressions create\,\emph{logical relationships} between $\mathcal{CH}$-propositions
\begin{itemize}
\item ``Logical relationship'': $X$ can be proved true if $A$, $B$,
$C$ are true
\item In logic, such a proof task is represented by a \textbf{sequent}
\begin{itemize}
\item Notation: $A,B,C\vdash X$; the \textbf{premises} are $A,B,C$ and
the \textbf{goal} is $X$
\end{itemize}
\item Proofs are achieved via axioms and derivation rules
\begin{itemize}
\item Axioms: sequents that are true without proof
\item Derivation rules: prove a sequent given proofs of some other sequent(s)
\end{itemize}
\end{itemize}
\end{itemize}
\end{frame}

\begin{frame}{From types to logical propositions II. Fully parametric code}

\vspace{-0.1cm}To determine the logical relationships between types,
we need to know \emph{all possible code snippets}

``Fully parametric'' code allows only combinations of these snippets:
\begin{itemize}
\item Use an existing value \texttt{\textcolor{blue}{\footnotesize{}x}}
of type \texttt{\textcolor{blue}{\footnotesize{}A}}. Scala code: \texttt{\textcolor{blue}{\footnotesize{}val
y:~A = x}}{\footnotesize\par}
\item Tuple type: \texttt{\textcolor{blue}{\footnotesize{}(A, B)}}{\footnotesize\par}
\begin{itemize}
\item Create: \texttt{\textcolor{blue}{\footnotesize{}val pair:\ (A, B)
= (a, b)}} 
\item Use: \texttt{\textcolor{blue}{\footnotesize{}val y:\ B = pair.\_2}}{\footnotesize\par}
\end{itemize}
\item Function type: \texttt{\textcolor{blue}{\footnotesize{}A => B}}{\footnotesize\par}
\begin{itemize}
\item Create: \texttt{\textcolor{blue}{\footnotesize{}def f:\ (A => B)
= \{ x: A => ... /{*}}}(may use \texttt{\textcolor{blue}{\footnotesize{}x}}
here)\texttt{\textcolor{blue}{\footnotesize{}{*}/ \}}} 
\item Use: \texttt{\textcolor{blue}{\footnotesize{}val y:\ B = f(a)}}{\footnotesize\par}
\end{itemize}
\item Disjunctive type: \texttt{\textcolor{blue}{\footnotesize{}Either{[}A,
B{]}}} 
\begin{itemize}
\item Create:\\
 \texttt{\textcolor{blue}{\footnotesize{}\ val x:\ Either{[}A, B{]}
= Left(a); val y:\ Either{[}A, B{]} = Right(b)}}{\footnotesize\par}
\item Use: \texttt{\textcolor{blue}{\footnotesize{}val z:\ C = x match
\{}}~\\
\texttt{\textcolor{blue}{\footnotesize{} case Left(a) => ...}}~\\
\texttt{\textcolor{blue}{\footnotesize{} case Right(b) => ...}}~\\
\texttt{\textcolor{blue}{\footnotesize{}\}}}{\footnotesize\par}
\end{itemize}
\item Unit type: \texttt{\textcolor{blue}{\footnotesize{}Unit}}{\footnotesize\par}
\begin{itemize}
\item Create: \texttt{\textcolor{blue}{\footnotesize{}val x:\ Unit = ()}}{\footnotesize\par}
\end{itemize}
\end{itemize}
\end{frame}

\begin{frame}{From types to logical propositions III. Sequents}

\begin{itemize}
\item Sequents correspond to code fragments that have specified types
\item A sequent $\mathcal{CH}(X),\mathcal{CH}(Y)\vdash\mathcal{CH}(Z)$
corresponds to an \emph{expression} of type \texttt{\textcolor{blue}{\footnotesize{}Z}}
that uses some previously defined values \texttt{\textcolor{blue}{\footnotesize{}x:X}}
and \texttt{\textcolor{blue}{\footnotesize{}y:Y}} 
\begin{itemize}
\item Sequents only describe the \emph{types} of expressions and their parts
\end{itemize}
\item Each allowed code snippet means that we can compute a value of some
type given value(s) of other type(s)
\end{itemize}
Express this in sequent notation as \textbf{derivation rules}:
\begin{itemize}
\item Use an existing value:{\small{} $\mathcal{CH}(A)\vdash\mathcal{CH}(A)$}{\small\par}
\item Create tuple:{\small{} $\mathcal{CH}(A),\mathcal{CH}(B)\vdash\mathcal{CH}($}\texttt{\textcolor{blue}{\small{}Tuple2(A,B)}}{\small{}$)$}{\small\par}
\item Use tuple: {\small{}$\mathcal{CH}($}\texttt{\textcolor{blue}{\small{}Tuple2(A,B)}}{\small{}$)\vdash\mathcal{CH}(A)$}
and {\small{}$\mathcal{CH}($}\texttt{\textcolor{blue}{\small{}Tuple2(A,B)}}{\small{}$)\vdash\mathcal{CH}(B)$ }{\small\par}
\item Create function:{\small{} $\emptyset\vdash\mathcal{CH}($}\texttt{\textcolor{blue}{\small{}A
=> B}}{\small{}$)$} if given {\small{}$\mathcal{CH}(A)\vdash\mathcal{CH}(B)$}\texttt{\textcolor{blue}{\small{} }}{\small\par}
\begin{itemize}
\item Function body is an expression of type \texttt{\textcolor{blue}{\footnotesize{}B}}
that uses \texttt{\textcolor{blue}{\footnotesize{}x}} of type \texttt{\textcolor{blue}{\footnotesize{}A}}{\footnotesize\par}
\end{itemize}
\item Use function:{\small{} $\mathcal{CH}($}\texttt{\textcolor{blue}{\small{}A
=> B}}{\small{}$),\mathcal{CH}(A)\vdash\mathcal{CH}(B)$}{\small\par}
\item Create disjunctive value:{\small{} $\mathcal{CH}(A)\vdash\mathcal{CH}($}\texttt{\textcolor{blue}{\small{}Either{[}A,
B{]}}}{\small{}$)$} and{\small{} $\mathcal{CH}(B)\vdash\mathcal{CH}($}\texttt{\textcolor{blue}{\small{}Either{[}A,
B{]}}}{\small{}$)$}{\small\par}
\item Use disjunctive value:\\
~ ~{\small{} $\mathcal{CH}($}\texttt{\textcolor{blue}{\small{}A
=> C}}{\small{}$),\mathcal{CH}($}\texttt{\textcolor{blue}{\small{}B
=> C}}{\small{}$),\mathcal{CH}($}\texttt{\textcolor{blue}{\small{}Either{[}A,
B{]}}}{\small{}$)\vdash\mathcal{CH}(C)$}{\small\par}
\item Create unit value: {\small{}$\emptyset\vdash\mathcal{CH}($}\texttt{\textcolor{blue}{\small{}Unit}}{\small{}$)$}{\small\par}
\end{itemize}
\end{frame}

\begin{frame}{Translating language constructions into the logic I}

\begin{itemize}
\item If there are no other allowed code snippets, we can summarize the
correspondence between type constructors and propositions:
\end{itemize}
\begin{center}
\begin{tabular}{|c|c|c|}
\hline 
\textbf{Scala type} & \textbf{Proposition in logic} & \textbf{Short type notation}\tabularnewline
\hline 
\hline 
\texttt{\textcolor{blue}{\footnotesize{}A}} & ${\cal CH}(A)$ & $A$\tabularnewline
\hline 
\texttt{\textcolor{blue}{\footnotesize{}(A, B)}} & ${\cal CH}(A)\wedge\mathcal{CH}(B)$ & $A\times B$\tabularnewline
\hline 
\texttt{\textcolor{blue}{\footnotesize{}Either{[}A, B{]}}} & ${\cal CH}(A)\vee\mathcal{CH}(B)$ & $A+B$\tabularnewline
\hline 
\texttt{\textcolor{blue}{\footnotesize{}A => B}} & ${\cal CH}(A)\Rightarrow\mathcal{CH}(B)$ & $A\rightarrow B$\tabularnewline
\hline 
\texttt{\textcolor{blue}{\footnotesize{}()}} & \emph{True} ; \emph{$\top$} & $\bbnum 1$\tabularnewline
\hline 
\texttt{\textcolor{blue}{\footnotesize{}Nothing}} & \emph{False} ; \emph{$\bot$} & $\bbnum 0$\tabularnewline
\hline 
\end{tabular}
\par\end{center}

We can now translate types into logic formulas and back
\begin{itemize}
\item Example: \texttt{\textcolor{blue}{\footnotesize{}def duplicate{[}A{]}: A
=> (A, A)}}{\footnotesize\par}
\begin{itemize}
\item The type of this function in the short type notation is $A\rightarrow A\times A$
\item This corresponds to the logical formula $\forall A.~\mathcal{CH}(A)\Rightarrow\mathcal{CH}(A)\wedge\mathcal{CH}(A)$
\end{itemize}
\item The question about the function \texttt{\textcolor{blue}{\footnotesize{}bad}}
is written in logic as this sequent:{\small{}
\[
{\cal CH}(A\rightarrow\bbnum 1+B),{\cal CH}(A+C),{\cal CH}(C\rightarrow B)\vdash{\cal CH}(B)
\]
}{\small\par}
\end{itemize}
\end{frame}

\begin{frame}{Translating language constructions into the logic II}


\framesubtitle{What are the axioms and the derivation rules in the logic of types?}

\vspace{-0.4\baselineskip}
The set of \emph{all well-typed fully parametric programs} $\cong$
the set of \emph{all valid derivations in the logic of types}
\begin{itemize}
\item Write just $A$ instead of $\mathcal{CH}(A)$ and use short type notation
\item Axioms:
\begin{itemize}
\item $\emptyset\vdash\top$ -- create the value of unit type
\item $A\vdash A$ -- use variable
\item $A,B\vdash(A\times B)$ -- create tuple
\item $(A\times B)\vdash A$ -- use left part of tuple
\item $(A\times B)\vdash B$ -- use right part of tuple
\item $A,(A\rightarrow B)\vdash B$ -- apply function to argument
\item $A\vdash(A+B)$ -- create left part of \texttt{\textcolor{blue}{\footnotesize{}Either}}{\footnotesize\par}
\item $B\vdash(A+B)$ -- create right part of \texttt{\textcolor{blue}{\footnotesize{}Either}}{\footnotesize\par}
\item $(A+B),(A\rightarrow C),(B\rightarrow C)\vdash C$ -- use \texttt{\textcolor{blue}{\footnotesize{}Either}}
via match/case
\end{itemize}
\item Derivation rules:
\begin{itemize}
\item ``create function'': we can prove $\emptyset\vdash(A\rightarrow B)$
given $A\vdash B$
\item ``add premise'': we can prove $A,...,C,D\vdash G$ given $A,...,C\vdash G$ 
\item ``reorder'': we can prove $B,A,C,...\vdash G$ given $A,B,C,...\vdash G$ 
\end{itemize}
\end{itemize}
\end{frame}

\begin{frame}{The logic of types I}

Now we have all the axioms and the derivation rules of the logic of
types.
\begin{itemize}
\item What theorems can we derive in this logic?
\item Example theorem: $\forall A.~\forall B.~A\rightarrow B\rightarrow A$
or $\emptyset\vdash A\rightarrow B\rightarrow A$
\begin{itemize}
\item Start with an axiom $A\vdash A$; add an unused premise $B$, get
$A,B\vdash A$
\item Use the ``create function'' rule with $B$ and $A$, get $A\vdash B\rightarrow A$
\item Use the ``create function'' rule with $A$ and $B\rightarrow A$,
get the final sequent $\emptyset\vdash A\rightarrow B\rightarrow A$
showing that $\forall A.~\forall B.~A\rightarrow B\rightarrow A$
is a \textbf{theorem} since $\emptyset\vdash A\rightarrow B\rightarrow A$
was derived from no premises for all $A$ and $B$
\end{itemize}
\item What code does this describe?
\begin{itemize}
\item The axiom $A\vdash A$ represents the expression \texttt{\textcolor{blue}{\footnotesize{}x:~A}} 
\item The unused premise $B$ corresponds to unused variable \texttt{\textcolor{blue}{\footnotesize{}y:~B}} 
\item The ``create function'' rule gives the function \texttt{\textcolor{blue}{\footnotesize{}\{
y:~B => x \}}} 
\item The second ``create function'' rule gives \texttt{\textcolor{blue}{\footnotesize{}\{
x:~A => y:~B => x \}}} 
\item Complete code:
\end{itemize}
\begin{lyxcode}
\textcolor{blue}{\footnotesize{}def~f{[}A,~B{]}:~A~=>~B~=>~A~=~\{~(x:~A)~=>~(y:~B)~=>~x~\}}{\footnotesize\par}
\end{lyxcode}
\item Any code expression's type can be translated into a sequent
\item A proof of a theorem directly guides us in writing code for that type
\item The code can be used to find a proof (every code snippet gives a rule)
\end{itemize}
\end{frame}

\begin{frame}{Correspondence between programs and proofs}

\begin{itemize}
\item By construction, any theorem can be implemented in code
\end{itemize}
\begin{center}
\begin{tabular}{|c|c|}
\hline 
\textbf{Proposition} & \textbf{Scala code}\tabularnewline
\hline 
\hline 
$\forall A.~A\rightarrow A$ & \texttt{\textcolor{blue}{\footnotesize{}\{ x:~A => x \}}}\tabularnewline
\hline 
$\forall A.~A\rightarrow1$ & \texttt{\textcolor{blue}{\footnotesize{}\{ x:~A => () \}}}\tabularnewline
\hline 
$\forall A.~\forall B.~A\rightarrow A+B$ & \texttt{\textcolor{blue}{\footnotesize{}Left.apply}}\tabularnewline
\hline 
$\forall A.~\forall B.~A\times B\rightarrow A$ & \texttt{\textcolor{blue}{\footnotesize{}\_.\_1}}\tabularnewline
\hline 
$\forall A.~\forall B.~A\rightarrow B\rightarrow A$ & \texttt{\textcolor{blue}{\footnotesize{}\{ x:~A => y:~B => x \}}}\tabularnewline
\hline 
\end{tabular}
\par\end{center}
\begin{itemize}
\item ``Types are propositions, programs are proofs''
\item Also, non-theorems \emph{cannot be implemented} in code 
\begin{itemize}
\item Examples of non-theorems:\\
 $\forall A.~1\rightarrow A$; \  \  $\quad\forall A.~\forall B.~A+B\rightarrow A$;
\\
$\forall A.~\forall B.~A\rightarrow A\times B$; \  $\quad\forall A.~\forall B.~(A\rightarrow B)\rightarrow A$
\end{itemize}
\item Given a type's formula, can we implement it in code? Not obvious.
\begin{itemize}
\item Example: $\forall A.~\forall B.~((((A\rightarrow B)\rightarrow A)\rightarrow A)\rightarrow B)\rightarrow B$
\begin{itemize}
\item Can we write a function with this type? Can we prove this formula?
\end{itemize}
\end{itemize}
\end{itemize}
\end{frame}

\begin{frame}{The logic of types II}


\framesubtitle{What kind of logic is this? What do mathematicians call this logic?}

This is called ``intuitionistic propositional logic'', IPL (also
``constructive'')
\begin{itemize}
\item This is a ``nonclassical'' logic because it is different from Boolean
logic
\item Disjunction works differently from Boolean logic
\begin{itemize}
\item Example: $(A\rightarrow B+C)\vdash(A\rightarrow B)+(A\rightarrow C)$
does not hold in IPL
\item This is counter-intuitive!
\item We cannot implement a function with this type:\vspace{-0.75\baselineskip}
\end{itemize}
\begin{lyxcode}
\textcolor{blue}{\footnotesize{}def~q{[}A,~B,~C{]}:~(A~=>~Either{[}B,~C{]})~=>~Either{[}A~=>~B,~A~=>~C{]}}{\footnotesize\par}
\end{lyxcode}
\begin{itemize}
\item Disjunction is ``constructive'': need to supply one of the parts
\begin{itemize}
\item ...but cannot compute \texttt{\textcolor{blue}{\footnotesize{}A =>
B}} or \texttt{\textcolor{blue}{\footnotesize{}A => C}} from \texttt{\textcolor{blue}{\footnotesize{}A
=> Either{[}B, C{]}}} 
\end{itemize}
\end{itemize}
\item Implication works differently
\begin{itemize}
\item Example: $\left(\left(A\rightarrow B\right)\rightarrow A\right)\rightarrow A$
holds in Boolean logic but not in IPL
\item Cannot compute an \texttt{\textcolor{blue}{\footnotesize{}x:~A}}
because of insufficient data
\end{itemize}
\item Conjunction works the same as in Boolean logic
\begin{itemize}
\item Example: $(A\rightarrow B\times C)\vdash\left(A\rightarrow B\right)\times\left(A\rightarrow C\right)$ 
\end{itemize}
\end{itemize}
\end{frame}

\begin{frame}{The logic of types III}


\framesubtitle{How to determine whether a given IPL formula is a theorem?}
\begin{itemize}
\item In Boolean logic, we can compute the ``truth value'' of a formula
and decide whether the formula is a theorem
\item The IPL cannot have a truth table with a fixed number of truth values
\begin{itemize}
\item This was proved by G\"odel in 1932 (see \href{https://en.wikipedia.org/wiki/Many-valued_logic}{Wikipedia page})
\end{itemize}
\item The IPL has a decision procedure (algorithm) that either finds a proof
for a given IPL formula, or determines that there is no proof
\item There may be several inequivalent proofs of an IPL theorem
\item Each proof can be \emph{automatically translated} into code
\begin{itemize}
\item The \href{https://hackage.haskell.org/package/djinn-ghc}{djinn-ghc}
compiler plugin and the \href{https://github.com/nomeata/ghc-justdoit}{JustDoIt plugin}
implement an IPL prover in Haskell, and generate Haskell code from
types
\item The \href{https://github.com/Chymyst/curryhoward}{curryhoward} library
implements an IPL prover as a Scala macro, and generates Scala code
from types
\end{itemize}
\item All these IPL provers use the same basic algorithm called LJT 
\begin{itemize}
\item presented in the paper {\footnotesize{}\href{https://rd.host.cs.st-andrews.ac.uk/publications/jsl57.pdf}{[Dyckhoff 1992]}}{\footnotesize\par}
\end{itemize}
\end{itemize}
\end{frame}

\begin{frame}{Proof search I: looking for an algorithm}


\framesubtitle{Why our initial presentation of IPL does not give a proof search
algorithm}

The FP type constructions give nine axioms and three derivation rules:

\begin{minipage}[t]{0.49\columnwidth}%
\begin{itemize}
\item $\Gamma,A,B\vdash A\times B$ 
\item $\Gamma,A\times B\vdash A$ 
\item $\Gamma,A\times B\vdash B$
\item $\Gamma,A\rightarrow B,A\vdash B$
\item $\Gamma,A\vdash A+B$ 
\item $\Gamma,B\vdash A+B$
\item $\Gamma,A+B,A\rightarrow C,B\rightarrow C\vdash C$
\item $\Gamma\vdash1$
\item $\Gamma,A\vdash A$
\end{itemize}
%
\end{minipage}%
\begin{minipage}[t]{0.49\columnwidth}%
\[
\frac{\Gamma,A\vdash B}{\Gamma\vdash A\rightarrow B}
\]
\[
\frac{\Gamma\vdash G}{\Gamma,D\vdash G}
\]
\[
\frac{\Gamma,A,B\vdash G}{\Gamma,B,A\vdash G}
\]
%
\end{minipage}

\medskip{}
Can we use these rules to obtain a finite and complete search tree?
No.
\begin{itemize}
\item Try proving $A,B+C\vdash A\times B+C$: cannot find matching rules
\begin{itemize}
\item Need a better formulation of the logic
\end{itemize}
\end{itemize}
\end{frame}

\begin{frame}{Proof search II: Gentzen's calculus LJ (1935)}

\begin{itemize}
\item A ``complete and sound calculus'' is a set of axioms and derivation
rules that will yield all (and only!) theorems of the logic
\begin{align*}
\text{(}A\text{ is atomic)\,}\frac{}{\Gamma,{\color{blue}A}\vdash A}\:Id & \qquad\frac{}{\Gamma\vdash{\color{blue}\top}}\,\top\\
\frac{\Gamma,A\rightarrow B\vdash A\quad\;\Gamma,B\vdash C}{\Gamma,{\color{blue}A\rightarrow B}\vdash C}\:L_{\rightarrow} & \qquad\frac{\Gamma,A\vdash B}{\Gamma\vdash{\color{blue}A\rightarrow B}}\,R_{\rightarrow}\\
\frac{\Gamma,A\vdash C\quad\;\Gamma,B\vdash C}{\Gamma,{\color{blue}A+B}\vdash C}\:L_{+} & \qquad\frac{\Gamma\vdash A_{i}}{\Gamma\vdash{\color{blue}A_{1}+A_{2}}}\,R_{+_{i}}\\
\frac{\Gamma,A_{i}\vdash C}{\Gamma,{\color{blue}A_{1}\times A_{2}}\vdash C}\:L_{\times_{i}} & \qquad\frac{\Gamma\vdash A\quad\;\Gamma\vdash B}{\Gamma\vdash{\color{blue}A\times B}}\,R_{\times}
\end{align*}
\item Two axioms and eight derivation rules
\begin{itemize}
\item Each derivation rule says: The sequent at bottom will be proved if
proofs are given for sequent(s) at top
\item The symbol $\Gamma$ means ``any number of premises, or $\emptyset$''
\end{itemize}
\item Use these rules ``bottom-up'' to perform a proof search
\begin{itemize}
\item Sequents are nodes and proofs are edges in the tree of proof search
\end{itemize}
\end{itemize}
\end{frame}

\begin{frame}{Proof search example I}

Example: to prove $\forall R.\,\forall Q.\,\left(\left(R\rightarrow R\right)\rightarrow Q\right)\rightarrow Q$
\begin{itemize}
\item Root sequent $S_{0}:\emptyset\vdash\left(\left(R\rightarrow R\right)\rightarrow Q\right)\rightarrow Q$
\item $S_{0}$ with rule $R_{\rightarrow}$ yields $S_{1}:\left(R\rightarrow R\right)\rightarrow Q\vdash Q$
\item $S_{1}$ with rule $L_{\rightarrow}$ yields $S_{2}:\left(R\rightarrow R\right)\rightarrow Q\vdash R\rightarrow R$
and $S_{3}:Q\vdash Q$
\item Sequent $S_{3}$ follows from the $Id$ axiom; it remains to prove
$S_{2}$
\item $S_{2}$ with rule $L_{\rightarrow}$ yields $S_{4}:\left(R\rightarrow R\right)\rightarrow Q\vdash R\rightarrow R$
and $S_{5}:Q\vdash R\rightarrow R$
\begin{itemize}
\item We are stuck here because $S_{4}=S_{2}$ (we are in a loop)
\item We can prove $S_{5}$ but that will not help
\item So we backtrack (erase $S_{4}$, $S_{5}$) and apply another rule
to $S_{2}$
\end{itemize}
\item $S_{2}$ with rule $R_{\rightarrow}$ yields $S_{6}:\left(R\rightarrow R\right)\rightarrow Q;R\vdash R$
\item Sequent $S_{6}$ follows from the $Id$ axiom
\end{itemize}
Therefore we have proved $S_{0}$

Since $\left(\left(R\rightarrow R\right)\rightarrow Q\right)\rightarrow Q$
is derived from no premises, it is a theorem

$Q.E.D.$
\end{frame}

\begin{frame}{Proof search III: The calculus LJT}


\framesubtitle{Gentzen-Vorobieff-Hudelmaier-Dyckhoff, 1935--1990}
\begin{itemize}
\item The Gentzen calculus LJ will loop if rule $L_{\rightarrow}$ is applied
$\geq2$ times
\item The calculus LJT keeps all rules of LJ except rule $L_{\rightarrow}$
\item Replace rule $L_{\rightarrow}$ by pattern-matching on $A$ in the
premise $A\rightarrow B$:
\begin{align*}
\text{(if }A\text{ is atomic)\,}\frac{\Gamma,A,B\vdash D}{\Gamma,A,{\color{blue}A\rightarrow B}\vdash D}\:L_{\rightarrow_{1}}\\
\frac{\Gamma,A\rightarrow B\rightarrow C\vdash D}{\Gamma,{\color{blue}(A\times B)\rightarrow C}\vdash D}\:L_{\rightarrow_{2}}\\
\frac{\Gamma,A\rightarrow C,B\rightarrow C\vdash D}{\Gamma,{\color{blue}(A+B)\rightarrow C}\vdash D}\:L_{\rightarrow_{3}}\\
\frac{\Gamma,B\rightarrow C\vdash A\rightarrow B\quad\quad\Gamma,C\vdash D}{\Gamma,{\color{blue}(A\rightarrow B)\rightarrow C}\vdash D}\:L_{\rightarrow_{4}}
\end{align*}
\item When using LJT rules, the proof tree has no loops and terminates
\begin{itemize}
\item See \href{http://citeseer.ist.psu.edu/viewdoc/summary?doi=10.1.1.35.2618}{this paper}
for an explicit decreasing measure on the proof tree
\end{itemize}
\end{itemize}
\end{frame}

\begin{frame}{Proof search IV: From deduction rules to code}

\begin{itemize}
\item The new rules are equivalent to the old rules, therefore...
\begin{itemize}
\item Proof of a sequent $A,B,C\vdash G$ $\Leftrightarrow$ code snippet
$t(a,b,c):G$
\item Also can be seen as a function $t$ from $A,B,C$ to $G$
\end{itemize}
\item Sequent in a proof follows from an axiom or from a transforming rule
\begin{itemize}
\item The two axioms are fixed expressions, $x^{:A}\rightarrow x$ and $1$
\item Each rule has a \emph{proof transformer} function: $\text{PT}_{R_{\rightarrow}}$
, $\text{PT}_{L_{+}}$ , etc.
\end{itemize}
\item Examples of proof transformer functions:
\end{itemize}
\[
\frac{\Gamma,A\vdash C\quad\;\Gamma,B\vdash C}{\Gamma,{\color{blue}A+B}\vdash C}\:L_{+}
\]
\vspace{-1\baselineskip}
\[
PT_{L_{+}}(t_{1}^{:A\rightarrow C},t_{2}^{:B\rightarrow C})=x^{:A+B}\rightarrow\ x\ \text{match}\begin{cases}
a^{:A}\rightarrow t_{1}(a)\\
b^{:B}\rightarrow t_{2}(b)
\end{cases}=\,\begin{array}{|c||c|}
 & C\\
\hline A & t_{1}\\
B & t_{2}
\end{array}
\]
\vspace{-0.4\baselineskip}
\begin{align*}
\frac{\Gamma,A\rightarrow B\rightarrow C\vdash D}{\Gamma,{\color{blue}(A\times B)\rightarrow C}\vdash D}\:L_{\rightarrow_{2}}\\
PT_{L_{\rightarrow_{2}}}(f^{:\left(A\rightarrow B\rightarrow C\right)\rightarrow D})=g^{:A\times B\rightarrow C}\rightarrow & f\,(x^{:A}\rightarrow y^{:B}\rightarrow g(x,y))
\end{align*}

\begin{itemize}
\item Verify that we can indeed produce PTs for every rule of LJT
\end{itemize}
\end{frame}

\begin{frame}{Proof search example II: code inference}

Once a proof tree is found, start from leaves and apply PTs
\begin{itemize}
\item For each sequent $S_{i}$, this will derive a \textbf{proof expression}
$t_{i}$
\item Example: to prove $S_{0}$, start from $S_{6}$ backwards:{\footnotesize{}
\begin{align*}
S_{6}:\left(R\rightarrow R\right)\rightarrow Q;R\vdash R\quad(\text{axiom }Id)\quad & \text{\texttt{def t6(rrq, r) = r}}\\
S_{2}:\left(R\rightarrow R\right)\rightarrow Q\vdash\left(R\rightarrow R\right)\quad\text{PT}_{R_{\rightarrow}}(t_{6})\quad & \text{\texttt{def t2(rrq) = r => t6(rrq, r)}}\\
S_{3}:Q\vdash Q\quad(\text{axiom }Id)\quad & \text{\texttt{def t3(q) = q}}\\
S_{1}:\left(R\rightarrow R\right)\rightarrow Q\vdash Q\quad\text{PT}_{L_{\rightarrow}}(t_{2},t_{3})\quad & \text{\texttt{def t1(rrq) = t3(rrq(t2(rrq)))}}\\
S_{0}:\emptyset\vdash\left(\left(R\rightarrow R\right)\rightarrow Q\right)\rightarrow Q\quad\text{PT}_{R_{\rightarrow}}(t_{1})\quad & \text{\texttt{def t0 = rrq => t1(rrq)}}
\end{align*}
}{\footnotesize\par}
\item The proof expression for $S_{0}$ is then obtained as
\begin{align*}
\text{\texttt{def t0 = \{ rrq => t3(rrq(t2(rrq))) \}}}\\
\text{\texttt{= \{ rrq => rrq(r => t6(rrq, r)) \}}}\\
\text{\texttt{= \{ rrq => rrq(r => r) \}}}
\end{align*}
Simplified final code having the required type: 
\end{itemize}
\begin{lyxcode}
\textcolor{blue}{\footnotesize{}def~t0{[}R,~Q{]}:~((R~=>~R)~=>~Q)~=>~Q~=~\{~x~=>~x(y~=>~y)~\}}{\footnotesize\par}
\end{lyxcode}
\end{frame}

\begin{frame}{Using the \texttt{curryhoward} library for code inference}

Two main use cases:
\begin{enumerate}
\item Define a type signature and derive an implementation automatically
\end{enumerate}
\begin{lyxcode}
\textcolor{blue}{\footnotesize{}def~map{[}E,~A,~B{]}(reader:~E~=>~A,~f:~A~=>~B):~E~=>~B~=~implement}{\footnotesize\par}
\end{lyxcode}
\begin{enumerate}
\item Automatically build an expression from previously computed values
\end{enumerate}
\begin{lyxcode}
\textcolor{blue}{\footnotesize{}val~f(a:~String,~b:~Boolean):~Int~=~\{...\}}{\footnotesize\par}

\textcolor{blue}{\footnotesize{}case~class~Result(x:~Int,~name:~String)}{\footnotesize\par}

\textcolor{blue}{\footnotesize{}val~result~=~ofType{[}Result{]}(\textquotedbl abc\textquotedbl ,~f,~true)}{\footnotesize\par}
\end{lyxcode}
Fixed types (\texttt{\textcolor{blue}{\footnotesize{}Int}}, \texttt{\textcolor{blue}{\footnotesize{}String}},
etc.) are treated as type parameters
\begin{itemize}
\item This is a practical application of the Curry-Howard correspondence
\item The CH correspondence works only for ``fully parametric'' code
\end{itemize}
\end{frame}

\begin{frame}{Summary}

\begin{itemize}
\item The CH correspondence maps the type system of each programming language
into a certain system of logical propositions 
\item Proof of logical propositions corresponds to implementation of the
type
\item If the logic of types is decidable, we can automatically produce code
from type signatures
\item Simple fully parametric code corresponds to IPL, which is decidable
\item Algorithms exist for proof search (and for disproof search) in IPL
\begin{itemize}
\item See the book by R.~Bornat: \emph{Proof and Disproof in Formal Logic}
(2005)
\end{itemize}
\item The CH correspondence provides powerful type-directed reasoning about
code, as long as we work with fully parametric functions
\item Software engineers need to develop intuition for this reasoning
\begin{itemize}
\item The decision procedures are not easy to use 
\item Software can help, e.g., the \texttt{\textcolor{blue}{\footnotesize{}curryhoward}}
library
\end{itemize}
\item CH is a mathematical theory that helps programmers write code
\begin{itemize}
\item Functional programming is an engineering discipline
\end{itemize}
\end{itemize}
\end{frame}

\end{document}
