\batchmode
\makeatletter
\def\input@path{{/Users/sergei.winitzki/Code/talks/ftt-fp/}}
\makeatother
\documentclass[english]{beamer}
\usepackage[T1]{fontenc}
\usepackage[latin9]{inputenc}
\setcounter{secnumdepth}{3}
\setcounter{tocdepth}{3}
\usepackage{babel}
\usepackage{amstext}
\usepackage{wasysym}
\usepackage[all]{xy}
\ifx\hypersetup\undefined
  \AtBeginDocument{%
    \hypersetup{unicode=true,pdfusetitle,
 bookmarks=true,bookmarksnumbered=false,bookmarksopen=false,
 breaklinks=false,pdfborder={0 0 1},backref=false,colorlinks=true}
  }
\else
  \hypersetup{unicode=true,pdfusetitle,
 bookmarks=true,bookmarksnumbered=false,bookmarksopen=false,
 breaklinks=false,pdfborder={0 0 1},backref=false,colorlinks=true}
\fi

\makeatletter
%%%%%%%%%%%%%%%%%%%%%%%%%%%%%% Textclass specific LaTeX commands.
% this default might be overridden by plain title style
\newcommand\makebeamertitle{\frame{\maketitle}}%
% (ERT) argument for the TOC
\AtBeginDocument{%
  \let\origtableofcontents=\tableofcontents
  \def\tableofcontents{\@ifnextchar[{\origtableofcontents}{\gobbletableofcontents}}
  \def\gobbletableofcontents#1{\origtableofcontents}
}
\newenvironment{lyxcode}
  {\par\begin{list}{}{
    \setlength{\rightmargin}{\leftmargin}
    \setlength{\listparindent}{0pt}% needed for AMS classes
    \raggedright
    \setlength{\itemsep}{0pt}
    \setlength{\parsep}{0pt}
    \normalfont\ttfamily}%
   \def\{{\char`\{}
   \def\}{\char`\}}
   \def\textasciitilde{\char`\~}
   \item[]}
  {\end{list}}

%%%%%%%%%%%%%%%%%%%%%%%%%%%%%% User specified LaTeX commands.
\usetheme[secheader]{Boadilla}
\usecolortheme{seahorse}
\title[Chapter 11: Monad transformers]{Chapter 11:
Computations in a functor context III}
\subtitle{Monad transformers}
\author{Sergei Winitzki}
\date{2019-01-05}
\institute[ABTB]{Academy by the Bay}
\setbeamertemplate{headline}{} % disable headline at top
\setbeamertemplate{navigation symbols}{} % disable navigation bar at bottom
\usepackage[all]{xy} % xypic
\usepackage[nocenter]{qtree} % simple tree drawing
\usepackage{relsize} % make math symbols larger or smaller
\newcommand{\bef}{\ensuremath\raisebox{2pt}{$\mathsmaller{\mathsmaller{\circ}}$}\hspace{-3.3pt},}
%\makeatletter
% Macros to assist LyX with XYpic when using scaling.
\newcommand{\xyScaleX}[1]{%
\makeatletter
\xydef@\xymatrixcolsep@{#1}
\makeatother
} % end of \xyScaleX
\makeatletter
\newcommand{\xyScaleY}[1]{%
\makeatletter
\xydef@\xymatrixrowsep@{#1}
\makeatother
} % end of \xyScaleY

\makeatother

\begin{document}
\frame{\titlepage}
\begin{frame}{Computations within a functor context: Combining monads}

Programs often need to combine monadic effects
\begin{itemize}
\item ``Effect'' $\equiv$ what else happens in {\footnotesize{}$A\Rightarrow M^{B}$}
besides computing $B$ from $A$
\item Examples of effects for some standard monads:
\begin{itemize}
\item \texttt{\textcolor{blue}{\footnotesize{}Option}} -- computation will
have no result or a single result
\item \texttt{\textcolor{blue}{\footnotesize{}List}} -- computation will
have zero, one, or multiple results
\item \texttt{\textcolor{blue}{\footnotesize{}Either}} -- computation may
fail to obtain its result, reports error
\item \texttt{\textcolor{blue}{\footnotesize{}Reader}} -- computation needs
to read an external context value
\item \texttt{\textcolor{blue}{\footnotesize{}Writer}} -- some value will
be appended to a (monoidal) accumulator
\item \texttt{\textcolor{blue}{\footnotesize{}Future}} -- computation will
be scheduled to run later
\end{itemize}
\item How to combine several effects in the same functor block (\texttt{\textcolor{blue}{\footnotesize{}for}}/\texttt{\textcolor{blue}{\footnotesize{}yield}})?
\end{itemize}
{\footnotesize{}\vspace{-0.35cm}}\texttt{\textcolor{blue}{\footnotesize{}}}%
\begin{minipage}[t]{0.49\columnwidth}%
\begin{lyxcode}
\textcolor{darkgray}{\footnotesize{}//~This~is~not~valid~Scala!}{\footnotesize\par}

\textcolor{blue}{\footnotesize{}val~result~=~for~\{~i~$\leftarrow$~1~to~n}{\footnotesize\par}

\textcolor{blue}{\footnotesize{}~~~j~$\leftarrow$~Future~\{~q(i)~\}}{\footnotesize\par}

\textcolor{blue}{\footnotesize{}~~~k~$\leftarrow$~maybeError(j)~:~Try{[}Int{]}}{\footnotesize\par}

\textcolor{blue}{\footnotesize{}\}~yield~f(k)}{\footnotesize\par}

\textcolor{darkgray}{\footnotesize{}//~What~should~be~the~type~of~result??}{\footnotesize\par}
\end{lyxcode}
%
\end{minipage}\texttt{\textcolor{blue}{\footnotesize{}~ ~ ~}}%
\begin{minipage}[t]{0.49\columnwidth}%
\begin{lyxcode}
\textcolor{blue}{\footnotesize{}~~}\textcolor{darkgray}{\footnotesize{}//~This~is~not~valid~Scala!}{\footnotesize\par}

\textcolor{blue}{\footnotesize{}(1~to~n).flatMap~\{~i~$\Rightarrow$}{\footnotesize\par}

\textcolor{blue}{\footnotesize{}~~~Future(q(i)).flatMap~\{~j~$\Rightarrow$}{\footnotesize\par}

\textcolor{blue}{\footnotesize{}~~~~~maybeError(j).map~\{~k~$\Rightarrow$}{\footnotesize\par}

\textcolor{blue}{\footnotesize{}~~~~~~~f(k)}{\footnotesize\par}

\textcolor{blue}{\footnotesize{}~~~~~~~~~\}\}\}}{\footnotesize\par}
\end{lyxcode}
%
\end{minipage}\texttt{\textcolor{blue}{\footnotesize{}\medskip{}
}}{\footnotesize\par}
\begin{itemize}
\item The code will work if we ``unify'' all effects in a new, larger
monad
\item Need to compute the type of new monad that contains all given effects
\end{itemize}
\end{frame}

\begin{frame}{Combining monadic effects I. Trial and error}

There are several ways of combining two monads into a new monad:
\begin{itemize}
\item If $M_{1}^{A}$ and $M_{2}^{A}$ are monads then $M_{1}^{A}\times M_{2}^{A}$
is also a monad
\begin{itemize}
\item But $M_{1}^{A}\times M_{2}^{A}$ describes two separate values with
two separate effects
\end{itemize}
\item If $M_{1}^{A}$ and $M_{2}^{A}$ are monads then $M_{1}^{A}+M_{2}^{A}$
is usually not a monad
\begin{itemize}
\item If it worked, it would be a choice between two different values /
effects
\end{itemize}
\item If $M_{1}^{A}$ and $M_{2}^{A}$ are monads then one of $M_{1}^{M_{2}^{A}}$
or $M_{2}^{M_{1}^{A}}$ is often a monad
\item Examples and counterexamples for functor composition:
\begin{itemize}
\item Combine $Z\Rightarrow A$ and $\text{List}^{A}$ as $Z\Rightarrow\text{List}^{A}$
\item Combine \texttt{\textcolor{blue}{\footnotesize{}Future{[}A{]}}} and
\texttt{\textcolor{blue}{\footnotesize{}Option{[}A{]}}} as \texttt{\textcolor{blue}{\footnotesize{}Future{[}Option{[}A{]}{]}}} 
\item But \texttt{\textcolor{blue}{\footnotesize{}Either{[}Z, Future{[}A{]}{]}}}
and \texttt{\textcolor{blue}{\footnotesize{}Option{[}Z $\Rightarrow$
A{]}}} are not monads
\item Neither \texttt{\textcolor{blue}{\footnotesize{}Future{[}State{[}A{]}{]}}}
nor \texttt{\textcolor{blue}{\footnotesize{}State{[}Future{[}A{]}{]}}}
are monads
\end{itemize}
\item The order of effects matters when composition works both ways: 
\begin{itemize}
\item Combine \texttt{\textcolor{blue}{\footnotesize{}Either}} ($M_{1}^{A}=Z+A$)
and \texttt{\textcolor{blue}{\footnotesize{}Writer}} ($M_{2}^{A}=W\times A$) 
\begin{itemize}
\item as $Z+W\times A$ -- either compute result and write a message, or
all fails
\item as $\left(Z+A\right)\times W$ -- message is always written, but
computation may fail
\end{itemize}
\end{itemize}
\item Find a general way of defining a new monad with combined effects
\item Derive properties required for the new monad
\end{itemize}
\end{frame}

\begin{frame}{Combining monadic effects II. Lifting into a larger monad}

{\footnotesize{}\vspace{-0.15cm}}If a ``big monad'' \texttt{\textcolor{blue}{\footnotesize{}BigM{[}A{]}}}
\emph{somehow} combines all the needed effects:

{\footnotesize{}\vspace{-0.15cm}\hspace{-0.35cm}}\texttt{\textcolor{blue}{\footnotesize{}}}%
\begin{minipage}[t]{0.49\columnwidth}%
\begin{lyxcode}
\textrm{\textcolor{darkgray}{\footnotesize{}//~This~could~be~valid~Scala...}}{\footnotesize\par}

\textcolor{blue}{\footnotesize{}val~result:~BigM{[}Int{]}~=~for~\{}{\footnotesize\par}

\textcolor{blue}{\footnotesize{}~~~i~$\leftarrow$~lift$_{1}$(1~to~n)}{\footnotesize\par}

\textcolor{blue}{\footnotesize{}~~~j~$\leftarrow$~lift$_{2}$(Future\{~q(i)~\})}{\footnotesize\par}

\textcolor{blue}{\footnotesize{}~~~k~$\leftarrow$~lift$_{3}$(maybeError(j))}{\footnotesize\par}

\textcolor{blue}{\footnotesize{}\}~yield~f(k)}{\footnotesize\par}
\end{lyxcode}
%
\end{minipage}\texttt{\textcolor{blue}{\footnotesize{} }}%
\begin{minipage}[t]{0.49\columnwidth}%
\begin{lyxcode}
\textcolor{blue}{\footnotesize{}~}\textrm{\textcolor{darkgray}{\footnotesize{}//~If~we~define~the~various}}{\footnotesize\par}

\textrm{\textcolor{darkgray}{\footnotesize{}~//~required~``lifting''~functions:}}{\footnotesize\par}

\textcolor{blue}{\footnotesize{}def~lift$_{1}${[}A{]}:~Seq{[}A{]}~$\Rightarrow$~BigM{[}A{]}~=~???}{\footnotesize\par}

\textcolor{blue}{\footnotesize{}def~lift$_{2}${[}A{]}:~Future{[}A{]}~$\Rightarrow$~BigM{[}A{]}~=~???}{\footnotesize\par}

\textcolor{blue}{\footnotesize{}def~lift$_{3}${[}A{]}:~Try{[}A{]}~$\Rightarrow$\vspace{-0.2cm}}Combine $Z\Rightarrow A$ and $1+A$:
only $Z\Rightarrow1+A$ works, not $1+\left(Z\Rightarrow A\right)$
\begin{itemize}
\item It is not possible to combine monads via a natural bifunctor $B^{M_{1},M_{2}}$
\item It is not possible to combine arbitrary monads as $M_{1}^{M_{2}^{\bullet}}$
or $M_{2}^{M_{1}^{\bullet}}$
\end{itemize}
\item The trick: for a fixed \textbf{base }monad $L^{\bullet}$, let $M^{\bullet}$
(\textbf{foreign }monad) vary
\item Call the desired result the ``$L$'s monad transformer'', $T_{L}^{M,A}$
\begin{itemize}
\item (We don't yet have a general formula for monad transformers)
\end{itemize}
\end{itemize}
{\footnotesize{}\vspace{-0.1cm}}A \textbf{monad transformer} for
a \textbf{base} monad $L^{\bullet}$ is a type constructor $T_{L}^{M,\bullet}$
parameterized by a monad $M^{\bullet}$, such that for all monads
$M^{\bullet}$
\begin{itemize}
\item $T_{L}^{M,\bullet}$ is a monad (the monad $M$ \textbf{transformed
with} $T_{L}$)
\item ``Lifting'' -- a monadic morphism $\text{lift}_{L}^{M}:M^{A}\leadsto T_{L}^{M,A}$,
natural in $M^{\bullet}$
\item ``Injection'' -- a monadic morphism $\text{inj}:L^{A}\leadsto T_{L}^{M,A}$ 
\item $T_{L}^{M,\bullet}$ is \textbf{monadically natural} in $M^{\bullet}$
\begin{itemize}
\item $T_{L}^{M,\bullet}$ is natural w.r.t.~a monadic functor $M^{\bullet}$
as a type parameter
\item For any monad $N^{\bullet}$ and a monadic morphism $f:M^{\bullet}\leadsto N^{\bullet}$
we need to have a monadic morphism $T_{L}^{M,\bullet}\leadsto T_{L}^{N,\bullet}$
for the transformed monads
\item If we implement $T_{L}^{M,\bullet}$ only via $M$'s monad methods,
naturality will hold 
\item Cf.~\texttt{\textcolor{blue}{\footnotesize{}traverse}}{\small{}$:L^{A}\Rightarrow\left(A\Rightarrow F^{B}\right)\Rightarrow F^{L^{B}}$
-- natural w.r.t.~applicative $F^{\bullet}$}{\small\par}
\end{itemize}
\end{itemize}
\end{frame}

\begin{frame}{Monad transformers II: First examples}

Recall these monad constructions:
\begin{itemize}
\item If $M^{A}$ is a monad then $R\Rightarrow M^{A}$ is also a monad
(for a fixed type $R$)
\item If $M^{A}$ is a monad then $M^{Z+A\times W}$ is also a monad (for
fixed $W,$ $Z$)
\end{itemize}
This gives the monad transformers for \texttt{\textcolor{blue}{\footnotesize{}Reader}},
\texttt{\textcolor{blue}{\footnotesize{}Writer}}, \texttt{\textcolor{blue}{\footnotesize{}Either}}
base monads:
\begin{lyxcode}
\textcolor{blue}{\footnotesize{}type~ReaderT{[}R,~M{[}\_{]},~A{]}~=~R~$\Rightarrow$~M{[}A{]}}{\footnotesize\par}

\textcolor{blue}{\footnotesize{}type~EitherT{[}Z,~M{[}\_{]},~A{]}~=~M{[}Either{[}Z,~A{]}{]}}{\footnotesize\par}

\textcolor{blue}{\footnotesize{}type~WriterT{[}W,~M{[}\_{]},~A{]}~=~M{[}(W,~A){]}}{\footnotesize\par}
\end{lyxcode}
\begin{itemize}
\item {\footnotesize{}\vspace{-0.2cm}}\texttt{\textcolor{blue}{\footnotesize{}ReaderT}}
wraps the foreign monad from the outside
\item \texttt{\textcolor{blue}{\footnotesize{}EitherT}} and \texttt{\textcolor{blue}{\footnotesize{}WriterT}}
require the foreign monad to wrap \emph{them}
\end{itemize}
Remaining questions:
\begin{itemize}
\item What are transformers for other standard monads (\texttt{\textcolor{blue}{\footnotesize{}List}},
\texttt{\textcolor{blue}{\footnotesize{}State}}, \texttt{\textcolor{blue}{\footnotesize{}Cont}})?
\begin{itemize}
\item ...in fact, these monads do not compose as either ``inside'' or
``outside''
\end{itemize}
\item How to derive a monad transformer for an arbitrary given monad?
\begin{itemize}
\item For monads obtained via known monad constructions?
\item For monads constructed via other monad transformers?
\end{itemize}
\item For a given monad, is the corresponding monad transformer unique?
\end{itemize}
\end{frame}

\begin{frame}{Monad transformers III: The zoology}

Need to select the correct monad transformer construction, per monad:
\begin{itemize}
\item ``Inside'' transformers: base monad inside foreign monad, $T_{L}^{M,A}=M^{L^{A}}$
\begin{itemize}
\item Examples: \texttt{\textcolor{blue}{\footnotesize{}OptionT}}, \texttt{\textcolor{blue}{\footnotesize{}WriterT}},
\texttt{\textcolor{blue}{\footnotesize{}EitherT}} 
\end{itemize}
\item ``Outside'' transformers: base monad is outside, $T_{L}^{M,A}=L^{M^{A}}$
\begin{itemize}
\item Examples: \texttt{\textcolor{blue}{\footnotesize{}ReaderT}} 
\end{itemize}
\item ``Recursive'': interleaves the base monad and the foreign monad
\begin{itemize}
\item Examples: \texttt{\textcolor{blue}{\footnotesize{}ListT}}, \texttt{\textcolor{blue}{\footnotesize{}FreeMonadT}} 
\end{itemize}
\item ``Irregular'': none of the above constructions apply
\begin{itemize}
\item Examples: \texttt{\textcolor{blue}{\footnotesize{}StateT}}, \texttt{\textcolor{blue}{\footnotesize{}ContT}}{\footnotesize\par}
\end{itemize}
\end{itemize}
\end{frame}

\begin{frame}{Exercises}
\begin{enumerate}
\item {\small{}Show that the method }\texttt{\textcolor{blue}{\footnotesize{}pure}}{\small{}$:A\Rightarrow M^{A}$
is a monadic morphism between monads $\text{Id}^{A}\equiv A$ and
$M^{A}$.}{\small\par}
\item {\small{}Show that $M_{1}^{A}+M_{2}^{A}$ is }\emph{\small{}not}{\small{}
a monad when $M_{1}^{A}\equiv1+A$ and $M_{2}^{A}\equiv Z\Rightarrow A$.}{\small\par}
\end{enumerate}
\end{frame}

\end{document}
