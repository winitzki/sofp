
\chapter{Mathematical formulas as code. I. Nameless functions\label{chap:1-Values,-types,-expressions,}}

\section{Translating mathematics into code}

\subsection{First examples}

We begin by implementing some computational tasks in Scala.

\subsubsection{Example \label{subsec:Example-Factorial-of-10}\ref{subsec:Example-Factorial-of-10}}

Find the product of integers from $1$ to $10$ (the \textbf{factorial\index{factorial function}}
of 10, usually denoted by $10!$).

\subparagraph{Solution}

First, we write a mathematical formula for the result:
\[
10!=1*2*...*10\quad,\quad\quad\text{or in mathematical notation}:\quad10!=\prod_{k=1}^{10}k\quad.
\]
We can then write Scala code in a way that resembles the last formula:
\begin{lstlisting}
scala> (1 to 10).product
res0: Int = 3628800
\end{lstlisting}

The syntax \lstinline!(1 to 10)! produces a sequence of integers
from \lstinline!1! to \lstinline!10!. The \lstinline!product! method
computes the product of the numbers in the sequence.

The code \texttt{}\lstinline!(1 to 10).product! is an \textbf{expression}\index{expression},
which means that (1) the code can be evaluated and yields a value,
and (2) the code can be used inside a larger expression. For example,
we could write:
\begin{lstlisting}
scala> 100 + (1 to 10).product + 100   // The code `(1 to 10).product` is a sub-expression.
res0: Int = 3629000

scala> 3628800 == (1 to 10).product 
res1: Boolean = true
\end{lstlisting}
The Scala interpreter indicates that the result of \texttt{}\lstinline!(1 to 10).product!
is a value $3628800$ of type \lstinline!Int!. If we need to define
a name for that value, we use the \textsf{``}\lstinline!val!\textsf{''} syntax:
\begin{lstlisting}
scala> val fac10 = (1 to 10).product
fac10: Int = 3628800
\end{lstlisting}


\subsubsection{Example \label{subsec:Example-Factorial-as-a-function}\ref{subsec:Example-Factorial-as-a-function}}

Define a function to compute the factorial of an integer $n$.

A mathematical formula for this function can be written as:
\[
f\left(n\right)=\prod_{k=1}^{n}k\quad.
\]
The corresponding Scala code is:
\begin{lstlisting}
def f(n: Int) = (1 to n).product
\end{lstlisting}

In Scala\textsf{'}s \texttt{}\lstinline!def! syntax, we need to specify the
type of a function\textsf{'}s argument; here, we wrote \lstinline!n: Int!.
In the usual mathematical notation, types of function arguments are
either not written at all, or written separately from the formula:
\begin{equation}
f(n)=\prod_{k=1}^{n}k\quad,\quad\quad\forall n\in\mathbb{N}\quad.\label{eq:factorial-as-function}
\end{equation}
Equation~(\ref{eq:factorial-as-function}) indicates that $n$ must
be from the set of positive integers, denoted by $\mathbb{N}$ in
mathematics. This is similar to specifying the type \texttt{(}\lstinline!n: Int!\texttt{)}
in the Scala code. So, the argument\textsf{'}s type in the code specifies the
\textbf{domain} of a function\index{domain of a function} (the set
of admissible values of a a function\textsf{'}s argument).

Having defined the function \lstinline!f!, we can now apply it to
an integer value \lstinline!10! (or, as programmers say, \textsf{``}call\textsf{''}
the function \lstinline!f! with argument \lstinline!10!):
\begin{lstlisting}
scala> f(10)
res6: Int = 3628800
\end{lstlisting}
It is a \textbf{type error}\index{type error} to apply \lstinline!f!
to a non-integer value:
\begin{lstlisting}
scala> f("abc")
<console>:13: error: type mismatch;
 found   : String("abc")
 required: Int
\end{lstlisting}


\subsection{Nameless functions\label{subsec:Nameless-functions}}

Both the code written above and Eq.~(\ref{eq:factorial-as-function})
involve \emph{naming} the function as \textsf{``}$f$\textsf{''}. Sometimes a function
does not really need a name, \textemdash{} say, if the function is
used only once. \textsf{``}Nameless\textsf{''} mathematical functions may be denoted
using the symbol $\rightarrow$ (pronounced \textsf{``}maps to\textsf{''}) like this:\footnote{In mathematics, an often used symbol for \textsf{``}maps to\textsf{''} is $\mapsto$,
but this book uses a simpler arrow symbol ($\rightarrow$) that is
visually similar.} 
\[
x\rightarrow\left(\text{some formula}\right)\quad.
\]
So, a mathematical notation for the nameless factorial function is:
\[
n\rightarrow\prod_{k=1}^{n}k\quad.
\]
This reads as \textsf{``}a function that maps $n$ to the product of all $k$
where $k$ goes from $1$ to $n$\textsf{''}. The Scala expression implementing
this mathematical formula is:
\begin{lstlisting}
(n: Int) => (1 to n).product
\end{lstlisting}
This expression shows Scala\textsf{'}s syntax for a \textbf{nameless function}\index{nameless function}.
Here, \lstinline!n: Int! is the function\textsf{'}s \textbf{argument} variable,\footnote{In computer science, argument variables are called \textsf{``}parameters\textsf{''}
of a function, while an \textsf{``}argument\textsf{''} is a value to which the function
is actually applied. This book uses the word \textsf{``}argument\textsf{''} for both,
following the mathematical usage.} while \lstinline!(1 to n).product! is the function\textsf{'}s \textbf{body}.
The function arrow (\lstinline!=>!) separates the argument variable
from the body.\footnote{Some programming languages use the symbols \lstinline!->! or \lstinline!=>!
for the function arrow; see Table~\ref{lambda-functions-table}.} 

Functions in Scala (whether named or nameless) are treated as values\index{function as a value},
which means that we can also define a Scala value as:
\begin{lstlisting}
scala> val fac = (n: Int) => (1 to n).product
fac: Int => Int = <function1>
\end{lstlisting}
We see that the value \lstinline!fac! has the type \lstinline!Int => Int!,
which means that the function \lstinline!fac! takes an integer (\lstinline!Int!)
argument and returns an integer result value. What is the value of
the function \lstinline!fac! \emph{itself}? As we have just seen,
the standard Scala interpreter prints \lstinline!<function1>! as
the \textsf{``}value\textsf{''} of \lstinline!fac!. Another Scala interpreter called
\lstinline!ammonite!\footnote{See \texttt{\href{https://ammonite.io/}{https://ammonite.io/}}}
prints this:
\begin{lstlisting}
scala@ val fac = (n: Int) => (1 to n).product
fac: Int => Int = ammonite.$sess.cmd0$$$Lambda$1675/2107543287@1e44b638
\end{lstlisting}
The long number could indicate an address in memory. We may imagine
that a \textsf{``}function value\textsf{''} represents a block\emph{ }of compiled
code. That code will run and evaluate the function\textsf{'}s body whenever
the function is applied to an argument.

Once defined, a function can be applied to an argument value like
this:
\begin{lstlisting}
scala> fac(10)
res1: Int = 3628800
\end{lstlisting}
Functions can be also used without naming them. We may directly apply
a nameless factorial function to an integer argument \lstinline!10!
instead of writing \lstinline!fac(10)!:
\begin{lstlisting}
scala> ((n: Int) => (1 to n).product)(10)
res2: Int = 3628800
\end{lstlisting}

We would rarely write code like this. Instead of creating a nameless
function and then applying it right away to an argument, it is easier
to evaluate the expression symbolically by substituting \lstinline!10!
instead of \lstinline!n! in the function body:
\begin{lstlisting}
((n: Int) => (1 to n).product)(10) == (1 to 10).product
\end{lstlisting}

If a nameless function uses the argument several times, as in this
code:
\begin{lstlisting}
((n: Int) => n*n*n + n*n)(12345)
\end{lstlisting}
it is still easier to substitute the argument and to eliminate the
nameless function. We could write:
\begin{lstlisting}
12345*12345*12345 + 12345*12345
\end{lstlisting}
but, of course, it is better to avoid repeating the value \lstinline!12345!.
To achieve that, we may define \texttt{}\lstinline!n! as a value
in an \textbf{expression block\index{expression block}} like this:
\begin{lstlisting}
scala> { val n = 12345; n*n*n + n*n }
res3: Int = 322687002
\end{lstlisting}

Defined in this way, the value \lstinline!n! is visible only within
the expression block. Outside the block, another value named \lstinline!n!
could be defined independently of this \lstinline!n!. For this reason,
the definition of \lstinline!n! is called a \textbf{local-scope}\index{local scope}
definition.

Nameless functions are convenient when they are themselves arguments
of other functions, as we will see next.

\subsubsection{Example \label{subsec:Example-prime-numbers}\ref{subsec:Example-prime-numbers}}

Define a function that takes an integer argument $n$ and determines
whether $n$ is a prime number.

A simple mathematical formula for this function can be written using
the \textsf{``}forall\textsf{''} symbol ($\forall$) as:
\begin{equation}
\text{isPrime}\left(n\right)=\forall k\in\left[2,n-1\right].\ (n\%k)\neq0\quad.\label{eq:is_prime_def}
\end{equation}
This formula has two parts: first, a range of integers from $2$ to
$n-1$, and second, a requirement that all these integers $k$ should
satisfy the given condition: $(n\%k)\neq0$. Formula~(\ref{eq:is_prime_def})
is translated into Scala code as:
\begin{lstlisting}
def isPrime(n: Int) = (2 to n-1).forall(k => n % k != 0)
\end{lstlisting}
This code looks closely similar to the mathematical notation, except
for the arrow after $k$ that introduces a nameless function (\lstinline*k => n % k != 0*).
We do not need to specify the type \texttt{}\lstinline!Int! for
the argument \texttt{}\lstinline!k! of that nameless function. The
Scala compiler knows that \texttt{}\lstinline!k! is going to iterate
over the \emph{integer} elements of the range \lstinline!(2 to n-1)!,
which effectively forces \texttt{}\lstinline!k! to be of type \lstinline!Int!
because types must match.

We can now apply the function \lstinline!isPrime! to some integer
values:
\begin{lstlisting}
scala> isPrime(12)
res3: Boolean = false

scala> isPrime(13)
res4: Boolean = true
\end{lstlisting}
As we can see from the output above, the function \lstinline!isPrime!
returns a value of type \lstinline!Boolean!. Therefore, the function
\lstinline!isPrime! has type \lstinline!Int => Boolean!.

A function that returns a \lstinline!Boolean! value is called a \textbf{predicate}\index{predicate}.

In Scala, it is strongly recommended (although often not mandatory)
to specify the return types of named functions. The required syntax
looks like this:
\begin{lstlisting}
def isPrime(n: Int): Boolean = (2 to n-1).forall(k => n % k != 0)
\end{lstlisting}


\subsection{Nameless functions and bound variables}

The code for \texttt{}\lstinline!isPrime! differs from the mathematical
formula\ (\ref{eq:is_prime_def}) in two ways.

One difference is that the interval $\left[2,n-1\right]$ is in front
of \lstinline!forall!. Another is that the Scala code uses a nameless
function \lstinline*(k => n % k != 0)*, while Eq.~(\ref{eq:is_prime_def})
does not seem to use such a function.

To understand the first difference, we need to keep in mind that the
Scala syntax such as \texttt{}\lstinline!(2 to n-1).forall(k => ...)!
means to apply a function called \texttt{}\lstinline!forall! to
\emph{two} arguments: the first argument is the range \texttt{}\lstinline!(2 to n-1)!\texttt{,}
and the second argument is the nameless function \lstinline!(k => ...)!.
In Scala, the \textbf{method} syntax\index{method syntax} \lstinline!x.f(z)!,
and the equivalent \textbf{infix} syntax\index{infix syntax} \lstinline!x f z!,
means that a function \texttt{}\lstinline!f! is applied to its \emph{two}
arguments, \texttt{}\lstinline!x! and \lstinline!z!. In the ordinary
mathematical notation, this would be $f(x,z)$. Infix notation is
widely used when it is easier to read: for instance, we write $x+y$
rather than something like $plus\,(x,y)$.

A single-argument function could be also defined as a method, and
then the syntax is \lstinline!x.f!, as in the expression \texttt{}\lstinline!(1 to n).product!
shown before.

The methods \texttt{}\lstinline!product! and \texttt{}\lstinline!forall!
are already provided in the Scala standard library, so it is natural
to use them. If we want to avoid the method syntax, we could define
a function \texttt{}\lstinline!forAll! with two arguments and write
code like this:
\begin{lstlisting}
forAll(2 to n-1, k => n % k != 0)
\end{lstlisting}

This would bring the syntax closer to Eq.\ (\ref{eq:is_prime_def}).
However, there still remains the second difference: The symbol $k$
is used as an \emph{argument} of a nameless function \lstinline*(k => n % k != 0)*
in the Scala code, while the formula:
\begin{equation}
\forall k\in\left[2,n-1\right].\ (n\%k)\neq0\label{eq:prime-formula-function}
\end{equation}
does not seem to define such a function but defines the symbol $k$
that goes over the range $\left[2,n-1\right]$. The variable $k$
is then used for writing the predicate $(n\%k)\neq0$. 

Let us investigate the role of $k$ more closely. The mathematical
variable $k$ is accessible \emph{only inside} the expression \textsf{``}$\forall k.\,...$\textsf{''}
and makes no sense outside that expression. This becomes clear by
looking at Eq.\ (\ref{eq:is_prime_def}): The variable $k$ is not
present in the left-hand side and could not possibly be used there.
The name \textsf{``}$k$\textsf{''} is accessible only in the right-hand side, where
it is first mentioned as the arbitrary element $k\in\left[2,n-1\right]$
and then used in the sub-expression \textsf{``}$n\%k$\textsf{''}.

So, the mathematical notation in Eq.~(\ref{eq:prime-formula-function})
says two things: First, we use the name $k$ for integers from $2$
to $n-1$. Second, for each of those $k$ we evaluate the expression
$(n\%k)\neq0$, which can be viewed as a certain \emph{function} \emph{of}
$k$ that returns a \lstinline!Boolean! value. Translating the mathematical
notation into code, it is natural to use the nameless function $k\rightarrow(n\%k)\neq0$
and to write Scala code applying this nameless function to each element
of the range $\left[2,n-1\right]$ and checking that all result values
be \lstinline!true!:
\begin{lstlisting}
(2 to n-1).forall(k => n % k != 0)
\end{lstlisting}

Just as the mathematical notation defines the variable $k$ only in
the right-hand side of Eq.\ (\ref{eq:is_prime_def}), the argument
\lstinline!k! of the nameless Scala function \lstinline*k => n % k != 0*
is defined  within that function\textsf{'}s body and cannot be used in any
code outside the expression \lstinline*n % k != 0*.

Variables that are defined only inside an expression and are invisible
outside are called \textbf{bound variables}\index{bound variable|textit},
or \textsf{``}variables bound in an expression\textsf{''}. Variables that are used
in an expression but are defined outside it are called \textbf{free
variables}\index{free variable}, or \textsf{``}variables occurring free in
an expression\textsf{''}. These concepts apply equally well to mathematical
formulas and to Scala code. For example, in the mathematical expression
$\forall k.\,(n\%k)\neq0$, the variable $k$ is bound (defined and
only visible within that expression\textsf{'}s scope) but the variable $n$
is free: it must be defined somewhere outside the expression $\forall k.\,(n\%k)\neq0$.

The main difference between free and bound variables is that bound
variables can be \emph{locally renamed} at will, unlike free variables.
To see this, consider that we could rename $k$ to $z$ and write
instead of Eq.\ (\ref{eq:is_prime_def}) an equivalent definition:
\[
\text{isPrime}\left(n\right)=\forall z\in\left[2,n-1\right].\ (n\%z)\neq0\quad.
\]
\begin{lstlisting}
def isPrime(n: Int): Boolean = (2 to n-1).forall(z => n % z != 0)
\end{lstlisting}

The argument \lstinline!z! in the nameless function \lstinline*z => n % z != 0*
is a bound variable: it may be renamed without changing any code outside
that function. But \lstinline!n! is a free variable within \lstinline*z => n % z != 0*
(it is not defined inside that function, so it must be defined outside).
If we wanted to rename \lstinline!n! in the sub-expression \lstinline*z => n % z != 0*,
we would also need to change all the code that involves the variable
\lstinline!n! outside that sub-expression, or else the program would
become incorrect.

Mathematical formulas use bound variables in constructions such as
$\forall k.\,p(k)$, $\exists k.\,p(k)$, $\sum_{k=a}^{b}f(k)$, $\int_{0}^{1}k^{2}dk$,
$\lim_{n\rightarrow\infty}f(n)$, and $\text{argmax}_{k}f(k)$. When
translating mathematical expressions into code, we need to recognize
the bound variables present in the mathematical notation. For each
bound variable, we create a nameless function whose argument is that
variable, e.g., \lstinline!k => p(k)! or \lstinline!k => f(k)! for
the examples just shown. Then our code will correctly reproduce the
behavior of bound variables in mathematical expressions.

As an example, the mathematical formula $\forall k\in\left[1,n\right].\,p(k)$
has a bound variable $k$ and is translated into Scala code as:

\begin{lstlisting}
(1 to n).forall(k => p(k))
\end{lstlisting}
At this point we can apply a simplification trick to this code. The
nameless function $k\rightarrow p(k)$ does exactly the same thing
as the (named) function $p$: It takes an argument, which we may call
$k$, and returns $p(k)$. So, we can simplify the Scala code above
to:

\begin{lstlisting}
(1 to n).forall(p)
\end{lstlisting}

The simplification of $x\rightarrow f(x)$ to just $f$ is always
possible for functions $f$ of a single argument.\footnote{Certain features of Scala allow programmers to write code that looks
like \lstinline!f(x)! but actually uses an automatic conversion for
the argument \lstinline!x!, default argument values, or implicit
arguments. In those cases, replacing the code \lstinline!x => f(x)!
by \lstinline!f! will fail to compile. }

\section{Aggregating data from sequences}

Consider the task of counting the \emph{even} numbers contained in
a given list $L$ of integers. For example, the list $\left[5,6,7,8,9\right]$
contains \emph{two} even numbers: $6$ and $8$.

A mathematical formula for this task can be written using the \textsf{``}sum\textsf{''}
operation (denoted by $\sum$):
\begin{align*}
\text{countEven}\left(L\right) & =\sum_{k\in L}\text{isEven}\left(k\right)\quad,\\
\text{isEven}\left(k\right) & =\begin{cases}
1 & \text{if }(k\%2)=0\quad,\\
0 & \text{otherwise}\quad.
\end{cases}
\end{align*}
Here we defined a helper function \texttt{}\lstinline!isEven! in
order to write more easily a formula for \lstinline!countEven!. In
mathematics, complicated formulas are often split into simpler parts
by defining helper expressions. 

We can write the Scala code similarly. We first define the helper
function \lstinline!isEven!; the Scala code can be written in a style
quite similar to the mathematical formula:

\begin{lstlisting}
def isEven(k: Int): Int = (k % 2) match {
  case 0 => 1 // First, check if it is zero.
  case _ => 0 // The underscore means "otherwise".
}
\end{lstlisting}

For such a simple computation, we could also write shorter code using
a nameless function:

\begin{lstlisting}
val isEven = (k: Int) => if (k % 2 == 0) 1 else 0
\end{lstlisting}

Given this function, we now need to translate into Scala code the
expression $\sum_{k\in L}\text{isEven}\left(k\right)$. We can represent
the list $L$ using the data type \lstinline!List[Int]! from the
Scala standard library.

To compute $\sum_{k\in L}\text{isEven}\left(k\right)$, we must apply
the function \texttt{}\lstinline!isEven! to each element of the
list $L$, which will produce a list of some (integer) results, and
then we will need to add all those results together. It is convenient
to perform these two steps separately. This can be done with the functions
\texttt{}\lstinline!map! and \lstinline!sum!, defined in the Scala
standard library as methods for the data type \lstinline!List!.

The method \texttt{}\lstinline!sum! is similar to \lstinline!product!
and is defined for any \lstinline!List! of numerical types (\lstinline!Int!,
\lstinline!Float!, \lstinline!Double!, etc.). It computes the sum
of all numbers in the list:
\begin{lstlisting}
scala> List(1, 2, 3).sum
res0: Int = 6
\end{lstlisting}

The method \texttt{}\lstinline!map! needs more explanation. This
method takes a \emph{function} as its second argument and applies
that function to each element of the list. All the results are stored
in a \emph{new }list, which is then returned as the result value:

\begin{lstlisting}
scala> List(1, 2, 3).map(x => x * x + 100 * x)
res1: List[Int] = List(101, 204, 309)
\end{lstlisting}

In this example, the argument of \lstinline!map! is the nameless
function $x\rightarrow x^{2}+100x$. This function will be used repeatedly
by \texttt{}\lstinline!map! to transform each integer from the sequence
\lstinline!List(1, 2, 3)!, creating a new list as a result.

It is equally possible to define the transforming function separately,
give it a name, and then use it as the argument to \lstinline!map!:
\begin{lstlisting}
scala> def func1(x: Int): Int = x * x + 100 * x
func1: (x: Int)Int 

scala> List(1, 2, 3).map(func1)
res2: List[Int] = List(101, 204, 309)
\end{lstlisting}
Short functions are often defined inline, while longer functions are
defined separately with a name.

A method, such as \lstinline!map!, can be also used with a \textsf{``}dotless\textsf{''}
(\textbf{infix}) syntax\index{infix syntax}:
\begin{lstlisting}
scala> List(1, 2, 3) map func1       // Same as List(1, 2, 3).map(func1)
res3: List[Int] = List(101, 204, 309)
\end{lstlisting}

If the transforming function is used only once (such as \lstinline!func1!
in the example above), and especially for simple computations such
as $x\rightarrow x^{2}+100x$, it is easier to use a nameless function.

We can now combine the methods \texttt{}\lstinline!map! and \texttt{}\lstinline!sum!
to define \lstinline!countEven!:

\begin{lstlisting}
def countEven(s: List[Int]) = s.map(isEven).sum
\end{lstlisting}

This code can be also written using a nameless function instead of
\lstinline!isEven!:

\begin{lstlisting}
def countEven(s: List[Int]): Int = s
    .map { k => if (k % 2 == 0) 1 else 0 }
    .sum
\end{lstlisting}

In Scala, methods are often used one after another, as if in a chain.
For instance, \lstinline!s.map(...).sum! means: first apply \lstinline!s.map(...)!,
which returns a \emph{new} list; then apply \texttt{}\lstinline!sum!
to that new list. To make the code more readable, we may put each
of the chained methods on a new line. 

To test this code, let us run it in the Scala interpreter. In order
to let the interpreter work correctly with multi-line code, we will
enclose the code in braces:
\begin{lstlisting}
scala> def countEven(s: List[Int]): Int = {
       | s.map { k => if (k % 2 == 0) 1 else 0 }
       |  .sum
       | }
def countEven: (s: List[Int])Int

scala> countEven(List(1,2,3,4,5))
res0: Int = 2

scala> countEven( List(1,2,3,4,5).map(x => x * 2) )
res1: Int = 5
\end{lstlisting}
Note that the Scala interpreter prints the types differently for named
functions (i.e., functions declared using \lstinline!def!). It prints
\lstinline!(s: List[Int])Int! for a function of type \lstinline!List[Int] => Int!.

\section{Filtering and truncating a sequence }

In addition to the methods \lstinline!sum!, \lstinline!product!,
\lstinline!map!, \texttt{}\lstinline!forall! that we have already
seen, the Scala standard library defines many other useful methods.
We will now take a look at using the methods \lstinline!max!, \lstinline!min!,
\lstinline!exists!, \lstinline!size!, \lstinline!filter!, and \lstinline!takeWhile!. 

The methods \lstinline!max!, \lstinline!min!, and \texttt{}\lstinline!size!
are self-explanatory:
\begin{lstlisting}
scala> List(10, 20, 30).max
res2: Int = 30

scala> List(10, 20, 30).min
res3: Int = 10

scala> List(10, 20, 30).size
res4: Int = 3
\end{lstlisting}

The methods \lstinline!forall!, \lstinline!exists!, \lstinline!filter!,
and \texttt{}\lstinline!takeWhile! require a predicate as an argument.
The \texttt{}\lstinline!forall! method returns \texttt{}\lstinline!true!
if and only if the predicate returns \lstinline!true! for all values
in the list. The \texttt{}\lstinline!exists! method returns \texttt{}\lstinline!true!
if and only if the predicate holds (returns \lstinline!true!) for
at least one value in the list. These methods can be written as mathematical
formulas like this:
\begin{align*}
\text{forall}\left(S,p\right) & =\forall k\in S.\,\big(p(k)=\text{true}\big)\quad,\\
\text{exists}\left(S,p\right) & =\exists k\in S.\,\big(p(k)=\text{true}\big)\quad.
\end{align*}

The \lstinline!filter! method returns a list that contains only
the values for which a predicate returns \texttt{}\lstinline!true!:

\begin{lstlisting}
scala> List(1, 2, 3, 4, 5).filter(k => k != 3)           // Exclude the value 3.
res5: List[Int] = List(1, 2, 4, 5)
\end{lstlisting}

The \texttt{}\lstinline!takeWhile! method truncates a given list.
More precisely, \lstinline!takeWhile! returns a new list that contains
the initial portion of values from the original list for which predicate
remains \lstinline!true!:
\begin{lstlisting}
scala> List(1, 2, 3, 4, 5).takeWhile(k => k != 3)    // Truncate at the value 3.
res6: List[Int] = List(1, 2)
\end{lstlisting}

In all these cases, the predicate\textsf{'}s argument, \lstinline!k!, will
be of the same type as the elements in the list. In the examples shown
above, the elements are integers (i.e., the lists have type \lstinline!List[Int]!),
therefore \texttt{}\lstinline!k! must be of type \lstinline!Int!.

The methods \lstinline!sum! and \texttt{}\lstinline!product! are
defined for lists of numeric types, such as \lstinline!Int! or \lstinline!Float!.
The methods \lstinline!max! and \lstinline!min! are defined on lists
of \textsf{``}orderable\textsf{''} types (including \lstinline!String!, \lstinline!Boolean!,
and the numeric types). The other methods are defined for lists of
all types.

Using these methods, we can solve many problems that involve transforming
and aggregating data stored in lists, arrays, sets, and other data
structures that work as \textsf{``}containers storing values\textsf{''}. In this context,
a \textbf{transformation}\index{data transformation} is a function
taking a container with values and returning a new container with
changed values. (We speak of \textsf{``}transformation\textsf{''} even though the
original container \emph{remains unchanged}.) Examples of transformations
are \texttt{}\lstinline!filter! and \lstinline!map!. An \textbf{aggregation}\index{aggregation}
is a function taking a container of values and returning a \emph{single}
value. Examples of aggregations are \texttt{}\lstinline!max! and
\lstinline!sum!.

Writing programs by chaining together various methods of transformation
and aggregation is known as programming in the \textbf{map/reduce}
\textbf{style}.\index{map/reduce programming style@\texttt{map}/\texttt{reduce} programming style|textit}

\section{Examples\index{examples (with code)}}

\subsection{Aggregations\label{subsec:Aggregation-solved-examples}}

\subsubsection{Example \label{subsec:ch1-aggr-Example-1}\ref{subsec:ch1-aggr-Example-1}}

Improve the code for \lstinline!isPrime! by limiting the search to
$k\leq\sqrt{n}$:
\[
\text{isPrime}\left(n\right)=\forall k\in\left[2,n-1\right]\text{ such that if }k*k\leq n\text{ then }(n\%k)\neq0\quad.
\]


\subparagraph{Solution}

Use \lstinline!takeWhile! to truncate the initial list when $k*k\leq n$
becomes false:
\begin{lstlisting}
def isPrime(n: Int): Boolean = {
  (2 to n-1)
    .takeWhile(k =>  k * k <= n)
    .forall(k =>  n % k != 0)
}
\end{lstlisting}


\subsubsection{Example \label{subsec:ch1-aggr-Example-2}\ref{subsec:ch1-aggr-Example-2}}

Compute this product of absolute values: $\prod_{k=1}^{10}\left|\sin\left(k+2\right)\right|$.

\subparagraph{Solution}

~
\begin{lstlisting}
(1 to 10)
  .map(k => math.abs(math.sin(k + 2)))
  .product
\end{lstlisting}


\subsubsection{Example \label{subsec:ch1-aggr-Example-3}\ref{subsec:ch1-aggr-Example-3}}

Compute $\sum_{k\in[1,10];~\cos k>0}\sqrt{\cos k}$ (the sum goes
only over $k$ such that $\cos k>0$).

\subparagraph{Solution}

~

\begin{lstlisting}
(1 to 10)
  .filter(k => math.cos(k) > 0)
  .map(k => math.sqrt(math.cos(k)))
  .sum
\end{lstlisting}
It is safe to compute $\sqrt{\cos k}$, because we have first filtered
the list by keeping only values $k$ for which $\cos k>0$. Let us
check that this is so:
\begin{lstlisting}
scala> (1 to 10).toList.filter(k => math.cos(k) > 0).map(x => math.cos(x))
res0: List[Double] = List(0.5403023058681398, 0.28366218546322625, 0.9601702866503661, 0.7539022543433046)
\end{lstlisting}


\subsubsection{Example \label{subsec:ch1-aggr-Example-4}\ref{subsec:ch1-aggr-Example-4}}

Compute the average of a non-empty list of type \lstinline!List[Double]!,
\[
\text{average}\left(s\right)=\frac{1}{n}\sum_{i=0}^{n-1}s_{i}\quad.
\]


\subparagraph{Solution}

We need to divide the sum by the length of the list:
\begin{lstlisting}
def average(s: List[Double]): Double = s.sum / s.size

scala> average(List(1.0, 2.0, 3.0))
res0: Double = 2.0
\end{lstlisting}


\subsubsection{Example \label{subsec:ch1-aggr-Example-5-Wallis-product}\ref{subsec:ch1-aggr-Example-5-Wallis-product}}

Given $n$, compute the Wallis product\index{Wallis product}\footnote{\texttt{\href{https://en.wikipedia.org/wiki/Wallis_product}{https://en.wikipedia.org/wiki/Wallis\_product}}}
truncated up to $\frac{2n}{2n+1}$: 
\[
\text{wallis}\left(n\right)=\frac{2}{1}\frac{2}{3}\frac{4}{3}\frac{4}{5}\frac{6}{5}\frac{6}{7}...\frac{2n}{2n+1}\quad.
\]


\subparagraph{Solution}

Define the helper function \lstinline!wallis_frac(i)! that computes
the $i^{\text{th}}$ fraction. The method \texttt{}\lstinline!toDouble!
converts integers to \texttt{}\lstinline!Double! numbers:
\begin{lstlisting}
def wallis_frac(i: Int): Double = ((2 * i).toDouble / (2 * i - 1)) * ((2 * i).toDouble / (2 * i + 1))

def wallis(n: Int) = (1 to n).map(wallis_frac).product

scala> math.cos(wallis(10000))  // Should be close to 0.
res0: Double = 3.9267453954401036E-5

scala> math.cos(wallis(100000)) // Should be even closer to 0.
res1: Double = 3.926966362362075E-6
\end{lstlisting}
The cosine of \lstinline!wallis(n)! tends to zero for large $n$
because the limit of the Wallis product is $\frac{\pi}{2}$.

\subsubsection{Example \label{subsec:ch1-aggr-Example-7}\ref{subsec:ch1-aggr-Example-7}}

Check numerically the following infinite product formula:
\[
\prod_{k=1}^{\infty}\left(1-\frac{x^{2}}{k^{2}}\right)=\frac{\sin\pi x}{\pi x}\quad.
\]


\subparagraph{Solution}

Compute this product up to $k=n$ for $x=0.1$ with a large value
of $n$, say $n=10^{5}$, and compare with the right-hand side:
\begin{lstlisting}
def sine_product(n: Int, x: Double): Double = (1 to n).map(k => 1.0 - x * x / k / k).product

scala> sine_product(n = 100000, x = 0.1) // Arguments may be named, for clarity.
res0: Double = 0.9836317414461351

scala> math.sin(pi * 0.1) / pi / 0.1
res1: Double = 0.9836316430834658
\end{lstlisting}


\subsubsection{Example \label{subsec:ch1-aggr-Example-8}\ref{subsec:ch1-aggr-Example-8}}

Define a function $p$ that takes a list of integers and a function
\lstinline!f: Int => Int!, and returns the largest value of $f(x)$
among all $x$ in the list.

\subparagraph{Solution}

~

\begin{lstlisting}
def p(s: List[Int], f: Int => Int): Int = s.map(f).max
\end{lstlisting}
Here is a test for this function:
\begin{lstlisting}
scala> p(List(2, 3, 4, 5), x => 60 / x)
res0: Int = 30
\end{lstlisting}


\subsection{Transformations}

\subsubsection{Example \label{subsec:ch1-Example-1}\ref{subsec:ch1-Example-1}}

Given a list of lists, \lstinline!s: List[List[Int]]!, select the
inner lists of size at least $3$. The result must be again of type
\lstinline!List[List[Int]]!. 

\subparagraph{Solution}

To \textsf{``}select the inner lists\textsf{''} means to compute a \emph{new} list
containing only the desired inner lists. We use \texttt{}\lstinline!filter!
on the outer list \lstinline!s!. The predicate for the filter is
a function that takes an inner list and returns \texttt{}\lstinline!true!
if the size of that list is at least $3$. Write the predicate as
a nameless function, \lstinline!t => t.size >= 3!, where \texttt{}\lstinline!t!
is of type \lstinline!List[Int]!:
\begin{lstlisting}
def f(s: List[List[Int]]): List[List[Int]] = s.filter(t => t.size >= 3)

scala> f(List(  List(1,2), List(1,2,3), List(1,2,3,4)  ))
res0: List[List[Int]] = List(List(1, 2, 3), List(1, 2, 3, 4)) 
\end{lstlisting}
The Scala compiler deduces from the code that the type of \lstinline!t!
is \lstinline!List[Int]! because we apply \lstinline!filter! to
a \emph{list of lists} of integers.

\subsubsection{Example \label{subsec:ch1-Example-2}\ref{subsec:ch1-Example-2}}

Find all integers $k\in\left[1,10\right]$ such that there are at
least three different integers $j$, where $1\leq j\leq k$, each
$j$ satisfying the condition $j*j>2*k$.

\subparagraph{Solution}

~

\begin{lstlisting}
scala> (1 to 10).toList.filter(k => (1 to k).filter(j => j*j > 2*k).size >= 3)
res0: List[Int] = List(6, 7, 8, 9, 10) 
\end{lstlisting}
The argument of the outer \lstinline!filter! is a nameless function
that also uses a \lstinline!filter!. The inner expression:
\begin{lstlisting}
(1 to k).filter(j => j*j > 2*k).size >= 3
\end{lstlisting}
computes a list of all $j$\textsf{'}s that satisfy the condition $j*j>2*k$.
The size of that list is then compared with $3$. In this way, we
impose the requirement that there should be at least $3$ values of
$j$. We can see how the Scala code closely follows the mathematical
formulation of the task.

\section{Summary}

Functional programs are mathematical formulas translated into code.
Table~\ref{tab:translating-mathematics-into-code} summarizes the
tools explained in this chapter and gives implementations of some
mathematical constructions in Scala. We have also shown methods such
as \lstinline!takeWhile! that do not correspond to widely used mathematical
symbols.

\begin{table}
\begin{centering}
\begin{tabular}{|c|c|}
\hline 
\textbf{Mathematical notation} & \textbf{Scala code}\tabularnewline
\hline 
\hline 
$x\rightarrow\sqrt{x^{2}+1}$ & \lstinline!x => math.sqrt(x * x + 1)!\tabularnewline
\hline 
$\left[1,~2,~...,~n\right]$ & \lstinline!(1 to n)!\tabularnewline
\hline 
$\left[f(1),~...,~f(n)\right]$ & \lstinline!(1 to n).map(k => f(k))!\tabularnewline
\hline 
$\sum_{k=1}^{n}k^{2}$ & \lstinline!(1 to n).map(k => k * k).sum!\tabularnewline
\hline 
$\prod_{k=1}^{n}f(k)$ & \lstinline!(1 to n).map(f).product!\tabularnewline
\hline 
$\forall k\in[1,...,n].\,p(k)\text{ holds}$ & \lstinline!(1 to n).forall(k => p(k))!\tabularnewline
\hline 
$\exists k\in[1,...,n].\,p(k)\text{ holds}$ & \lstinline!(1 to n).exists(k => p(k))!\tabularnewline
\hline 
${\displaystyle \sum_{k\in S\text{ such that }p(k)\text{ holds}}}f(k)$ & \lstinline!s.filter(p).map(f).sum!\tabularnewline
\hline 
\end{tabular}
\par\end{centering}
\caption{Translating mathematics into code.\label{tab:translating-mathematics-into-code}}
\end{table}

What problems can one solve with these techniques?
\begin{itemize}
\item Compute mathematical expressions involving sums, products, and quantifiers,
based on integer ranges, such as $\sum_{k=1}^{n}f(k)$.
\item Transform and aggregate data from lists using \lstinline!map!, \lstinline!filter!,\textbf{
}\lstinline!sum!, and other methods from the Scala standard library.
\end{itemize}
What are examples of problems that are \emph{not} solvable with these
tools?
\begin{itemize}
\item Example~1: Compute the smallest $n\geq1$ such that $f(f(f(...f(0)...)))\geq1000$,
where the given function $f$ is applied $n$ times.
\item Example~2: Compute a list of partial sums from a given list of integers.
For example, the list $\left[1,2,3,4\right]$ should be transformed
into $\left[1,3,6,10\right]$. 
\item Example~3: Perform binary search over a sorted list of integers.
\end{itemize}
These computations require a general case of \emph{mathematical induction}\index{mathematical induction}.
Chapter\ \ref{chap:2-Mathematical-induction} will explain how to
implement these tasks using recursion as well as using library methods
such as \lstinline!foldLeft!.

Library functions we have seen so far, such as \texttt{}\lstinline!map!
and \lstinline!filter!, implement a restricted class of iterative
operations on lists: namely, operations that process each element
of a given list independently and accumulate results. In those cases,
the number of iterations is known (or at least bounded) in advance.
For instance, when computing \lstinline!s.map(f)!, the number of
function applications is given by the size of the initial list. However,
Example\ 1 requires applying a function $f$ repeatedly until a given
condition holds \textemdash{} that is, repeating for an \emph{initially
unknown} number of times. So it is impossible to write an expression
containing \lstinline!map!, \lstinline!filter!, \lstinline!takeWhile!,
etc., that solves Example\ 1. We could write the solution of Example\ 1
as a formula by using mathematical induction, but we have not yet
seen how to translate that into Scala code. 

An implementation of Example\ 2 is shown in Section~\ref{sec:Transforming-a-sequence}.
This cannot be implemented with operations such as \texttt{}\lstinline!map!
and \texttt{}\lstinline!filter! because they cannot produce sequences
whose next elements depend on previous values.

Example\ 3 defines the search result by induction: the list is split
in half, and search is performed recursively (i.e., using the inductive
hypothesis) in the half that contains the required value. This computation
requires an initially unknown number of steps.

\section{Exercises\label{sec:beginner-Exercises}\index{exercises}}

\subsection{Aggregations}

\subsubsection{Exercise \label{subsec:ch1-aggr-Exercise-1-1}\ref{subsec:ch1-aggr-Exercise-1-1}}

Define a function that computes a \textsf{``}staggered factorial\index{staggered factorial function}\textsf{''}
(denoted by $n!!$) for positive integers. It is defined as either
$1\cdot3\cdot...\cdot n$ or as $2\cdot4\cdot...\cdot n$, depending
on whether $n$ is even or odd. For example, $8!!=384$ and $9!!=945$.

\subsubsection{Exercise \label{subsec:ch1-aggr-Exercise-1}\ref{subsec:ch1-aggr-Exercise-1}}

Machin\textsf{'}s formula\index{Machin\textsf{'}s formula}\footnote{\texttt{\href{http://turner.faculty.swau.edu/mathematics/materialslibrary/pi/machin.html}{http://turner.faculty.swau.edu/mathematics/materialslibrary/pi/machin.html}}}
converges to $\pi$ faster than Example~\ref{subsec:ch1-aggr-Example-5-Wallis-product}:
\begin{align*}
\frac{\pi}{4} & =4\arctan\frac{1}{5}-\arctan\frac{1}{239}\quad,\\
\arctan\frac{1}{n} & =\frac{1}{n}-\frac{1}{3}\frac{1}{n^{3}}+\frac{1}{5}\frac{1}{n^{5}}-...=\sum_{k=0}^{\infty}\frac{\left(-1\right)^{k}}{2k+1}n^{-2k-1}\quad.
\end{align*}
Implement a function that computes the series for $\arctan\frac{1}{n}$
up to a given number of terms, and compute an approximation of $\pi$
using this formula. Show that $12$ terms of the series are sufficient
for a full-precision \lstinline!Double! approximation of $\pi$.

\subsubsection{Exercise \label{subsec:ch1-aggr-Example-6}\ref{subsec:ch1-aggr-Example-6}}

Check numerically that $\sum_{k=1}^{\infty}\frac{1}{k^{2}}=\frac{\pi^{2}}{6}$.
First, define a function of $n$ that computes a partial sum of that
series until $k=n$. Then compute the partial sum for a large value
of $n$ and compare with the limit value.

\subsubsection{Exercise \label{subsec:ch1-aggr-Exercise-2}\ref{subsec:ch1-aggr-Exercise-2}}

Using the function \lstinline!isPrime!, check numerically the Euler
product\index{Euler product} formula\footnote{\texttt{\href{http://tinyurl.com/4rjj2rvc}{http://tinyurl.com/4rjj2rvc}}}
for the Riemann\textsf{'}s zeta function\index{Riemann\textsf{'}s zeta function} $\zeta(4)$.
It is known\footnote{\texttt{\href{https://ocw.mit.edu/courses/mathematics/18-104-seminar-in-analysis-applications-to-number-theory-fall-2006/projects/chan.pdf}{https://tinyurl.com/yxey4tsd}}}
that $\zeta(4)=\frac{\pi^{4}}{90}$:
\[
\zeta(4)=\prod_{k\geq2;~k\text{ is prime}}\frac{1}{1-\frac{1}{k^{4}}}=\frac{\pi^{4}}{90}\quad.
\]


\subsection{Transformations}

\subsubsection{Exercise \label{subsec:ch1-transf-Exercise-1}\ref{subsec:ch1-transf-Exercise-1}}

Define a function \texttt{}\lstinline!add20! of type \texttt{}\lstinline!List[List[Int]] => List[List[Int]]!
that adds $20$ to every element of every inner list. A sample test:
\begin{lstlisting}
scala> add20( List( List(1), List(2, 3) ) )
res0: List[List[Int]] = List(List(21), List(22, 23))
\end{lstlisting}


\subsubsection{Exercise \label{subsec:ch1-transf-Exercise-2}\ref{subsec:ch1-transf-Exercise-2}}

An integer $n$ is called a \textsf{``}$3$-factor\textsf{''} if it is divisible by
only three different integers $i$, $j$, $k$ such that $1<i<j<k<n$.
Compute the set of all \textsf{``}$3$-factor\textsf{''} integers $n$ among $n\in[1,...,1000]$
.

\subsubsection{Exercise \label{subsec:ch1-transf-Exercise-3}\ref{subsec:ch1-transf-Exercise-3}}

Given a function \lstinline!f: Int => Boolean!, an integer $n$ is
called a \textsf{``}$3$-$f$\textsf{''} if there are only three different integers
$1<i<j<k<n$ such that $f(i)$, $f(j)$, and $f(k)$ are all \lstinline!true!.
Define a function that takes $f$ as an argument and returns a sequence
of all \textsf{``}$3$-$f$\textsf{''} integers among $n\in[1,...,1000]$. What is
the type of that function? Implement Exercise~\ref{subsec:ch1-transf-Exercise-2}
using that function.

\subsubsection{Exercise \label{subsec:ch1-transf-Exercise-4}\ref{subsec:ch1-transf-Exercise-4}}

Define a function \lstinline!at100! of type \texttt{}\lstinline!List[List[Int]] => List[List[Int]]!
that selects only those inner lists whose largest value is at least
$100$. Test with:
\begin{lstlisting}
scala> at100( List( List(0, 1, 100), List(60, 80), List(1000) ) )
res0: List[List[Int]] = List(List(0, 1, 100), List(1000))
\end{lstlisting}


\subsubsection{Exercise \label{subsec:ch1-transf-Exercise-5}\ref{subsec:ch1-transf-Exercise-5}}

Define a function of type \texttt{}\lstinline!List[Double] => List[Double]!
that performs a \textsf{``}normalization\textsf{''} of a list: it finds the element
having the largest absolute value and, if that value is zero, returns
the original list; if that value is nonzero, divides all elements
by that value and returns a new list. Test with:
\begin{lstlisting}
scala> normalize(List(1.0, -4.0, 2.0))
res0: List[Double] = List(0.25, -1.0, 0.5)
\end{lstlisting}


\section{Discussion}

\subsection{Functional programming as a paradigm}

Functional programming\index{functional programming paradigm} (FP)
is a \textbf{paradigm}\index{paradigm of programming} \textemdash{}
an approach that guides programmers to write code in specific ways,
applicable to a wide range of tasks.

The main idea of FP is to write code \emph{as a mathematical expression
or formula}. This allows programmers to derive code through logical
reasoning rather than through guessing, similarly to how books on
mathematics reason about mathematical formulas and derive results
systematically, without guessing or \textsf{``}debugging.\textsf{''} Like mathematicians
and scientists who reason about formulas, functional programmers can
\emph{reason about code} systematically and logically, based on rigorous
principles. This is possible only because code is written as a mathematical
formula.

Mathematical intuition is useful for programming tasks because it
is backed by the vast experience of working with data over millennia
of human history. It took centuries to invent flexible and powerful
notation, such as $\sum_{k\in S}p(k)$, and to develop the corresponding
rules of calculation. Converting formulas into code, FP capitalizes
on the power of these reasoning tools.

As we have seen, the Scala code for certain computational tasks corresponds
quite closely to mathematical formulas (although programmers do have
to write out some details that are omitted in the mathematical notation).
Just as in mathematics, large code expressions may be split into smaller
expressions when needed. Expressions can be reused, composed in various
ways, and written independently from each other. Over the years, the
FP community has developed a toolkit of functions (such as \lstinline!map!,
\texttt{}\lstinline!filter!, \lstinline!flatMap!, etc.), which
are not standard in mathematical literature but proved to be useful
in practical programming.

Mastering FP involves practicing to write programs as \textsf{``}formulas
translated into code\textsf{''}, building up the specific kind of applied
mathematical intuition, and getting familiar with certain concepts
adapted to a programmer\textsf{'}s needs. The FP community has discovered a
number of specific programming idioms founded on mathematical principles
but driven by practical necessities of writing software. This book
explains the theory behind those idioms, starting from code examples
and heuristic ideas, and gradually building up the techniques of rigorous
reasoning.

This chapter explored the first significant idiom of FP: iterative
calculations performed without loops in the style of mathematical
expressions. This technique can be used in any programming language
that supports nameless functions. 

\subsection{Iteration without loops}

In mathematical notation, iterative computations are written without
loops. As an example, consider the formula for the standard deviation
($\sigma$) estimated from a data sample $\left[x_{1},...,x_{n}\right]$:
\[
\sigma=\sqrt{\frac{1}{n-1}\sum_{i=1}^{n}x_{i}^{2}-\frac{1}{n\left(n-1\right)}\left(\sum_{i=1}^{n}x_{i}\right)^{2}}\quad.
\]
Here the index $i$ goes over the integer range $\left[1,...,n\right]$.
And yet no mathematics textbook would define this formula via loops
or by saying \textsf{``}now repeat this equation ten times\textsf{''}. Indeed, it
is unnecessary to evaluate a formula such as $x_{i}^{2}$ ten times,
as the value of $x_{i}^{2}$ remains the same every time. It is just
as unnecessary to \textsf{``}repeat\textsf{''} a mathematical equation.

Instead of loops, mathematicians write \emph{expressions} such as
$\sum_{i=1}^{n}s_{i}$, where symbols such as $\sum_{i=1}^{n}$ denote
certain iterative computations. Such computations are can be defined
rigorously using mathematical induction\index{mathematical induction}.
The FP paradigm has developed rich tools for translating mathematical
induction into code. This chapter focuses on methods such as \lstinline!map!,
\lstinline!filter!, and \lstinline!sum!. The next chapter shows
more general methods for implementing inductive computations. These
methods can be combined in flexible ways, enabling programmers to
write iterative code without loops. For example, the value $\sigma$
defined by the formula shown above is computed by code that looks
like this:
\begin{lstlisting}
def sigma(xs: Seq[Double]): Double = {
  val n = xs.length.toDouble
  val xsum = xs.sum
  val x2sum = xs.map(x => x * x).sum
  math.sqrt(x2sum / (n - 1) - xsum * xsum / n / (n - 1))
}

scala> sigma(Seq(10, 20, 30))
res0: Double = 10.0
\end{lstlisting}

The programmer can avoid writing loops because all iterative computations
are delegated to functions such as \lstinline!map!, \lstinline!filter!,
\lstinline!sum!, and others. It is the job of the library and the
compiler to translate those high-level functions into low-level machine
code. The machine code \emph{will} likely contain loops, but the programmer
does not need to see that machine code or to reason about it.

\subsection{The mathematical meaning of \textquotedblleft variables\textquotedblright}

The usage of variables in functional programming is similar to how
mathematical literature uses variables. In mathematics, \textbf{variables}\index{variable}
are used first of all as \emph{arguments} of functions; e.g., the
formula:
\[
f(x)=x^{2}+x
\]
contains the variable $x$ and defines a function $f$ that takes
$x$ as its argument (to be definite, assume that $x$ is an integer)
and computes the value $x^{2}+x$. The body of the function is the
expression $x^{2}+x$. 

Mathematics has the convention that a variable, such as $x$, does
not change its value within a formula. Indeed, there is no mathematical
notation even to talk about \textsf{``}changing\textsf{''} the value of $x$ \emph{inside}
the formula $x^{2}+x$. It would be quite confusing if a mathematics
textbook said \textsf{``}before adding the last $x$ in the formula $x^{2}+x$,
we change that $x$ by adding $4$ to it\textsf{''}. If the \textsf{``}last $x$\textsf{''}
in $x^{2}+x$ needs to have a $4$ added to it, a mathematics textbook
will just write the formula $x^{2}+x+4$.

Arguments of nameless functions are also immutable. Consider, for
example:
\[
f(n)=\sum_{k=0}^{n}\,(k^{2}+k)\quad.
\]
Here, $n$ is the argument of the function $f$, while $k$ is the
argument of the nameless function $k\rightarrow k^{2}+k$. Neither
$n$ nor $k$ can be \textsf{``}modified\textsf{''} in any sense within the expressions
where they are used. The symbols $k$ and $n$ stand for some integer
values, and these values are immutable. Indeed, it is meaningless
to say that we \textsf{``}modified the integer $4$\textsf{''}. In the same way, we
cannot modify $k$.

So, a variable in mathematics remains constant \emph{within} \emph{the
expression} where it is defined; in that expression, a variable is
essentially a \textsf{``}named constant\textsf{''}. Of course, a function $f$ can
be applied to different values $x$, to compute a different result
$f(x)$ each time. However, a given value of $x$ will remain unmodified
within the body of the function $f$ while $f(x)$ is being computed.

Functional programming adopts this convention from mathematics: variables
are immutable named constants. (Scala also has \emph{mutable} variables,
but we will not consider them in this book.)

In Scala, function arguments are immutable within the function body:
\begin{lstlisting}
def f(x: Int) = x * x + x // Cannot modify `x` here.
\end{lstlisting}

The \emph{type} of each mathematical variable (such as integer, vector,
etc.) is also fixed. Each variable is a value from a specific set
(e.g., the set of all integers, the set of all vectors, etc.). Mathematical
formulas such as $x^{2}+x$ do not express any \textsf{``}checking\textsf{''} that
$x$ is indeed an integer and not, say, a vector, in the middle of
evaluating $x^{2}+x$. The types of all variables are checked in advance.

Functional programming adopts the same view: Each argument of each
function must have a \textbf{type}\index{types} that represents the
set of possible allowed values for that function argument. The programming
language\textsf{'}s compiler will automatically check the types of all arguments
in advance, \emph{before} the program runs. A program that calls functions
on arguments of incorrect types will not compile.

The second usage of \textbf{variables}\index{variable} in mathematics
is to denote expressions that will be reused. For example, one writes:
let $z=\frac{x-y}{x+y}$ and now compute $\cos z+\cos2z+\cos3z$.
Again, the variable $z$ remains immutable, and its type remains fixed.

In Scala, this construction (defining an expression to be reused later)
is written with the \textsf{``}\lstinline!val!\textsf{''} syntax. Each variable defined
using \textsf{``}\lstinline!val!\textsf{''} is a named constant, and its type and
value are fixed at the time of definition. Type annotations for \textsf{``}\lstinline!val!'\textsf{'}s
are optional in Scala. For instance, we could write:
\begin{lstlisting}
val x: Int = 123
\end{lstlisting}
We could also omit the type annotation \textsf{``}\lstinline!:Int!\textsf{''} and
write more concisely:
\begin{lstlisting}
val x = 123
\end{lstlisting}
Here, it is clear that this \texttt{}\lstinline!x! is an integer.
Nevertheless, it is often helpful to write out the types. If we do
so, the compiler will check that the types match correctly and give
an error message whenever wrong types are used. For example, a type
error is detected when using a \lstinline!String! instead of an \lstinline!Int!:
\begin{lstlisting}
scala> val x: Int = "123"
<console>:11: error: type mismatch;
 found   : String("123")
 required: Int
        val x: Int = "123"
                     ^
\end{lstlisting}


\subsection{Nameless functions in mathematical notation\label{subsec:Nameless-functions-in-mathematical-notation}}

Functions in mathematics are mappings from one set to another. A function
does not necessarily \emph{need} a name; we just need to define the
mapping. However, nameless functions have not been widely used in
the conventional mathematical notation. It turns out that nameless
functions are important in functional programming because, in particular,
they allow programmers to write code with a straightforward and consistent
syntax.

Nameless functions contain bound variables that are invisible outside
the function\textsf{'}s scope. This property is directly reflected by the prevailing
mathematical conventions. Compare the formulas:
\[
f\left(x\right)=\int_{0}^{x}\frac{dx}{1+x}\quad;\quad f\left(x\right)=\int_{0}^{x}\frac{dz}{1+z}\quad.
\]
The mathematical convention is that one may rename the integration
variable at will, and so these formulas define the same function $f$.

In programming, one situation when a variable \textsf{``}may be renamed at
will\textsf{''} is when the variable represents an argument of a function.
We can see that the notations $\frac{dx}{1+x}$ and $\frac{dz}{1+z}$
correspond to a nameless function whose argument was renamed from
$x$ to $z$. In FP notation, this nameless function would be denoted
as $z\rightarrow\frac{1}{1+z}$, and the integral rewritten as code
such as:
\begin{lstlisting}
integration(0, x, { z => 1.0 / (1 + z) } )
\end{lstlisting}

Now compare the mathematical notations for integration and for summation:
$\int_{0}^{x}\frac{dz}{1+z}$ and $\sum_{k=0}^{100}\frac{1}{1+k}$.
The integral defines a bound variable $z$ via the special symbol
\textsf{``}$d$\textsf{''}, while the summation places a bound variable $k$ in a
subscript under $\sum$. The notation could be made more consistent
by using nameless functions explicitly, for example like this:
\begin{align*}
\text{denote summation by }\sum_{0}^{x} & \left(k\rightarrow\frac{1}{1+k}\right)\text{ instead of }\sum_{k=0}^{x}\frac{1}{1+k}\quad,\\
\text{denote integration by }\int_{0}^{x} & \left(z\rightarrow\frac{1}{1+z}\right)\text{ instead of }\int_{0}^{x}\frac{dz}{1+z}\quad.
\end{align*}
In the new notation, the summation symbol $\sum_{0}^{x}$ does not
mention the name \textsf{``}$k$\textsf{''} but takes a function as an argument. Similarly,
the integration symbol $\int_{0}^{x}$ does not mention \textsf{``}$z$\textsf{''}
and does not use the special symbol \textsf{``}$d$\textsf{''} but takes a function
as an argument. Written in this way, the operations of summation and
integration become \emph{functions} that take functions as arguments.
The above summation may be written as a Scala function:
\begin{lstlisting}
summation(0, x, { y => 1.0 / (1 + y) } )
\end{lstlisting}

We could implement \texttt{}\lstinline!summation(a, b, g)! as:
\begin{lstlisting}
def summation(a: Int, b: Int, g: Int => Double): Double = (a to b).map(g).sum

scala> summation(1, 10, x => math.sqrt(x))
res0: Double = 22.4682781862041
\end{lstlisting}

Integration requires longer code since the computations are more complicated.
\index{Simpson\textsf{'}s rule}Simpson\textsf{'}s rule\footnote{\texttt{\href{https://en.wikipedia.org/wiki/Simpson\%27s_rule}{https://en.wikipedia.org/wiki/Simpson\%27s\_rule}}}
gives the following formulas for numerical integration:
\begin{align*}
\text{simpson}\left(a,b,g,\varepsilon\right) & =\frac{\delta}{3}\big(g(a)+g(b)+4s_{1}+2s_{2}\big)\quad,\\
\text{where }~~~n & =2\left\lfloor \frac{b-a}{\varepsilon}\right\rfloor \quad,\quad\quad\delta_{x}=\frac{b-a}{n}\quad,\\
s_{1}=\sum_{k=1,3,...,n-1}g(a+k\delta_{x}) & \quad,\quad\quad s_{2}=\sum_{k=2,4,...,n-2}g(a+k\delta_{x})\quad.
\end{align*}
Here is a straightforward line-by-line translation of these formulas
into Scala:
\begin{lstlisting}
def simpson(a: Double, b: Double, g: Double => Double, eps: Double): Double = {
   // First, we define some helper values and functions corresponding
   // to the definitions "where n = ..." in the mathematical formulas.
   val n: Int = 2 * ((b - a) / eps).toInt
   val delta_x = (b - a) / n
   val s1 = (1 to (n - 1) by 2).map { k => g(a + k * delta_x) }.sum
   val s2 = (2 to (n - 2) by 2).map { k => g(a + k * delta_x) }.sum
   // Now we can write the expression for the final result.
   delta_x / 3 * (g(a) + g(b) + 4 * s1 + 2 * s2)
}

scala> simpson(0, 5, x => x*x*x*x, eps = 0.01)       // The answer is 625.
res0: Double = 625.0000000004167

scala> simpson(0, 7, x => x*x*x*x*x*x, eps = 0.01)   // The answer is 117649.
res1: Double = 117649.00000014296
\end{lstlisting}

The entire code is one large \emph{expression}, with a few sub-expressions
(\lstinline!s1!, \lstinline!s2!, etc.) defined for within the \textbf{local
scope}\index{local scope} of the function (that is, within the function\textsf{'}s
body). The code contains no loops. This is similar to the way a mathematical
text would define Simpson\textsf{'}s rule. In other words, this code is written
in the FP paradigm. Similar code can be written in any programming
language that supports nameless functions as arguments of other functions. 

\subsection{Named and nameless expressions and their uses}

It is a significant advantage if a programming language supports unnamed
(or \textsf{``}nameless\textsf{''}) expressions. To see this, consider a familiar
situation where we take the absence of names for granted.

In today\textsf{'}s programming languages, we may directly write expressions
such as \texttt{}\lstinline!(x + 123) * y / (4 + x)!. Note that
the entire expression does not need to have a name. Parts of that
expression (e.g., the sub-expressions \texttt{}\lstinline!x + 123!
or \lstinline!4 + x!) also do not have separate names. It would be
inconvenient if we \emph{needed} to assign a name to each sub-expression.
The code for \lstinline!(x + 123) * y / (4 + x)! would look like
this:

\begin{lstlisting}
{
  val r0 = 123
  val r1 = x + r0
  val r2 = r1 * y
  val r3 = 4 
  val r4 = r3 + x
  val r5 = r2 / r4     // Do we still remember what `r2` means?
  r5
}
\end{lstlisting}

This style of programming resembles assembly languages\index{assembly language}:
every sub-expression \textemdash{} that is, every step of every calculation
\textemdash{} needs to be written into a separate memory address or
a CPU register.

Programmers become more productive when their programming language
supports nameless expressions. This is also common practice in mathematics;
names are assigned when needed, but most expressions remain nameless.

It is also useful to be able to create nameless data structures. For
instance, a \textbf{dictionary}\index{dictionary} (also called a
\textsf{``}map\textsf{''} or a \textsf{``}hashmap\textsf{''}) is created in Scala with this code:
\begin{lstlisting}
Map("a" -> 1, "b" -> 2, "c" -> 3)
\end{lstlisting}
This is a nameless expression whose value is a dictionary. In programming
languages that do not have such a construction, programmers have to
write special code that creates an initially empty dictionary and
then fills in one value at a time:
\begin{lstlisting}[language=Java]
// Scala code creating a dictionary:
Map("a" -> 1, "b" -> 2, "c" -> 3)

// Shortest Java code for the same:
new HashMap<String, Integer>() {{
   put("a", 1);
   put("b", 2);
   put("c", 3);
}}
\end{lstlisting}

Nameless functions are useful for the same reason as other nameless
values: they allow us to build larger programs from simpler parts
in a uniform way.

\subsection{Historical perspective on nameless functions}

\begin{table}
\begin{centering}
\begin{tabular}{|c|c|c|}
\hline 
\textbf{\small{}Language} & \textbf{\small{}Year} & \textbf{\small{}Code for }{\small{}$k\rightarrow k+1$}\tabularnewline
\hline 
\hline 
{\footnotesize{}$\lambda$-calculus} & {\footnotesize{}1936} & $\lambda k.~add~k~1$\tabularnewline
\hline 
{\footnotesize{}typed $\lambda$-calculus} & {\footnotesize{}1940} & $\lambda k:int.~add~k~1$\tabularnewline
\hline 
{\footnotesize{}LISP} & {\footnotesize{}1958} & \texttt{\footnotesize{}}\lstinline!(lambda (k) (+ k 1))!\tabularnewline
\hline 
{\footnotesize{}ALGOL 68} & {\footnotesize{}1968} & \texttt{\footnotesize{}}\lstinline!(INT k) INT: k + 1!\tabularnewline
\hline 
{\footnotesize{}Standard ML} & {\footnotesize{}1973} & \texttt{\footnotesize{}}\lstinline!fn (k: int) => k + 1!\tabularnewline
\hline 
{\footnotesize{}Caml} & {\footnotesize{}1985} & \lstinline!fun (k: int) -> k + 1!\tabularnewline
\hline 
{\footnotesize{}Erlang} & {\footnotesize{}1986} & \lstinline!fun(K) -> K + 1 end!\tabularnewline
\hline 
{\footnotesize{}Haskell} & {\footnotesize{}1990} & \lstinline!\ k -> k + 1!\tabularnewline
\hline 
{\footnotesize{}Oz} & {\footnotesize{}1991} & \lstinline!fun {$ K} K + 1!\tabularnewline
\hline 
{\footnotesize{}R} & {\footnotesize{}1993} & \lstinline!function(k) k + 1!\tabularnewline
\hline 
{\footnotesize{}Python 1.0} & {\footnotesize{}1994} & \lstinline!lambda k: k + 1!\tabularnewline
\hline 
{\footnotesize{}JavaScript} & {\footnotesize{}1995} & \lstinline!function(k) { return k + 1; }!\tabularnewline
\hline 
{\footnotesize{}Mercury} & {\footnotesize{}1995} & \lstinline!func(K) = K + 1!\tabularnewline
\hline 
{\footnotesize{}Ruby} & {\footnotesize{}1995} & \lstinline!lambda { |k| k + 1 }!\tabularnewline
\hline 
{\footnotesize{}Lua 3.1} & {\footnotesize{}1998} & \lstinline!function(k) return k + 1 end!\tabularnewline
\hline 
{\footnotesize{}Scala} & {\footnotesize{}2003} & \lstinline!(k: Int) => k + 1!\tabularnewline
\hline 
{\footnotesize{}F\#} & {\footnotesize{}2005} & \lstinline!fun (k: int) -> k + 1!\tabularnewline
\hline 
{\footnotesize{}C\# 3.0} & {\footnotesize{}2007} & \lstinline!delegate(int k) { return k + 1; }!\tabularnewline
\hline 
{\footnotesize{}Clojure} & {\footnotesize{}2009} & \lstinline!(fn [k] (+ k 1))!\tabularnewline
\hline 
{\footnotesize{}C++ 11} & {\footnotesize{}2011} & \lstinline![] (int k) { return k + 1; }!\tabularnewline
\hline 
{\footnotesize{}Go} & {\footnotesize{}2012} & \lstinline!func(k int) { return k + 1 }!\tabularnewline
\hline 
{\footnotesize{}Julia} & {\footnotesize{}2012} & \lstinline!function(k :: Int) k + 1 end!\tabularnewline
\hline 
{\footnotesize{}Kotlin} & {\footnotesize{}2012} & \lstinline!{ k: Int -> k + 1 }!\tabularnewline
\hline 
{\footnotesize{}Swift} & {\footnotesize{}2014} & \lstinline!{ (k: int) -> int in return k + 1 }!\tabularnewline
\hline 
{\footnotesize{}Java 8} & {\footnotesize{}2014} & \lstinline!(int k) -> k + 1!\tabularnewline
\hline 
{\footnotesize{}Rust} & {\footnotesize{}2015} & \lstinline!|k: i32| k + 1!\tabularnewline
\hline 
\end{tabular}
\par\end{centering}
\caption{Nameless functions in various programming languages.\label{lambda-functions-table}}

\end{table}

What this book calls (for clarity) a \textsf{``}nameless function\textsf{''} is also
known as an anonymous function,\index{anonymous function!see \textsf{``}nameless functions\textsf{''}}
a function expression, a function literal, a closure, a \index{lambda-function!see \textsf{``}nameless function\textsf{''}}lambda
function, a lambda expression, or just a \textsf{``}lambda\textsf{''}.

Nameless functions were first used in 1936 in a theoretical programming
language called \textsf{``}$\lambda$-calculus\index{\$lambda\$@$\lambda$-calculus}\textsf{''}.
In that language,\footnote{Although called a \textsf{``}calculus,\textsf{''} it is a (drastically simplified)
\emph{programming} \emph{language}, not related to differential or
integral calculus. Practitioners of functional programming do not
need to study the theory of $\lambda$-calculus. The practically relevant
knowledge that comes from $\lambda$-calculus will be explained in
Chapter~\ref{chap:Higher-order-functions}.} all functions are nameless and have a single argument. The Greek
letter $\lambda$ is a syntax separator that denotes function arguments
in nameless functions. For example, the nameless function $x\rightarrow x+1$
would be written as $\lambda x.~add~x~1$ in $\lambda$-calculus if
it had a function $add$ for adding integers (but it does not).

In most programming languages that were in use until around 1990,
all functions required names. But by 2015, the use of nameless functions
in the \lstinline!map!/\lstinline!reduce! programming style\index{map/reduce programming style@\texttt{map}/\texttt{reduce} programming style}
turned out to be so productive that most newly created languages included
nameless functions, while older languages added that feature. Table~\ref{lambda-functions-table}
shows the year when various languages supported nameless functions.
