\global\long\def\bef{\forwardcompose}%
\global\long\def\bbnum#1{\custombb{#1}}%


\chapter{Summary and problems}

\section{Exercises in applied functional type theory\label{chap:Exercises-in-AFTT}}

\subsection{Exercises\index{exercises}}

The following exercises can be solved using techniques developed in
this book.
\begin{enumerate}
\item Compute the smallest integer expressible as a sum of two cubed integers
in more than one way.
\item Read a text file, split it by spaces into words, and print the word
counts, sorted by decreasing count.
\item FPIS exercise 2.2: Check whether a sequence \lstinline!Seq[A]! is
sorted according to a given ordering function of type \lstinline!(A, A) => Boolean!.
\item FPIS exercise 3.24: Implement a function \lstinline!hasSubsequence!
that checks whether a \lstinline!List! contains another \lstinline!List!
as a subsequence. For instance, \lstinline!List(1,2,3,4)! would have
\lstinline!List(1,2)!, \lstinline!List(2,3)!, and \lstinline!List(4)!
as subsequences, among others. (Dynamic programming?)
\item The Seasoned Schemer, chapter 13: Implement \lstinline!remberUptoLast!
and \lstinline!remberBeyondLast!.
\begin{lstlisting}
def remberUptoLast[A](a: A, s: List[A]): List[A] = ???
def remberBeyondLast[A](a: A, s: List[A]): List[A] = ???

scala> remberUptoLast(4, List(1, 2, 3, 4, 5, 4, 5, 6, 7, 8))
List(5, 6, 7, 8)

scala> remberBeyondLast(4, List(1, 2, 3, 4, 5, 4, 5, 6, 7, 8))
List(1, 2, 3, 4, 5)
\end{lstlisting}
The function \lstinline!remberUptoLast! removes all elements from
a list up to and including the last occurrence of a given value. The
function \lstinline!remberBeyondLast! removes all elements from a
list starting from the last occurrence of a given value. Both functions
return the original list if the given value does not occur in the
list.
\item (Bird-de Moor, page 20) Derive the following identity between functions
$F^{A}\rightarrow F^{A}$, for any filterable functor $F$ and any
predicate $p^{:A\rightarrow\bbnum 2}$: 
\[
\text{filt}_{F}(p)=(\Delta\bef\text{id}\boxtimes p)^{\uparrow F}\bef\text{filt}_{F}(\pi_{2})\bef\pi_{1}^{\uparrow F}\quad.
\]
\item Define a monoid of partial functions with fixed types $P\rightarrow Q$.
\begin{lstlisting}
final case class PFM[P, Q](pf: PartialFunction[P, Q])
// After defining a monoid instance, the following code must work:
val p1 = PFM[Option[Int], String] { case Some(3) => "three" }
val p2 = PFM[Option[Int], String] {
  case Some(20)   => "twenty"
  case None       => "empty"
}
p1 |+| p2 // Must be the same as the concatenation of all `case` clauses.
\end{lstlisting}
\item Consider a typeclass called \textsf{``}\lstinline!Splittable!\textsf{''} for functors
$F^{\bullet}$ that have an additional method 
\[
\text{split}^{A,B}:F^{A+B}\rightarrow F^{A}+B
\]
with a non-degeneracy law
\[
(x^{:A}\rightarrow x+\bbnum 0^{:B})^{\uparrow F}\bef\text{split}=y^{:F^{A}}\rightarrow y+\bbnum 0^{:B}
\]
and a special associativity law for functions $F^{A+B+C}\rightarrow F^{A}+B+C$,
\[
\text{split}^{A+B,C}\bef\,\begin{array}{|c||cc|}
 & F^{A}+B & C\\
\hline F^{A+B} & \text{split}^{A,B} & \bbnum 0\\
C & \bbnum 0 & \text{id}
\end{array}\,=\text{split}^{A,B+C}\quad.
\]
Show that all polynomial functors $F^{\bullet}$ belong to this typeclass.
Show that exponential functors such as $F^{A}\triangleq Z\rightarrow A$
do not.
\item Given a monad $M$ and a fixed type $Z$, consider the functor $F^{A}\triangleq(A\rightarrow M^{Z})\rightarrow Z$.
Show that $F$ is a semimonad but not a full monad. Hint: use the
flipped Kleisli technique.
\item Given two fixed types $P$, $Q$ that are not known to be equivalent,
consider the contrafunctors $F^{A}\triangleq\left(\left(A\rightarrow P\right)\rightarrow P\right)\rightarrow Q$
and $G^{A}\triangleq A\rightarrow Q$. Show that there exist natural
transformations $F^{\bullet}\leadsto G^{\bullet}$ and $G^{\bullet}\leadsto F^{\bullet}$.
Show that these transformations are not isomorphisms.
\item Given two fixed types $P$, $Q$ that are not known to be equivalent,
show that the functor $L^{A}\triangleq\left(\left(\left(A\rightarrow P\right)\rightarrow Q\right)\rightarrow Q\right)\rightarrow P$
is a semimonad but not a full monad. (When $P\cong Q$, the functor
$L$ is also not a full monad because $L$ is then equivalent to a
composition of the continuation monad with itself. See Exercise~\ref{subsec:Exercise-monad-composition-mm}.)
\item When $M$ is a given semimonad, show that $M\circ M\circ...\circ M$
(with finitely many $M$) is also a lawful semimonad.
\item If $M$ is a commutative monad and $W$ is a commutative monoid then
the monoid $M^{W}$ is commutative.
\end{enumerate}

\subsection{Open problems\index{open problems}}

The author of this book does not know how to answer the following
questions.

\subsubsection{Problem \label{par:Problem-monads}\ref{par:Problem-monads}}

Are there any polynomial monads not obtained by a chain of constructions
shown in Section~\ref{subsec:Constructions-of-polynomial-monads}?
The available constructions are:
\begin{itemize}
\item The polynomial monad $F^{A}\triangleq Z+W\times A$, where $W$ is
a monoid and $Z$ is a fixed type.
\item The free pointed monad $L^{A}\triangleq A+F^{A}$, where $F$ is a
monad.
\item The product monad $L^{A}\triangleq F^{A}\times G^{A}$, where $F$
and $G$ are monads.
\item The composed monad $L^{A}\triangleq F^{Z+W\times A}$, where $F$
is a monad, $W$ is a monoid, and $Z$ is a fixed type.
\end{itemize}

\subsubsection{Problem \label{par:Problem-monads-1}\ref{par:Problem-monads-1}}

Do all polynomial functors of the form $P_{n}^{A}\triangleq\bbnum 1+\overbrace{A\times A\times...\times A}^{n\text{ times, }n\ge2}$
fail to be monads? An example is the functor $P_{2}^{A}\triangleq\bbnum 1+A\times A$,
which is not a monad because the monad laws are not satisfied by any
implementations of \lstinline!pure! and \lstinline!flatMap! methods
for that functor.\footnote{See discussion here: \texttt{\href{https://stackoverflow.com/questions/49742377}{https://stackoverflow.com/questions/49742377}}}

\subsubsection{Problem \label{subsec:Problem-co-pointed-applicative}\ref{subsec:Problem-co-pointed-applicative}}

Statement~\ref{subsec:Statement-co-pointed-applicative-example}
shows that co-pointed applicative functors of the form $L^{A}\triangleq A\times G^{A}$
satisfy the compatibility law~(\ref{eq:compatibility-law-of-extract-and-zip}).
Statement~\ref{subsec:Statement-co-pointed-applicative-example-failing-compatibility-law}
shows that $L^{A}\triangleq Z\times\left(Z\rightarrow A\right)$ is
applicative and co-pointed but fails the compatibility law. Do there
exist any co-pointed applicative functors that satisfy the compatibility
law~(\ref{eq:compatibility-law-of-extract-and-zip}) and are \emph{not}
of the form $A\times G^{A}$ for some applicative functor $G^{\bullet}$? 

\subsubsection{Problem \label{par:Problem-monads-2}\ref{par:Problem-monads-2}}

Does a monad transformer exist for every monad whose \lstinline!pure!
and \lstinline!flatMap! methods are implemented using fully parametric
code? Given the code for an arbitrary monad, can we derive the type
constructor and the implementation of the corresponding monad transformer?

\subsubsection{Problem \label{par:Problem-monads-5-1}\ref{par:Problem-monads-5-1}}

We know that the monad transformers $T_{L}$ can be defined using
a \lstinline!swap! function for certain monads. Is this also the
case for any monad stacks built out of such monads? If each of the
monads $L_{1}$, $L_{2}$, ..., $L_{k}$ admits a transformer defined
via a lawful \lstinline!swap! function, will the monad $L_{1}\varangle L_{2}\varangle...\varangle L_{k}$
also admit a transformer with a \lstinline!swap! function? (See Section~\ref{subsec:Does-a-composition-have-swap}
for some results.)

\subsubsection{Problem \label{par:Problem-identity-natural-monad-morphism}\ref{par:Problem-identity-natural-monad-morphism}}

Are there any monadically natural monad morphisms $M\leadsto M$ that
are not identity? Equivalently, are there any natural transformations
$\text{Id}^{\bullet}\leadsto\text{Id}^{\bullet}$ between identity
functors in the category of monads, other than the identity transformation?

We look for a monad morphism $\varepsilon^{M,A}:M^{A}\rightarrow M^{A}$
that is defined for all monads $M$ and is monadically natural in
the parameter $M$. So, the function $\varepsilon$ must satisfy the
following laws:
\begin{align*}
{\color{greenunder}\text{naturality law}:}\quad & \varepsilon^{M,A}\bef(f^{:A\rightarrow B})^{\uparrow M}=(f^{:A\rightarrow B})^{\uparrow M}\bef\varepsilon^{M,B}\quad,\\
{\color{greenunder}\text{monad morphism laws}:}\quad & \text{pu}_{M}\bef\varepsilon=\text{pu}_{M}\quad,\quad\quad\varepsilon^{\uparrow M}\bef\varepsilon\bef\text{ftn}_{M}=\text{ftn}_{M}\bef\varepsilon\quad,\\
{\color{greenunder}\text{monadic naturality law}:}\quad & \varepsilon^{M,A}\bef\phi^{:M^{:A}\rightarrow N^{A}}=\phi^{:M^{A}\rightarrow N^{A}}\bef\varepsilon^{N,A}\quad,
\end{align*}
where $f^{:A\rightarrow B}$ is an arbitrary function, $M$, $N$
are arbitrary monads, and $\phi:M\leadsto N$ is an arbitrary monad
morphism. The problem is either to prove that any such $\varepsilon$
must be an identity function, $\varepsilon=\text{id}^{:M^{A}\rightarrow M^{A}}$,
or to show an example of such $\varepsilon$ not equal to identity.\footnote{See discussion here: \texttt{\href{https://stackoverflow.com/questions/61444425/}{https://stackoverflow.com/questions/61444425/}}}

If it were possible to show that any natural monad morphism $M\leadsto M$
is identity, there would be no need to verify the non-degeneracy law
for monad transformers\textsf{'} base runners (see page~\pageref{par:Open-question-monad-id-trans}). 

\subsubsection{Problem \label{par:Problem-monads-3}\ref{par:Problem-monads-3}}

\textsf{``}Rigid functors\textsf{''} are defined in Section~\ref{subsec:Rigid-functors}.

\textbf{(a)} Are there any rigid functors that are not monads? 

\textbf{(b)} Are there any rigid functors that are not applicative?

\textbf{(c)} Is it true that any applicative rigid functor is a monad?

\subsubsection{Problem \label{par:Problem-monads-3-1}\ref{par:Problem-monads-3-1}}

Let $L$ be a fixed monad and $H$ be an $L$-filterable contrafunctor,
then the functor $F^{A}\triangleq H^{A}\rightarrow L^{A}$ is a lawful
monad (see Section~\ref{subsec:Constructions-of-M-filterables}).
What is a monad transformer for the monad $F$? 

Two nontrivial examples of $L$-filterable contrafunctors are $H^{A}\triangleq A\rightarrow L^{Z}$
and $H^{A}\triangleq L^{A}\rightarrow Z$ (where $Z$ is a fixed type).
For these cases, the monad transformers are defined by
\begin{align*}
 & \text{monad: }(A\rightarrow L^{Z})\rightarrow L^{A}\quad,\quad\quad\text{transformer: }(A\rightarrow T_{L}^{M,Z})\rightarrow T_{L}^{M,A}\quad,\\
 & \text{monad: }(L^{A}\rightarrow Z)\rightarrow L^{A}\quad,\quad\quad\text{transformer: }(T_{L}^{M,A}\rightarrow Z)\rightarrow T_{L}^{M,A}\quad,
\end{align*}
where $T_{L}^{M}$ is the monad $L$\textsf{'}s transformer applied to the
foreign monad $M$.

The problem is to implement the monad $F$\textsf{'}s transformer for an arbitrary
$L$-filterable contrafunctor $H$, and to prove that the transformer
laws hold.

\subsubsection{Problem \label{par:Problem-monads-5-2}\ref{par:Problem-monads-5-2}}

Assume an arbitrary unknown monad $M$ and define $L^{A}\triangleq\bbnum 1+A\times M^{L^{A}}$.
Can one define a lawful monad instance for the functor $L$? (This
is the \lstinline!List! monad\textsf{'}s transformer without the outer layer
of $M$. See Exercise~\ref{subsec:Exercise-effectful-list-not-monad}.)

\subsubsection{Problem \label{par:Problem-monads-5-2-1}\ref{par:Problem-monads-5-2-1}}

These questions concern the \lstinline!ListT! transformer: 
\[
T^{A}\triangleq M^{L^{A}}\quad,\quad\quad L^{A}\triangleq\bbnum 1+A\times M^{L^{A}}\quad,
\]
where $M$ is an arbitrary foreign monad. Normally, we cannot implement
fully parametric base runner $T^{A}\rightarrow M^{A}$ because we
cannot have a fully parametric runner $\text{List}^{A}\rightarrow A$
operating on arbitrary types $A$. However, for a monoid type $R$
with binary operation $\oplus_{R}$ and empty element $e_{R}$, the
type signature $\text{List}^{R}\rightarrow R$ is implemented by the
standard \lstinline!reduce! operation: 
\[
\text{reduce}:\text{List}^{R}\rightarrow R\quad,\quad\quad\text{reduce}\triangleq\,\begin{array}{|c||c|}
 & R\\
\hline \bbnum 1 & 1\rightarrow e_{R}\\
R\times\text{List}^{R} & h\times t\rightarrow h\oplus_{R}\overline{\text{reduce}}\,(t)
\end{array}\quad.
\]
We can similarly implement a base runner (\lstinline!brun!) for the
transformer $T_{\text{List}}$ if we restrict its usage to \emph{monoid}
types $R$. The function \lstinline!brun! with the type signature
$M^{L^{R}}\rightarrow M^{R}$ aggregates all elements of the effectful
list into a single value of type $M^{R}$ (which is also a monoid
type):
\[
\text{brun}:M^{L^{R}}\rightarrow M^{R}\quad,\quad\quad\text{brun}\triangleq\text{flm}_{M}\bigg(\,\begin{array}{|c||c|}
 & M^{R}\\
\hline \bbnum 1 & 1\rightarrow\text{pu}_{M}(e_{R})\\
R\times M^{L^{R}} & h\times t\rightarrow\text{pu}_{M}(h)\oplus_{M}\overline{\text{brun}}\,(t)
\end{array}\,\bigg)\quad.
\]
Here, we use the binary operation $\oplus_{M}$ of the monoid $M^{R}$,
defined by
\[
p^{:M^{R}}\oplus_{M}q^{:M^{R}}\triangleq p\triangleright\text{flm}_{M}\big(u^{:R}\rightarrow q\triangleright(v^{:R}\rightarrow u\oplus_{R}v)^{\uparrow M}\big)\quad.
\]

\textbf{(a)} Is \lstinline!brun! a monoid morphism $T^{A}\rightarrow A$?
(Note that $T^{A}$ is a monoid since $T$ is a lawful monad.)

\textbf{(b)} Do the monad morphism laws of \lstinline!brun! hold
when restricted to a monoid type $A$?
\begin{align*}
{\color{greenunder}\text{for all monoid types }A:}\quad & a^{:A}\triangleright\text{pu}_{T}\bef\text{brun}=a^{:A}\triangleright\text{pu}_{M}\quad,\\
{\color{greenunder}\text{composition law}:}\quad & p^{:T^{T^{A}}}\triangleright\text{ftn}_{T}\bef\text{brun }=p^{:T^{T^{A}}}\triangleright\text{brun}\bef\text{brun}^{\uparrow M}\bef\text{ftn}_{M}\quad.
\end{align*}
(If so, Exercise~\ref{subsec:Exercise-traversables-10-1} would show
that \lstinline!brun! is also a \emph{monoid} morphism $M^{L^{A}}\rightarrow M^{A}$.) 

The monoid morphism identity law holds for \lstinline!brun!. Does
the composition law hold?

\subsubsection{Problem \label{subsec:Problem-monatron-lift-reset-and-shift}\ref{subsec:Problem-monatron-lift-reset-and-shift}}

The continuation monad\textsf{'}s operations \lstinline!reset! and \lstinline!shift!
are defined by
\begin{align*}
 & \text{reset}:\forall S.\,\text{Cont}^{R,R}\rightarrow\text{Cont}^{S,R}\quad,\quad\quad\text{reset}\triangleq c^{:\left(R\rightarrow R\right)\rightarrow R}\rightarrow k^{:R\rightarrow S}\rightarrow k(c(\text{id}))\quad,\\
 & \text{shift}:\forall A.\,((A\rightarrow R)\rightarrow\text{Cont}^{R,R})\rightarrow\text{Cont}^{R,A}\quad,\quad\quad\text{shift}\triangleq g^{:\left(A\rightarrow R\right)\rightarrow\text{Cont}^{R,R}}\rightarrow k^{:A\rightarrow R}\rightarrow g(k)(\text{id})\quad.
\end{align*}
How to lift these operations to an arbitrary monad stack $P$ that
contains a continuation monad?\footnote{See \texttt{\href{https://github.com/renormalist/pugs/blob/master/src/Pugs/AST/Eval.hs}{https://github.com/renormalist/pugs/blob/master/src/Pugs/AST/Eval.hs}}
for custom code (in Haskell) that lifts \lstinline!reset! and \lstinline!shift!
to the \lstinline!ContT! monad transformer.} What are the appropriate type signatures for the lifted operations?

\begin{comment}
Failed attempts to verify the composition law:

Write its two sides separately:
\begin{align*}
{\color{greenunder}\text{left-hand side}:}\quad & \text{ftn}_{T}\bef\text{brun}=\text{flm}_{M}(\text{prod})\bef\text{flm}_{M}\bigg(\,\begin{array}{||c|}
1\rightarrow\text{pu}_{M}(e_{R})\\
r\times t\rightarrow\text{pu}_{M}(r)\oplus_{M}\overline{\text{brun}}\,(t)
\end{array}\,\bigg)\\
{\color{greenunder}\text{associativity of }\text{flm}_{M}:}\quad & \quad=\text{flm}_{M}\bigg(\text{prod}\bef\text{flm}_{M}\bigg(\,\begin{array}{||c|}
1\rightarrow\text{pu}_{M}(e_{R})\\
r\times t\rightarrow\text{pu}_{M}(r)\oplus_{M}\overline{\text{brun}}\,(t)
\end{array}\,\bigg)\bigg)\quad,\\
{\color{greenunder}\text{right-hand side}:}\quad & \text{brun}^{T^{R}}\bef\text{brun}^{\uparrow M}\bef\text{ftn}_{M}=\text{flm}_{M}\bigg(\,\begin{array}{||c|}
1\rightarrow\text{pu}_{M}(e_{R})\\
r\times t\rightarrow\text{pu}_{M}(r)\oplus_{M^{T^{R}}}\overline{\text{brun}}{}^{T^{R}}(t)
\end{array}\,\bigg)\bef\text{flm}_{M}(\text{brun})\\
{\color{greenunder}\text{associativity of }\text{flm}_{M}:}\quad & \quad=\text{flm}_{M}\bigg(\,\begin{array}{||c|}
1\rightarrow\text{pu}_{M}(e_{T^{R}})\\
r\times t\rightarrow\text{pu}_{M}(r)\oplus_{M^{T^{R}}}\overline{\text{brun}}{}^{T^{R}}(t)
\end{array}\,\bef\text{flm}_{M}(\text{brun})\bigg)\quad.
\end{align*}
The remaining difference (under $\text{flm}_{M}$) is an equation
between functions of type $L^{M^{L^{R}}}\rightarrow M^{R}$:
\begin{align*}
 & \text{prod}\bef\text{flm}_{M}\bigg(\,\begin{array}{||c|}
1\rightarrow\text{pu}_{M}(e_{R})\\
r\times t\rightarrow\text{pu}_{M}(r)\oplus_{M}\overline{\text{brun}}\,(t)
\end{array}\,\bigg)=\text{prod}\bef\text{brun}\\
 & \quad\overset{?}{=}\,\begin{array}{||c|}
1\rightarrow\text{pu}_{M}(e_{T^{R}})\\
r^{:T^{R}}\times t^{:T^{T^{R}}}\rightarrow\text{pu}_{M}(r)\oplus_{M^{T^{R}}}\overline{\text{brun}}{}^{T^{R}}(t)
\end{array}\,\bef\text{flm}_{M}(\text{brun})\quad.
\end{align*}
It is inconvenient to use matrices at this step because the code of
$\text{flm}_{M}$ is unknown. Instead, we will substitute into both
sides an arbitrary value of type $L^{M^{L^{R}}}$, which can be one
of two possibilities, $\text{Nil}$ or $\bbnum 0+h^{:T^{R}}\times t^{T^{T^{R}}}$.
Substituting $\text{Nil}$, we get:
\begin{align*}
{\color{greenunder}\text{left-hand side}:}\quad & \gunderline{\text{Nil}\triangleright\text{prod}}\bef\text{brun}=\text{Nil}\triangleright\text{pu}_{M}\bef\text{brun}\\
{\color{greenunder}\text{use Eq.~(\ref{eq:listt-brun-derivation1})}:}\quad & \quad=e_{R}\triangleright\text{pu}_{M}\quad.\\
{\color{greenunder}\text{right-hand side}:}\quad & \text{Nil}\triangleright\,\begin{array}{||c|}
1\rightarrow\text{pu}_{M}(e_{T^{R}})\\
r\times t\rightarrow\text{pu}_{M}(r)\oplus_{M}\overline{\text{brun}}\,(t)
\end{array}\,\bef\text{flm}_{M}(\text{brun})\\
 & \quad=e_{T^{R}}\triangleright\gunderline{\text{pu}_{M}\triangleright\text{flm}_{M}}(\text{brun})=e_{R}\triangleright\gunderline{\text{pu}_{T}\bef\text{brun}}\\
{\color{greenunder}\text{identity law of }\text{brun}:}\quad & \quad=e_{R}\triangleright\text{pu}_{M}\quad.
\end{align*}
The two sides are now equal. It remains to substitute the second possibility:
\begin{align*}
{\color{greenunder}\text{left-hand side}:}\quad & (\bbnum 0+h\times t)\triangleright\text{prod}\bef\text{brun}=\\
 & \quad=(\bbnum 0+h\times t)\triangleright\,\begin{array}{||c|}
1\rightarrow\text{Nil}\triangleright\text{pu}_{M}\\
m\times p\rightarrow\text{comb}\,(m)(p\triangleright\text{flm}_{M}(\overline{\text{prod}})
\end{array}\bef\text{brun}\\
 & \quad=\big(\text{comb}\,(h)(t\triangleright\text{flm}_{M}(\overline{\text{prod}}))\big)\triangleright\text{brun}\\
 & \quad=h\triangleright\text{flm}_{M}\big(t\triangleright\text{flm}_{M}(\overline{\text{prod}})\triangleright\xi\big)\bef\text{flm}_{M}\bigg(\,\begin{array}{||c|}
1\rightarrow\text{pu}_{M}(e_{R})\\
r\times t\rightarrow\text{pu}_{M}(r)\oplus_{M}\overline{\text{brun}}\,(t)
\end{array}\,\bigg)\\
 & \quad=h\triangleright\text{flm}_{M}\big(t\triangleright\text{flm}_{M}(\overline{\text{prod}})\bef\xi\bef\text{brun}\big)\\
 & \quad=h\triangleright\text{flm}_{M}\big(t\triangleright\text{flm}_{M}(\overline{\text{prod}})\bef\big)\\
{\color{greenunder}\text{right-hand side}:}\quad & (\bbnum 0+h\times t)\triangleright\,\begin{array}{||c|}
1\rightarrow\text{pu}_{M}(e_{T^{R}})\\
h\times t\rightarrow\text{pu}_{M}(h)\oplus_{M}\overline{\text{brun}}\,(t)
\end{array}\,\bef\text{flm}_{M}(\text{brun})\\
 & \quad=(\text{pu}_{M}(h)\oplus_{M}\overline{\text{brun}}\,(t))\triangleright\text{flm}_{M}(\text{brun})\\
 & \quad=t\triangleright\overline{\text{brun}}\triangleright(v\rightarrow h\oplus_{R}v)^{\uparrow M}\triangleright\text{flm}_{M}(\text{brun})
\end{align*}
This is suspicious: we need to show that an expression $h\triangleright\text{flm}_{M}(t\triangleright...)$
is equal to $t\triangleright...$, and it seems impossible to convert
one into another, given that $h$ and $t$ are arbitrary values.

Note that
\[
\text{pu}_{M}(r^{:R})\oplus_{M}q^{:M^{R}}=r\triangleright\gunderline{\text{pu}_{M}\triangleright\text{flm}_{M}}(u\rightarrow q\triangleright(v\rightarrow u\oplus_{R}v)^{\uparrow M})=q\triangleright(v\rightarrow r\oplus_{R}v)^{\uparrow M}\quad.
\]
In particular,
\[
\text{pu}_{M}(p)\oplus_{M}\text{pu}_{M}(q)=q\triangleright\text{pu}_{M}\triangleright(v\rightarrow p\oplus_{R}v)^{\uparrow M}=q\triangleright(v\rightarrow p\oplus_{R}v)\triangleright\text{pu}_{M}=\text{pu}_{M}(p\oplus_{R}q)\quad.
\]
We also have the property of \lstinline!comb!:
\begin{align*}
 & \big(\text{comb}\,(p)(q)\big)\triangleright\text{flm}_{M}(g)=p\triangleright\text{flm}_{M}(q\triangleright\xi)\bef\text{flm}_{M}(g)=p\triangleright\text{flm}_{M}((q\triangleright\xi)\bef\text{flm}_{M}(g))\\
 & =p\triangleright\text{flm}_{M}\bigg(\begin{array}{||c|}
1\rightarrow q\\
h\times t\rightarrow\text{pu}_{M}\big(\bbnum 0+h\times\overline{\text{comb}}\,(t)(q)
\end{array}\,\bef\text{flm}_{M}(g)\bigg)
\end{align*}
\end{comment}

