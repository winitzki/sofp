\global\long\def\bef{\forwardcompose}%
\global\long\def\bbnum#1{\custombb{#1}}%


\chapter{Summary of the book\label{chap:Summary-of-the}}

The book has been written in a tutorial format, motivating and deriving
the results gradually. This makes it easier to learn but harder to
navigate. In this summary chapter, the results are listed without
detailed explanations or proofs, referring to the places of the book
where more detail is found.

\section{Main points and results by chapter}

\subsection{In Chapter~\ref{chap:1-Values,-types,-expressions,}}

Standard mathematical notation such as $\sum_{n=1}^{100}(n^{2}+n)$
implicitly uses nameless functions.

Functional programming improves upon this kind of mathematical notation
by using nameless functions explicitly and consistently. For example,
the computation $\sum_{n=1}^{100}(n^{2}+n)$ is implemented in Scala
as \lstinline!(1 to 100).map(n => n*n + n).sum!.

The methods \lstinline!sum! and \lstinline!product! are examples
of \textsf{``}aggregations\textsf{''}. The method \lstinline!filter! selects some
elements from a sequence; \lstinline!takeWhile! truncates a sequence
once a certain condition is achieved. Those methods are examples of
\textsf{``}transformations\textsf{''}.

Functional programming means formulating the solution of a problem
as a mathematical expression and then translating the mathematical
formula into code. Mathematical formulas do not use loops and do not
modify variables. Instead, iteration is expressed using special operators
such as $\sum$. So, functional programming encodes transformations
and aggregations by using special  operations (\lstinline!map!, \lstinline!filter!,
and so on) instead of loops.

\subsection{In Chapter~\ref{chap:2-Mathematical-induction}}

{*}{*}{*}

\subsection{In Appendix~\ref{app:Proofs-of-naturality-parametricity}}

This appendix studies \textsf{``}parametricity\textsf{''} properties that apply to
all fully parametric code:

A given type constructor may have one fully parametric and lawful
implementation of the \lstinline!Functor! or \lstinline!Contrafunctor!
typeclass instance. (For most other typeclasses, such as \lstinline!Filterable!
or \lstinline!Monad!, many type constructors have several inequivalent
and lawful typeclass instances.) The unique implementations are defined
by the type constructions from Sections~\ref{subsec:f-Functor-constructions}
and~\ref{subsec:f-Contrafunctor-constructions}.

The lifting methods of any fully parametric bifunctor, profunctor,
or bi-contrafunctor obey the commutativity law such as Eq.~(\ref{eq:f-fmap-fmap-bifunctor-commutativity}). 

Any fully parametric expression $t:\forall A.\,Q^{A}$ satisfies the
relational naturality law~(\ref{eq:relational-naturality-law-1}).
In general, the relational naturality law expresses a property of
\emph{relations} rather than functions and is not equivalent to any
equation satisfied by $t$. The chapter explains how relations are
defined and what operations are available for relations, and shows
the proof of the relational naturality law.

All fully parametric functions of type $P^{A,A}\rightarrow Q^{A,A}$
(where $P$, $Q$ are profunctors) obey the dinaturality law~(\ref{eq:dinaturality-law-for-profunctors}).
The form of the law depends only on the function\textsf{'}s type signature
and applies to all fully parametric implementations of that type signature.

If the type signature of $t$ satisfies the conditions of Statements~\ref{subsec:Statement-post-wedge-entails-strong-dinaturality}
or~\ref{subsec:Statement-functor-post-pre-wedge}, the function $t$
satisfies the \textsf{``}strong dinaturality\textsf{''} law~(\ref{eq:strong-dinaturality-law}),
which gives more information than the dinaturality law but is simpler
to use than the relational naturality law~(\ref{eq:relational-naturality-law-1}).

\section{Topics not covered in this book}

This book focuses on mathematical theory that has proven its relevance
to the functional programming practice. This section lists some topics
that were omitted from this edition of the book, and explains why.

\subsection{Trampolines and stack-safe recursion in functor blocks}

Recursion with applicative and monadic functors. Cats\textsf{'} monad with
extra method for stack safety. Trampolines and free monads can be
stack safe.

\subsection{Strictness, laziness, termination}

\subsection{Combined typeclass laws}

\subsection{Lenses, prisms, and other \textquotedblleft functional optics\textquotedblright}

\subsection{Comonads and comonad transformers}

\subsection{Dependent types}

\subsection{Linear types and other non-standard types}

{*}{*}{*}

\section{Additional exercises and problems\label{chap:Exercises-in-AFTT}}

The following is a sample set of problems that can be solved using
techniques developed in this book.

\subsection{Exercises\index{exercises}}

\subsubsection{Exercise \label{par:Exercise-additional}\ref{par:Exercise-additional}}

Find the smallest integer expressible as a sum of two cubed integers
in more than one way.

\subsubsection{Exercise \label{par:Exercise-additional-18}\ref{par:Exercise-additional-18}}

Show that the following type signatures have \emph{no} fully parametric
implementations:

\begin{lstlisting}
def f[A]: Option[A] => A
def g[A, B]: (A => B) => A
def h[A, B]: (A => B) => (B => A)
def k[A, B, C]: (A => B) => (B => C) => (C => A)
\end{lstlisting}

Hint: set some type parameters to the void type (\lstinline!Nothing!).

\subsubsection{Exercise \label{par:Exercise-additional-1}\ref{par:Exercise-additional-1}}

Implement a function \lstinline!norepeat! that removes consecutive
repetitions from sequences:
\begin{lstlisting}
def norepeat[A]: Seq[A] => Seq[A] = ???

scala> norepeat(Seq(1,2,2,1,1,3,3,3,0,3))
res0: Seq[Int] = Seq(1,2,1,3,0,3)
\end{lstlisting}


\subsubsection{Exercise \label{par:Exercise-additional-2}\ref{par:Exercise-additional-2}}

Read a text file, split it by spaces into words, and print the word
counts in decreasing order.%
\begin{comment}
\begin{enumerate}
\item FPIS exercise 2.2: Check whether a sequence \lstinline!Seq[A]! is
sorted according to a given ordering function of type \lstinline!(A, A) => Boolean!.
\item FPIS exercise 3.24: Implement a function \lstinline!hasSubsequence!
that checks whether a \lstinline!List! contains another \lstinline!List!
as a subsequence. For instance, \lstinline!List(1,2,3,4)! would have
\lstinline!List(1,2)!, \lstinline!List(2,3)!, and \lstinline!List(4)!
as subsequences, among others. (Dynamic programming?)
\end{enumerate}
\end{comment}


\subsubsection{Exercise \label{par:Exercise-additional-3}\ref{par:Exercise-additional-3}}

(Exercise 4-1 from Hu Zhenjiang\textsf{'}s course \texttt{\href{http://www.prg.nii.ac.jp/course/2015/msp15/}{http://www.prg.nii.ac.jp/course/2015/msp15/}})
Express the \lstinline!filter! method for sequences via \lstinline!flatMap!:
\begin{lstlisting}
def filter[A](p: A => Boolean)(s: Seq[A]): Seq[A] = s.flatMap { a => ??? }
\end{lstlisting}


\subsubsection{Exercise \label{par:Exercise-additional-4}\ref{par:Exercise-additional-4}}

(Bird-de Moor, page 20) Derive the following identity between functions
$F^{A}\rightarrow F^{A}$, for any filterable functor $F$ and any
predicate $p^{:A\rightarrow\bbnum 2}$: 
\[
\text{filt}_{F}(p)=(\Delta\bef\text{id}\boxtimes p)^{\uparrow F}\bef\text{filt}_{F}(\pi_{2})\bef\pi_{1}^{\uparrow F}\quad.
\]


\subsubsection{Exercise \label{par:Exercise-additional-5}\ref{par:Exercise-additional-5}}

Define a monoid of partial functions with fixed types $P\rightarrow Q$:
\begin{lstlisting}
final case class PFM[P, Q](pf: PartialFunction[P, Q])
// After defining a monoid instance, the following code must work:
val p1 = PFM[Option[Int], String] { case Some(3) => "three" }
val p2 = PFM[Option[Int], String] {
  case Some(20)   => "twenty"
  case None       => "empty"
}
p1 |+| p2 // Must be the same as the concatenation of all `case` clauses.
\end{lstlisting}


\subsubsection{Exercise \label{par:Exercise-additional-6}\ref{par:Exercise-additional-6}}

Consider a typeclass called \textsf{``}\lstinline!Splittable!\textsf{''} for functors\index{splittable functors}
$F$ that have an additional method:
\[
\text{split}^{A,B}:F^{A+B}\rightarrow F^{A}+B\quad,
\]
satisfying a \index{non-degeneracy law!of splittable functors}non-degeneracy
law:
\[
(x^{:A}\rightarrow x+\bbnum 0^{:B})^{\uparrow F}\bef\text{split}=y^{:F^{A}}\rightarrow y+\bbnum 0^{:B}\quad,
\]
and a special associativity law\index{associativity law!of splittable functors},
which is an equation for functions of type $F^{A+B+C}\rightarrow F^{A}+B+C$:
\[
\text{split}^{A+B,C}\bef\,\begin{array}{|c||cc|}
 & F^{A}+B & C\\
\hline F^{A+B} & \text{split}^{A,B} & \bbnum 0\\
C & \bbnum 0 & \text{id}
\end{array}\,=\text{split}^{A,B+C}\quad.
\]
Show that all polynomial functors $F^{\bullet}$ belong to this typeclass.
Show that exponential functors such as $F^{A}\triangleq Z\rightarrow A$
do not belong to this typeclass.

\subsubsection{Exercise \label{par:Exercise-additional-7}\ref{par:Exercise-additional-7}}

Given a monad $M$ and a fixed type $Z$, consider the functor $F^{A}\triangleq(A\rightarrow M^{Z})\rightarrow Z$.
Show that $F$ is a semimonad but not a full monad. Hint: use the
flipped Kleisli technique.

\subsubsection{Exercise \label{par:Exercise-additional-8}\ref{par:Exercise-additional-8}}

Given two fixed types $P$, $Q$ that are not known to be equivalent,
consider the contrafunctors $F^{A}\triangleq\left(\left(A\rightarrow P\right)\rightarrow P\right)\rightarrow Q$
and $G^{A}\triangleq A\rightarrow Q$. Show that there exist natural
transformations $F^{\bullet}\leadsto G^{\bullet}$ and $G^{\bullet}\leadsto F^{\bullet}$.
Show that these transformations are \emph{not} isomorphisms.

\subsubsection{Exercise \label{par:Exercise-additional-9}\ref{par:Exercise-additional-9}}

Given two fixed types $P$, $Q$ that are not known to be equivalent,
show that the functor $L^{A}\triangleq\left(\left(\left(A\rightarrow P\right)\rightarrow Q\right)\rightarrow Q\right)\rightarrow P$
is a semimonad but not a full monad. (When $P\cong Q$, the functor
$L$ is also not a full monad because $L$ is then equivalent to a
composition of the continuation monad with itself. See Exercise~\ref{subsec:Exercise-monad-composition-mm}.)

\subsubsection{Exercise \label{par:Exercise-additional-10}\ref{par:Exercise-additional-10}}

When $M$ is a given semimonad, show that $M\circ M\circ...\circ M$
(with finitely many $M$) is also a lawful semimonad.

\subsubsection{Exercise \label{par:Exercise-additional-11}\ref{par:Exercise-additional-11}}

If $M$ is any lawful monad then $M^{A+M^{A}}$ is also a lawful monad.

\subsubsection{Exercise \label{par:Exercise-additional-12}\ref{par:Exercise-additional-12}}

If $M$ is a commutative monad then $M\circ M$ is also a lawful commutative
monad.

\subsubsection{Exercise \label{par:Exercise-additional-13}\ref{par:Exercise-additional-13}}

If $M^{\bullet}$ is a commutative monad and $W$ is a commutative
monoid then the monoid $M^{W}$ is commutative.

\subsubsection{Exercise{*}{*} \label{par:Exercise-additional-14}\ref{par:Exercise-additional-14}}

Define a type constructor \lstinline!Triang[A]! representing \textsf{``}triangular
matrices\textsf{''} with elements of type \lstinline!A!. Example values $t_{1}$,
$t_{2}$, $t_{3}$ of type \lstinline!Triang[A]! are:
\[
t_{1}=\left|\begin{array}{c}
a_{1}\end{array}\right|\quad,\quad\quad t_{2}=\left|\begin{array}{cc}
a_{1}\\
a_{2} & a_{3}
\end{array}\right|\quad,\quad\quad t_{3}=\left|\begin{array}{cccc}
a_{1}\\
a_{2} & a_{3}\\
a_{4} & a_{5} & a_{6}\\
a_{7} & a_{8} & a_{9} & a_{10}
\end{array}\right|\quad.
\]
Unlike \lstinline!List[List[A]]!, it should \emph{not} be possible
to have a value of type \lstinline!Triang[A]! that has an unexpected
shape:
\[
t=\left|\begin{array}{cccc}
a_{1}\\
a_{2} & a_{3}\\
a_{7} & a_{8} & a_{9} & a_{10}
\end{array}\right|\quad\text{is not of type }\text{Triang}^{A}\quad.
\]
 Implement \lstinline!Functor!, \lstinline!Applicative!, and \lstinline!Traversable!
instances for \lstinline!Triang!.

\subsubsection{Exercise \label{par:Exercise-additional-15}\ref{par:Exercise-additional-15}}

Use the Curry-Howard correspondence\index{Curry-Howard correspondence}
and the algorithms LJ or LJT\index{LJT algorithm} to prove that there
exists only one fully parametric function with type signature $\forall A.\,((A\rightarrow A)\rightarrow A)\rightarrow A$,
or in Scala:
\begin{lstlisting}
def f[A]: ((A => A) => A) => A
\end{lstlisting}
From that, prove the type equivalence $\forall A.\,((A\rightarrow A)\rightarrow A)\rightarrow A\cong\bbnum 1$.

\subsubsection{Exercise{*}{*} \label{par:Problem-Peirce-law}\ref{par:Problem-Peirce-law}}

Consider the functor $F^{R}$ defined by:
\[
F^{R}\triangleq\forall A.\,((A\rightarrow R)\rightarrow A)\rightarrow A\quad,
\]
where all functions of type $F^{R}$ are assumed to be fully parametric.
Show that $F^{R}\cong R$.

Note that \index{Peirce\textsf{'}s law}Peirce\textsf{'}s law (see Eq.~(\ref{eq:ch-example-3-peirce-law}))
is expressed as the type $\forall R.\,F^{R}$. Peirce\textsf{'}s law does not
hold in the constructive logic. The Curry-Howard correspondence says
that the corresponding type $\forall R.\,F^{R}$ should be void, and
it is: $\forall R.\,F^{R}=\forall R.\,R=\bbnum 0$. 


\subsubsection{Exercise{*}{*} \label{par:Problem-Peirce-law-1}\ref{par:Problem-Peirce-law-1}}

 Consider the profuctor $F^{R,S}$ defined by:
\[
F^{R,S}\triangleq\forall A.\,((A\rightarrow A)\rightarrow R)\rightarrow S\quad,
\]
where all functions of type $F^{R,S}$ are assumed to be fully parametric. 

An equivalent (but not really simpler) type expression is:
\[
F^{R,S}\triangleq(\exists A.\,(A\rightarrow A)\rightarrow R)\rightarrow S\quad.
\]
So, we have $F^{R,S}=G^{R}\rightarrow S$, where the functor $G^{R}$
is defined by:
\[
G^{R}\triangleq\exists A.\,(A\rightarrow A)\rightarrow R\quad.
\]
Show that $G^{R}\cong R$ and $F^{R,S}\cong R\rightarrow S$.


\subsubsection{Exercise \label{par:Exercise-additional-16}\ref{par:Exercise-additional-16}}

Define a monad transformer $T_{\text{Cod}_{F}^{L}}^{M,A}$ for the
composed codensity monad (Exercise~\ref{subsec:Exercise-combined-codensity-monad})
with type parameters $F$ (an arbitrary but fixed functor), $L$ (an
arbitrary but fixed monad), $M$ (a foreign monad), and $A$ (the
value type). Find out which laws hold for that transformer.

\subsubsection{Exercise{*}{*} \label{par:Exercise-additional-16-1}\ref{par:Exercise-additional-16-1}}

Consider the (non-covariant) type constructor $G^{A}\triangleq A\rightarrow A$.\footnote{See \texttt{\href{https://stackoverflow.com/questions/72490608/}{https://stackoverflow.com/questions/72490608/}}
for discussion about monads having multiple transformers.}

\textbf{(a)} Show that codensity monad on $G$ ($\text{Cod}^{G,\bullet}$)
is equivalent to $\text{List}^{\bullet}$ via monad morphisms.

\textbf{(b)} Show that the corresponding monad transformer: 
\[
T_{\text{Cod}^{G}}^{M,A}\triangleq\forall R.\,(A\rightarrow M^{G^{R}})\rightarrow M^{G^{R}}=\forall R.\,(A\rightarrow M^{R\rightarrow R})\rightarrow M^{R\rightarrow R}
\]
is \emph{not} equivalent to the \lstinline!List! monad\textsf{'}s standard
transformer (\lstinline!ListT!) shown in Section~\ref{subsec:Transformer-for-the-List-monad}.

\textbf{(c)} Show that the type constructor $U$ defined by:
\[
U^{M^{\bullet},A}\triangleq\forall R.\,(A\rightarrow G^{M^{R}})\rightarrow G^{M^{R}}=\forall R.\,(A\rightarrow M^{R}\rightarrow M^{R})\rightarrow M^{R}\rightarrow M^{R}
\]
is also a lawful monad transformer (with the foreign monad $M$) for
the \lstinline!List! monad. Show that the transformer $U$ (known
the \textsf{``}\lstinline!LogicT! monad transformer\textsf{''}\footnote{See \texttt{\href{https://github.com/Bodigrim/logict}{https://github.com/Bodigrim/logict}}
for an example implementation in Haskell.}) is not equivalent to that defined in \textbf{(b)}. 

\textbf{(d)} Generalize \textbf{(c)} using an arbitrary (covariant)
functor $F$ and two fixed types $P$, $Q$:
\[
V^{F^{\bullet},P,Q,M^{\bullet},A}\triangleq\forall R.\,(A\rightarrow F^{M^{R}}\rightarrow P\times M^{R}+Q)\rightarrow F^{M^{R}}\rightarrow P\times M^{R}+Q\quad.
\]
Show that there exists a monad morphism $M^{A}\rightarrow V^{F^{\bullet},P,Q,M^{\bullet},A}$,
and that the converse function of type $V^{F^{\bullet},P,Q,M^{\bullet},A}\rightarrow M^{A}$
exists when $Q=\bbnum 0$ (but is \emph{not} a monad morphism).

\textbf{(e)} Show that the Church-encoded free monoid on $A$ (see
Section~\ref{subsec:Free-constructions-for-inductive-typeclasses}):
\[
\text{FM}^{A}\triangleq\forall X^{:\text{Monoid}}.\,(A\rightarrow X)\rightarrow X
\]
can be modified to the type constructor denoted by \lstinline!FMT!:
\[
\text{FMT}^{M^{\bullet},A}\triangleq\forall X^{:\text{Monoid}}.\,(A\rightarrow M^{X})\rightarrow M^{X}\quad,
\]
which is a lawful monad transformer (with the foreign monad $M$)
for the \lstinline!List! monad. Show that this transformer is not
equivalent to the transformers defined in \textbf{(b)}, \textbf{(c)}. 

\subsubsection{Exercise{*}{*} \label{par:Exercise-additional-17}\ref{par:Exercise-additional-17}}

\textbf{(a)} For any fixed type $Z$, functor $F$ and lawful monad
$P$, show that $L^{A}\triangleq F^{A\rightarrow P^{Z}}\rightarrow P^{A}$
is a lawful monad.

\textbf{(b)} Show that $L$\textsf{'}s monad transformer is $T_{L}^{M,A}\triangleq F^{A\rightarrow T_{P}^{M,Z}}\rightarrow T_{P}^{M,A}$,
where $T_{P}^{M,A}$ is $P$\textsf{'}s monad transformer.

\subsection{Open problems\index{open problems}}

The author of this book does not know how to answer the following
questions and also could not find any answers in existing books or
papers.

\subsubsection{Problem \label{par:Problem-monads-1}\ref{par:Problem-monads-1}}

Do all polynomial functors of the form $P_{n}^{A}\triangleq\bbnum 1+\overbrace{A\times A\times...\times A}^{n\text{ times, }n\ge2}$
fail to be monads? An example is the functor $P_{2}^{A}\triangleq\bbnum 1+A\times A$,
which is not a monad because all possible implementations of \lstinline!pure!
and \lstinline!flatMap! methods for $P_{2}$ fail the monad laws.\footnote{See discussion here: \texttt{\href{https://stackoverflow.com/questions/49742377}{https://stackoverflow.com/questions/49742377}}}

\subsubsection{Problem \label{par:Problem-monads}\ref{par:Problem-monads}}

Section~\ref{subsec:Constructions-of-polynomial-monads} shows four
constructions that make new monads:
\begin{enumerate}
\item The polynomial monad $F^{A}\triangleq Z+W\times A$, where $W$ is
a monoid and $Z$ is a fixed type.
\item The free pointed monad $L^{A}\triangleq A+F^{A}$, where $F$ is a
monad.
\item The product monad $L^{A}\triangleq F^{A}\times G^{A}$, where $F$
and $G$ are monads.
\item The monad $L^{A}\triangleq F^{Z+W\times A}$, where $F$ is a monad,
$W$ is a monoid, and $Z$ is a fixed type.
\end{enumerate}
If we do not assume any existing monads and just keep applying these
constructions, we will obtain a number of polynomial monads. But are
there any polynomial monads \emph{not} obtained by a chain of these
constructions?

\subsubsection{Problem \label{subsec:Problem-co-pointed-applicative}\ref{subsec:Problem-co-pointed-applicative}}

By Statement~\ref{subsec:Statement-co-pointed-applicative-example},
any co-pointed applicative functor of the form $L^{A}\triangleq A\times G^{A}$
satisfies the compatibility law~(\ref{eq:compatibility-law-of-extract-and-zip}).
Statement~\ref{subsec:Statement-co-pointed-applicative-example-failing-compatibility-law}
shows that $L^{A}\triangleq Z\times\left(Z\rightarrow A\right)$ is
applicative and co-pointed but fails the compatibility law. Does there
exist any co-pointed applicative functor that satisfies the law~(\ref{eq:compatibility-law-of-extract-and-zip})
but is \emph{not} of the form $A\times G^{A}$ with some applicative
functor $G^{\bullet}$? 

\subsubsection{Problem \label{par:Problem-traverse-law}\ref{par:Problem-traverse-law}}

Show that the applicative naturality law~(\ref{eq:traverse-applicative-naturality-law})
and the composition law~(\ref{eq:composition-law-of-traverse}) of
\lstinline!traverse! guarantee that \lstinline!traverse! collects
each $F$-effect exactly once. (Can we use a composition of an applicative
functor $F$ with a \lstinline!State! monad to count the number of
times $F$-effects were collected?)

\subsubsection{Problem \label{par:Problem-monads-2}\ref{par:Problem-monads-2}}

Monad transformers are defined in different ways for different monads.
If someone comes up with a new monad, it is not certain that the new
monad\textsf{'}s transformer will be obtained through one of the known methods.
Can we prove that a monad transformer will exist for every monad whose
\lstinline!pure! and \lstinline!flatMap! methods are implemented
via fully parametric code? Given the code for an arbitrary monad,
can we derive an implementation of the corresponding monad transformer?

\subsubsection{Problem \label{par:Problem-monads-5-1}\ref{par:Problem-monads-5-1}}

For certain monads $L$, the monad transformers $T_{L}$ can be defined
using a suitable \lstinline!swap! function. Is this always the case
for any monad stacks built out of such monads? If each of the monads
$L_{1}$, $L_{2}$, ..., $L_{k}$ admits a transformer defined via
a lawful \lstinline!swap! function, will the monad $L_{1}\varangle L_{2}\varangle...\varangle L_{k}$
also admit a transformer with a \lstinline!swap! function? (See Section~\ref{subsec:Does-a-composition-have-swap}
for some partial results.)

\subsubsection{Problem \label{par:Problem-identity-natural-monad-morphism}\ref{par:Problem-identity-natural-monad-morphism}}

Are there any monadically natural monad morphisms $M\leadsto M$ that
are not identity functions? (Equivalently, any non-identical natural
transformations $\text{Id}^{\bullet}\leadsto\text{Id}^{\bullet}$
between identity functors in the category of monads?) If it were possible
to prove that any natural monad morphism $M\leadsto M$ equals an
identity function, there would be no need to verify the non-degeneracy
law for monad transformers\textsf{'} base runners (see page~\pageref{par:Open-question-monad-id-trans}).

We look for a monad morphism $\varepsilon^{M,A}:M^{A}\rightarrow M^{A}$
that is defined for all monads $M$ and is monadically natural in
the parameter $M$. So, the function $\varepsilon$ must satisfy the
following laws:
\begin{align*}
{\color{greenunder}\text{naturality law}:}\quad & \varepsilon^{M,A}\bef(f^{:A\rightarrow B})^{\uparrow M}=(f^{:A\rightarrow B})^{\uparrow M}\bef\varepsilon^{M,B}\quad,\\
{\color{greenunder}\text{monad morphism laws}:}\quad & \text{pu}_{M}\bef\varepsilon=\text{pu}_{M}\quad,\quad\quad\varepsilon^{\uparrow M}\bef\varepsilon\bef\text{ftn}_{M}=\text{ftn}_{M}\bef\varepsilon\quad,\\
{\color{greenunder}\text{monadic naturality law}:}\quad & \varepsilon^{M,A}\bef\phi^{:M^{:A}\rightarrow N^{A}}=\phi^{:M^{A}\rightarrow N^{A}}\bef\varepsilon^{N,A}\quad,
\end{align*}
where $f^{:A\rightarrow B}$ is an arbitrary function, $M$ and $N$
are arbitrary monads, and $\phi:M\leadsto N$ is an arbitrary monad
morphism. We need to prove that any such $\varepsilon$ must be an
identity function, $\varepsilon=\text{id}^{:M^{A}\rightarrow M^{A}}$,
or to find an example of such $\varepsilon$ not equal to identity.\footnote{See discussion here: \texttt{\href{https://stackoverflow.com/questions/61444425/}{https://stackoverflow.com/questions/61444425/}}}

\subsubsection{Problem \label{par:Problem-monads-3}\ref{par:Problem-monads-3}}

\textsf{``}Rigid functors\textsf{''} are\index{rigid functors!open questions} defined
in Section~\ref{subsec:Rigid-functors}.

\textbf{(a)} Are there any rigid functors that are not monads? 

\textbf{(b)} Are there any rigid functors that are not applicative?

\textbf{(c)} Is it true that any applicative rigid functor is a monad?

\subsubsection{Problem \label{par:Problem-monads-3-1}\ref{par:Problem-monads-3-1}}

Let $L$ be a fixed monad and $H$ be an $L$-filterable contrafunctor.
Then the functor $F^{A}\triangleq H^{A}\rightarrow L^{A}$ is a lawful
monad (see Section~\ref{subsec:Constructions-of-M-filterables}).
What is a monad transformer for the monad $F$? 

Two nontrivial examples of $L$-filterable contrafunctors are $H^{A}\triangleq A\rightarrow L^{Z}$
and $H^{A}\triangleq L^{A}\rightarrow Z$ (where $Z$ is a fixed type).
For these cases, the monad transformers are defined by:
\begin{align*}
 & \text{monad: }(A\rightarrow L^{Z})\rightarrow L^{A}\quad,\quad\quad\text{transformer: }(A\rightarrow T_{L}^{M,Z})\rightarrow T_{L}^{M,A}\quad,\\
 & \text{monad: }(L^{A}\rightarrow Z)\rightarrow L^{A}\quad,\quad\quad\text{transformer: }(T_{L}^{M,A}\rightarrow Z)\rightarrow T_{L}^{M,A}\quad,
\end{align*}
where $T_{L}^{M}$ is the monad $L$\textsf{'}s transformer applied to the
foreign monad $M$.

The problem is to implement the monad $F$\textsf{'}s transformer for an arbitrary
$L$-filterable contrafunctor $H$ and to prove that the transformer
laws hold.

\subsubsection{Problem \label{par:Problem-monads-5-2}\ref{par:Problem-monads-5-2}}

Assume an arbitrary unknown monad $M$ and define recursively $L^{A}\triangleq\bbnum 1+A\times M^{L^{A}}$.
Can one define a lawful monad instance for the functor $L$? (This
is the \lstinline!List! monad\textsf{'}s transformer without the outer layer
of $M$. See Exercise~\ref{subsec:Exercise-effectful-list-not-monad}.)

\subsubsection{Problem \label{par:Problem-monads-5-2-1}\ref{par:Problem-monads-5-2-1}}

These questions concern the monad transformer \lstinline!ListT! (here
denoted just by $T$): 
\[
T^{A}\triangleq M^{L^{A}}\quad,\quad\quad L^{A}\triangleq\bbnum 1+A\times M^{L^{A}}\quad,
\]
where $M$ is an arbitrary foreign monad. Normally, we cannot implement
fully parametric base runner $T^{A}\rightarrow M^{A}$ because we
cannot have a fully parametric runner $\text{List}^{A}\rightarrow A$
operating on arbitrary types $A$. However, for a \emph{monoid} type
$R$ with binary operation $\oplus_{R}$ and empty element $e_{R}$,
the type signature $\text{List}^{R}\rightarrow R$ is implemented
by the standard \lstinline!reduce! operation: 
\[
\text{reduce}:\text{List}^{R}\rightarrow R\quad,\quad\quad\text{reduce}\triangleq\,\begin{array}{|c||c|}
 & R\\
\hline \bbnum 1 & 1\rightarrow e_{R}\\
R\times\text{List}^{R} & h\times t\rightarrow h\oplus_{R}\overline{\text{reduce}}\,(t)
\end{array}\quad.
\]
We can similarly implement a base runner (\lstinline!brun!) for the
transformer $T_{\text{List}}$ if we restrict its usage to \emph{monoid}
types $R$. The function \lstinline!brun! with the type signature
$M^{L^{R}}\rightarrow M^{R}$ aggregates all elements of the effectful
list into a single value of type $M^{R}$ (which is also a monoid
type):
\[
\text{brun}:M^{L^{R}}\rightarrow M^{R}\quad,\quad\quad\text{brun}\triangleq\text{flm}_{M}\bigg(\,\begin{array}{|c||c|}
 & M^{R}\\
\hline \bbnum 1 & 1\rightarrow\text{pu}_{M}(e_{R})\\
R\times M^{L^{R}} & h\times t\rightarrow\text{pu}_{M}(h)\oplus_{M}\overline{\text{brun}}\,(t)
\end{array}\,\bigg)\quad.
\]
Here, we use the binary operation $\oplus_{M}$ of the monoid $M^{R}$,
which is defined by:
\[
p^{:M^{R}}\oplus_{M}q^{:M^{R}}\triangleq p\triangleright\text{flm}_{M}\big(u^{:R}\rightarrow q\triangleright(v^{:R}\rightarrow u\oplus_{R}v)^{\uparrow M}\big)\quad.
\]

\textbf{(a)} Is \lstinline!brun! a monoid morphism $T^{A}\rightarrow A$?
(Note that $T^{A}$ is a monoid since $T$ is a lawful monad.) The
monoid morphism identity law holds for \lstinline!brun!. Does the
composition law hold?

\textbf{(b)} Do the monad morphism laws of \lstinline!brun! hold
when restricted to a monoid type $A$?
\begin{align*}
{\color{greenunder}\text{for all monoid types }A:}\quad & a^{:A}\triangleright\text{pu}_{T}\bef\text{brun}=a^{:A}\triangleright\text{pu}_{M}\quad,\\
{\color{greenunder}\text{composition law}:}\quad & p^{:T^{T^{A}}}\triangleright\text{ftn}_{T}\bef\text{brun }=p^{:T^{T^{A}}}\triangleright\text{brun}\bef\text{brun}^{\uparrow M}\bef\text{ftn}_{M}\quad.
\end{align*}
(If so, Exercise~\ref{subsec:Exercise-traversables-10-1} would show
that \lstinline!brun! is also a \emph{monoid} morphism $M^{L^{A}}\rightarrow M^{A}$.)
\begin{comment}
Failed attempts to verify the composition law:

Write its two sides separately:
\begin{align*}
{\color{greenunder}\text{left-hand side}:}\quad & \text{ftn}_{T}\bef\text{brun}=\text{flm}_{M}(\text{prod})\bef\text{flm}_{M}\bigg(\,\begin{array}{||c|}
1\rightarrow\text{pu}_{M}(e_{R})\\
r\times t\rightarrow\text{pu}_{M}(r)\oplus_{M}\overline{\text{brun}}\,(t)
\end{array}\,\bigg)\\
{\color{greenunder}\text{associativity of }\text{flm}_{M}:}\quad & \quad=\text{flm}_{M}\bigg(\text{prod}\bef\text{flm}_{M}\bigg(\,\begin{array}{||c|}
1\rightarrow\text{pu}_{M}(e_{R})\\
r\times t\rightarrow\text{pu}_{M}(r)\oplus_{M}\overline{\text{brun}}\,(t)
\end{array}\,\bigg)\bigg)\quad,\\
{\color{greenunder}\text{right-hand side}:}\quad & \text{brun}^{T^{R}}\bef\text{brun}^{\uparrow M}\bef\text{ftn}_{M}=\text{flm}_{M}\bigg(\,\begin{array}{||c|}
1\rightarrow\text{pu}_{M}(e_{R})\\
r\times t\rightarrow\text{pu}_{M}(r)\oplus_{M^{T^{R}}}\overline{\text{brun}}{}^{T^{R}}(t)
\end{array}\,\bigg)\bef\text{flm}_{M}(\text{brun})\\
{\color{greenunder}\text{associativity of }\text{flm}_{M}:}\quad & \quad=\text{flm}_{M}\bigg(\,\begin{array}{||c|}
1\rightarrow\text{pu}_{M}(e_{T^{R}})\\
r\times t\rightarrow\text{pu}_{M}(r)\oplus_{M^{T^{R}}}\overline{\text{brun}}{}^{T^{R}}(t)
\end{array}\,\bef\text{flm}_{M}(\text{brun})\bigg)\quad.
\end{align*}
The remaining difference (under $\text{flm}_{M}$) is an equation
between functions of type $L^{M^{L^{R}}}\rightarrow M^{R}$:
\begin{align*}
 & \text{prod}\bef\text{flm}_{M}\bigg(\,\begin{array}{||c|}
1\rightarrow\text{pu}_{M}(e_{R})\\
r\times t\rightarrow\text{pu}_{M}(r)\oplus_{M}\overline{\text{brun}}\,(t)
\end{array}\,\bigg)=\text{prod}\bef\text{brun}\\
 & \quad\overset{?}{=}\,\begin{array}{||c|}
1\rightarrow\text{pu}_{M}(e_{T^{R}})\\
r^{:T^{R}}\times t^{:T^{T^{R}}}\rightarrow\text{pu}_{M}(r)\oplus_{M^{T^{R}}}\overline{\text{brun}}{}^{T^{R}}(t)
\end{array}\,\bef\text{flm}_{M}(\text{brun})\quad.
\end{align*}
It is inconvenient to use matrices at this step because the code of
$\text{flm}_{M}$ is unknown. Instead, we will substitute into both
sides an arbitrary value of type $L^{M^{L^{R}}}$, which can be one
of two possibilities, $\text{Nil}$ or $\bbnum 0+h^{:T^{R}}\times t^{T^{T^{R}}}$.
Substituting $\text{Nil}$, we get:
\begin{align*}
{\color{greenunder}\text{left-hand side}:}\quad & \gunderline{\text{Nil}\triangleright\text{prod}}\bef\text{brun}=\text{Nil}\triangleright\text{pu}_{M}\bef\text{brun}\\
{\color{greenunder}\text{use Eq.~(\ref{eq:listt-brun-derivation1})}:}\quad & \quad=e_{R}\triangleright\text{pu}_{M}\quad.\\
{\color{greenunder}\text{right-hand side}:}\quad & \text{Nil}\triangleright\,\begin{array}{||c|}
1\rightarrow\text{pu}_{M}(e_{T^{R}})\\
r\times t\rightarrow\text{pu}_{M}(r)\oplus_{M}\overline{\text{brun}}\,(t)
\end{array}\,\bef\text{flm}_{M}(\text{brun})\\
 & \quad=e_{T^{R}}\triangleright\gunderline{\text{pu}_{M}\triangleright\text{flm}_{M}}(\text{brun})=e_{R}\triangleright\gunderline{\text{pu}_{T}\bef\text{brun}}\\
{\color{greenunder}\text{identity law of }\text{brun}:}\quad & \quad=e_{R}\triangleright\text{pu}_{M}\quad.
\end{align*}
The two sides are now equal. It remains to substitute the second possibility:
\begin{align*}
{\color{greenunder}\text{left-hand side}:}\quad & (\bbnum 0+h\times t)\triangleright\text{prod}\bef\text{brun}=\\
 & \quad=(\bbnum 0+h\times t)\triangleright\,\begin{array}{||c|}
1\rightarrow\text{Nil}\triangleright\text{pu}_{M}\\
m\times p\rightarrow\text{comb}\,(m)(p\triangleright\text{flm}_{M}(\overline{\text{prod}})
\end{array}\bef\text{brun}\\
 & \quad=\big(\text{comb}\,(h)(t\triangleright\text{flm}_{M}(\overline{\text{prod}}))\big)\triangleright\text{brun}\\
 & \quad=h\triangleright\text{flm}_{M}\big(t\triangleright\text{flm}_{M}(\overline{\text{prod}})\triangleright\xi\big)\bef\text{flm}_{M}\bigg(\,\begin{array}{||c|}
1\rightarrow\text{pu}_{M}(e_{R})\\
r\times t\rightarrow\text{pu}_{M}(r)\oplus_{M}\overline{\text{brun}}\,(t)
\end{array}\,\bigg)\\
 & \quad=h\triangleright\text{flm}_{M}\big(t\triangleright\text{flm}_{M}(\overline{\text{prod}})\bef\xi\bef\text{brun}\big)\\
 & \quad=h\triangleright\text{flm}_{M}\big(t\triangleright\text{flm}_{M}(\overline{\text{prod}})\bef\big)\\
{\color{greenunder}\text{right-hand side}:}\quad & (\bbnum 0+h\times t)\triangleright\,\begin{array}{||c|}
1\rightarrow\text{pu}_{M}(e_{T^{R}})\\
h\times t\rightarrow\text{pu}_{M}(h)\oplus_{M}\overline{\text{brun}}\,(t)
\end{array}\,\bef\text{flm}_{M}(\text{brun})\\
 & \quad=(\text{pu}_{M}(h)\oplus_{M}\overline{\text{brun}}\,(t))\triangleright\text{flm}_{M}(\text{brun})\\
 & \quad=t\triangleright\overline{\text{brun}}\triangleright(v\rightarrow h\oplus_{R}v)^{\uparrow M}\triangleright\text{flm}_{M}(\text{brun})
\end{align*}
This is suspicious: we need to show that an expression $h\triangleright\text{flm}_{M}(t\triangleright...)$
is equal to $t\triangleright...$, and it seems impossible to convert
one into another, given that $h$ and $t$ are arbitrary values.

Note that
\[
\text{pu}_{M}(r^{:R})\oplus_{M}q^{:M^{R}}=r\triangleright\gunderline{\text{pu}_{M}\triangleright\text{flm}_{M}}(u\rightarrow q\triangleright(v\rightarrow u\oplus_{R}v)^{\uparrow M})=q\triangleright(v\rightarrow r\oplus_{R}v)^{\uparrow M}\quad.
\]
In particular,
\[
\text{pu}_{M}(p)\oplus_{M}\text{pu}_{M}(q)=q\triangleright\text{pu}_{M}\triangleright(v\rightarrow p\oplus_{R}v)^{\uparrow M}=q\triangleright(v\rightarrow p\oplus_{R}v)\triangleright\text{pu}_{M}=\text{pu}_{M}(p\oplus_{R}q)\quad.
\]
We also have the property of \lstinline!comb!:
\begin{align*}
 & \big(\text{comb}\,(p)(q)\big)\triangleright\text{flm}_{M}(g)=p\triangleright\text{flm}_{M}(q\triangleright\xi)\bef\text{flm}_{M}(g)=p\triangleright\text{flm}_{M}((q\triangleright\xi)\bef\text{flm}_{M}(g))\\
 & =p\triangleright\text{flm}_{M}\bigg(\begin{array}{||c|}
1\rightarrow q\\
h\times t\rightarrow\text{pu}_{M}\big(\bbnum 0+h\times\overline{\text{comb}}\,(t)(q)
\end{array}\,\bef\text{flm}_{M}(g)\bigg)
\end{align*}
\end{comment}
{} 

\subsubsection{Problem \label{subsec:Problem-monatron-lift-reset-and-shift}\ref{subsec:Problem-monatron-lift-reset-and-shift}}

The continuation monad\textsf{'}s operations \lstinline!reset! and \lstinline!shift!
are defined by:
\begin{align*}
 & \text{reset}:\forall S.\,\text{Cont}^{R,R}\rightarrow\text{Cont}^{S,R}\quad,\quad\quad\text{reset}\triangleq c^{:\left(R\rightarrow R\right)\rightarrow R}\rightarrow k^{:R\rightarrow S}\rightarrow k(c(\text{id}))\quad,\\
 & \text{shift}:\forall A.\,((A\rightarrow R)\rightarrow\text{Cont}^{R,R})\rightarrow\text{Cont}^{R,A}\quad,\quad\quad\text{shift}\triangleq g^{:\left(A\rightarrow R\right)\rightarrow\text{Cont}^{R,R}}\rightarrow k^{:A\rightarrow R}\rightarrow g(k)(\text{id})\quad.
\end{align*}
How to lift these operations to an arbitrary monad stack $P$ that
contains a continuation monad?\footnote{See \texttt{\href{https://github.com/renormalist/pugs/blob/master/src/Pugs/AST/Eval.hs}{https://github.com/renormalist/pugs/blob/master/src/Pugs/AST/Eval.hs}}
for custom code (in Haskell) that lifts \lstinline!reset! and \lstinline!shift!
to the \lstinline!ContT! monad transformer.} What are the appropriate type signatures for the lifted operations?

\subsubsection{Problem \label{subsec:Problem-unique-functor-liftings}\ref{subsec:Problem-unique-functor-liftings}}

For any fully parametric type constructor $P^{A}$ covariant in $A$,
the lifting of a function $f^{:A\rightarrow B}$ to $P$ is performed
via the \lstinline!fmap! method of $P$, so that \lstinline!fmap(f)!
is a function of type $P^{A}\rightarrow P^{B}$ denoted by $f^{\uparrow F}$
in this book. The standard code of \lstinline!fmap! is defined by
induction on the type structure of $P$ and satisfies the functor
laws, as shown in Chapter~\ref{chap:Functors,-contrafunctors,-and}.
The question is to show that there is no non-standard, alternative
implementation \lstinline!fmap!$^{\prime}$ that still satisfies
the functor laws. If the code of \lstinline!fmap!$^{\prime}$ is
fully parametric, Section~\ref{sec:Uniqueness-of-functor-and-contrafunctor}
shows that \lstinline!fmap!$^{\prime}=$ \lstinline!fmap!. However,
parametricity (or naturality) does not seem to follow from functor
laws alone. Does there exist an implementation of \lstinline!fmap!$^{\prime}$
that satisfies the functor laws but is \emph{not} fully parametric?
