
\addchap{Preface}

\global\long\def\gunderline#1{\mathunderline{greenunder}{#1}}%
\global\long\def\bef{\forwardcompose}%
\global\long\def\bbnum#1{\custombb{#1}}%
This book is a reference text and a tutorial that teaches functional
programmers how to reason mathematically about types and code, in
a manner directly relevant to software practice. The material ranges
from introductory (Part I) to advanced (Part III). %
\begin{comment}
Readers will need to learn some difficult concepts through prolonged
mental concentration and effort. 
\end{comment}
The book assumes a certain amount of mathematical experience (at about
the level of undergraduate algebra or calculus) as well as some experience
writing code in general-purpose programming languages.

The vision of this book is to explain the mathematical theory that
guides the practice of functional programming. So, all mathematical
developments in this book are motivated by practical programming issues
and are accompanied by Scala code illustrating their usage. For instance,
the laws for standard typeclasses (functors, monads, etc.) are first
motivated heuristically through code examples. Then the laws are formulated
as mathematical equations and proved rigorously. 

To achieve a clearer presentation of the material, the book uses certain
non-standard notations (see Appendix~\ref{chap:Appendix-Notations}
on page~\pageref{chap:Appendix-Notations}) and terminology (Appendix~\ref{chap:Appendix-Glossary-of-terms}
on page~\pageref{chap:Appendix-Glossary-of-terms}). The presentation
is self-contained, defining and explaining all required techniques,
notations, and Scala features. All code examples have been tested
to work but are intended only for explanation and illustration. As
a rule, the code is not optimized for performance. Although the code
examples are in Scala, the material in this book also applies to many
other languages.

A software engineer needs to learn only those few fragments of mathematical
theory that answer questions arising in the programming practice.
So, this book keeps theoretical material at the minimum: \emph{vita
brevis, ars longa}. The scope of the required mathematical knowledge
is limited to first notions of set theory, formal logic, and category
theory. Concepts such as functors or natural transformations arise
from the practice of reasoning about code and are first explained
without reference to category theory. 

This book is \emph{not} an introduction to current theoretical research
in functional programming. Instead, the focus is on material known
to be practically useful. The book organically develops the scope
of theoretical concepts that help programmers write code or answer
practical questions about code. That includes constructions such as
\textsf{``}filterable functors\textsf{''} and \textsf{``}applicative contrafunctors\textsf{''} but
excludes a number of theoretical developments that do not appear to
have significant applications. For instance, this book does not talk
about introduction/elimination rules, strong normalization, complete
partial orders, domain theory, model theory, adjoint functors, co-ends,
pullbacks, or topoi.

The first part of the book introduces functional programming. Readers
already familiar with functional programming could skim the glossary
(Appendix~\ref{chap:Appendix-Glossary-of-terms} on page~\pageref{chap:Appendix-Glossary-of-terms})
for unfamiliar terminology and then start reading Chapter~\ref{chap:5-Curry-Howard}. 

Participation in the meetup \textsf{``}San Francisco Types, Theorems, and
Programming Languages\textsf{''}\footnote{\texttt{\href{https://www.meetup.com/sf-types-theorems-and-programming-languages/}{https://www.meetup.com/sf-types-theorems-and-programming-languages/}}}
initially motivated the author to begin working on this book. Thanks
are due to Adrian King, Hew Wolff, Peter Vanderbilt, and Young-il
Choo for inspiration and support in that meetup. The author appreciates
the work of Joseph Kim and Jim Kleck who did many of the exercises
and reported some errors in earlier versions of this book. The author
also thanks Bill Venners for many helpful comments on the draft, and
Harald Gliebe, Andreas R\"ohler, and Philip Schwarz for contributing
corrections to the text via \texttt{github}. The author is grateful
to Vadim Kuzin, Frederick Pitts, Hew Wolff, and several anonymous
\texttt{github} contributors who reported errors in the draft and
made helpful suggestions, and to Barisere Jonathan for valuable assistance
with setting up automatic builds.

\emph{No generative AI was used for creating or editing this book.}

\addsec{Formatting conventions used in this book}
\begin{itemize}
\item Text in boldface indicates a new concept or term that is being defined
at that place in the text. Italics means logical emphasis. Example:
\end{itemize}
\begin{quotation}
An \textbf{aggregation\index{aggregation}} is a function from a collection
of values to a \emph{single} value.
\end{quotation}
\begin{itemize}
\item Equations are numbered per chapter: Eq.~(\ref{eq:prime-formula-function}).
Statements, examples, and exercises are numbered per subsection: Example~\ref{subsec:ch1-aggr-Example-1}
is in subsection~\ref{subsec:Aggregation-solved-examples}, which
belongs to Chapter~\ref{chap:1-Values,-types,-expressions,}.
\item Scala code is written inline using a small monospaced font: \lstinline!val a = "xyz"!.
Longer code examples are written in separate code blocks and may also
show the Scala interpreter\textsf{'}s output for certain lines:
\begin{lstlisting}[mathescape=true]
val s = (1 to 10).toList

scala> s.product
res0: Int = 3628800 
\end{lstlisting}
\item In the introductory chapters, type expressions and code examples are
written in the Scala syntax. In Chapters~\ref{chap:Higher-order-functions}\textendash \ref{chap:5-Curry-Howard},
the book introduces a mathematical notation for types: e.g., the Scala
type expression \lstinline!((A, B)) => Option[A]! is written as $A\times B\rightarrow\bbnum 1+A$.
Chapters~\ref{chap:Higher-order-functions}\textendash \ref{chap:Reasoning-about-code}
also develop a more concise notation for code. For example, the functor
composition law (in Scala: \lstinline!_.map(f).map(g) == _.map(f andThen g)!)
is written in the code notation as:
\[
f^{\uparrow L}\bef g^{\uparrow L}=\left(f\bef g\right)^{\uparrow L}\quad,
\]
where $L$ is a functor and $f^{:A\rightarrow B}$ and $g^{:B\rightarrow C}$
are arbitrary functions of the specified types. The notation $f^{\uparrow L}$
denotes the function $f$ lifted to the functor $L$ and replaces
Scala\textsf{'}s syntax \lstinline!x.map(f)! where \lstinline!x! is of type
\lstinline!L[A]!. The symbol $\bef$ denotes the forward composition
of functions (Scala\textsf{'}s method \lstinline!andThen!). If the notation
still appears hard to follow after going through Chapters~\ref{chap:5-Curry-Howard}\textendash \ref{chap:Functors,-contrafunctors,-and},
readers will benefit from working through Chapter~\ref{chap:Reasoning-about-code},
which explains the code notation more systematically and clarifies
it with additional examples. Appendix~\ref{chap:Appendix-Notations}
on page~\pageref{chap:Appendix-Notations} summarizes this book\textsf{'}s
notations.
\item Frequently used methods of standard typeclasses, such as Scala\textsf{'}s \lstinline!flatten!,
\lstinline!flatMap!, etc., are denoted by shorter words and are labeled
by the type constructor they belong to. For instance, the methods
\lstinline!pure!, \lstinline!flatten!, and \lstinline!flatMap!
for a monad $M$ are denoted by $\text{pu}_{M}$, $\text{ftn}_{M}$,
and $\text{flm}_{M}$ when writing code formulas and proofs of laws.
\item Derivations are written in a two-column format. The right column contains
formulas in the code notation. The left column gives an explanation
or indicates the property or law used to derive the expression at
right from the previous expression. A green underline shows the parts
of an expression that will be rewritten in the \emph{next} step:
\begin{align*}
{\color{greenunder}\text{expect to equal }\text{pu}_{M}:}\quad & \gunderline{\text{pu}_{M}^{\uparrow\text{Id}}}\bef\text{pu}_{M}\bef\text{ftn}_{M}\\
{\color{greenunder}\text{lifting to the identity functor}:}\quad & =\text{pu}_{M}\bef\gunderline{\text{pu}_{M}\bef\text{ftn}_{M}}\\
{\color{greenunder}\text{left identity law of }M:}\quad & =\text{pu}_{M}\quad.
\end{align*}
When the two-column presentation becomes too wide to fit the page,
the explanations are placed before the next step\textsf{'}s line:
\begin{align*}
 & \gunderline{\text{pu}_{M}^{\uparrow\text{Id}}}\bef\text{pu}_{M}\bef\text{ftn}_{M}\\
 & \quad{\color{greenunder}\text{lifting to the identity functor}:}\quad\\
 & =\text{pu}_{M}\bef\gunderline{\text{pu}_{M}\bef\text{ftn}_{M}}\\
 & \quad{\color{greenunder}\text{left identity law of }M:}\quad\\
 & =\text{pu}_{M}\quad.
\end{align*}
 A green underline is sometimes also used at the last step of a derivation,
to indicate the sub-expression that resulted from the most recent
rewriting. Other than providing hints to help clarify the steps, the
green text and the green underlines \emph{play no role} in symbolic
derivations.
\item The symbol $\square$ is used occasionally to indicate more clearly
the end of a definition, a derivation, or a proof.
\end{itemize}
\cleardoublepage%
\begin{comment}
Necessary to make \textsf{``}Part\textsf{''} appear after a blank page. We are resetting
page numbers in each volume, and this confuses the layout engine.
\end{comment}

