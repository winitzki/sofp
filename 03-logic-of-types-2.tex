%% LyX 2.2.0 created this file.  For more info, see http://www.lyx.org/.
%% Do not edit unless you really know what you are doing.
\documentclass[english]{beamer}
\usepackage[T1]{fontenc}
\usepackage[latin9]{inputenc}
\setcounter{secnumdepth}{3}
\setcounter{tocdepth}{3}
\usepackage{babel}
\usepackage{amstext}
\usepackage{amssymb}
\ifx\hypersetup\undefined
  \AtBeginDocument{%
    \hypersetup{unicode=true,pdfusetitle,
 bookmarks=true,bookmarksnumbered=false,bookmarksopen=false,
 breaklinks=false,pdfborder={0 0 1},backref=false,colorlinks=true}
  }
\else
  \hypersetup{unicode=true,pdfusetitle,
 bookmarks=true,bookmarksnumbered=false,bookmarksopen=false,
 breaklinks=false,pdfborder={0 0 1},backref=false,colorlinks=true}
\fi

\makeatletter

%%%%%%%%%%%%%%%%%%%%%%%%%%%%%% LyX specific LaTeX commands.
%% Because html converters don't know tabularnewline
\providecommand{\tabularnewline}{\\}

%%%%%%%%%%%%%%%%%%%%%%%%%%%%%% Textclass specific LaTeX commands.
 % this default might be overridden by plain title style
 \newcommand\makebeamertitle{\frame{\maketitle}}%
 % (ERT) argument for the TOC
 \AtBeginDocument{%
   \let\origtableofcontents=\tableofcontents
   \def\tableofcontents{\@ifnextchar[{\origtableofcontents}{\gobbletableofcontents}}
   \def\gobbletableofcontents#1{\origtableofcontents}
 }
 \newenvironment{lyxcode}
   {\par\begin{list}{}{
     \setlength{\rightmargin}{\leftmargin}
     \setlength{\listparindent}{0pt}% needed for AMS classes
     \raggedright
     \setlength{\itemsep}{0pt}
     \setlength{\parsep}{0pt}
     \normalfont\ttfamily}%
    \def\{{\char`\{}
    \def\}{\char`\}}
    \def\textasciitilde{\char`\~}
    \item[]}
   {\end{list}}

%%%%%%%%%%%%%%%%%%%%%%%%%%%%%% User specified LaTeX commands.
\usetheme[secheader]{Boadilla}
\usecolortheme{seahorse}
\title[Chapter 3: Logic of Types II]{Chapter 3: The Logic of Types, Part II}
\subtitle{Disjunction types}
\author{Sergei Winitzki}
\date{December 5, 2017}
\institute[ABTB]{Academy by the Bay}
\setbeamertemplate{navigation symbols}{}

\makeatother

\begin{document}
\frame{\titlepage}
\begin{frame}{Tuples with names: ``case classes''}

\begin{itemize}
\item Pair of values: \texttt{\textcolor{blue}{\footnotesize{}val a:\ (Int,
String) = (123, \textquotedbl{}xyz\textquotedbl{})}}{\footnotesize \par}
\item For convenience, we can define a name for this type:\\
 \texttt{\textcolor{blue}{\footnotesize{}type MyPair = (Int, String);
val a:\ MyPair = (123, \textquotedbl{}xyz\textquotedbl{})}}{\footnotesize \par}
\item We can define a name for each value and also for the type:\texttt{\textcolor{blue}{\footnotesize{}}}~\\
\texttt{\textcolor{blue}{\footnotesize{}case class MySocks(size:\ Double,
color:\ String)}}~\\
\texttt{\textcolor{blue}{\footnotesize{}val a:\ MySocks = MySocks(10.5,
\textquotedbl{}white\textquotedbl{})}}{\footnotesize \par}
\item Case classes can be nested: \texttt{\textcolor{blue}{\footnotesize{}}}~\\
\texttt{\textcolor{blue}{\footnotesize{}case class BagOfSocks(socks:\ MySocks,
count:\ Int)}}~\\
\texttt{\textcolor{blue}{\footnotesize{}val bag = BagOfSocks(MySocks(10.5,
\textquotedbl{}white\textquotedbl{}), 6)}}{\footnotesize \par}
\item Parts of the case class can be accessed by name: \texttt{\textcolor{blue}{\footnotesize{}}}~\\
\texttt{\textcolor{blue}{\footnotesize{}val c:\ String = bag.socks.color}}{\footnotesize \par}
\item Parts can be given in any order by using names:\texttt{\textcolor{blue}{\footnotesize{}}}~\\
\texttt{\textcolor{blue}{\footnotesize{}val y = MySocks(color = \textquotedbl{}black\textquotedbl{},
size = 11.0) }}{\footnotesize \par}
\item Default values can be defined for parts: \texttt{\textcolor{blue}{\footnotesize{}}}~\\
\texttt{\textcolor{blue}{\footnotesize{}case class Shirt(color:\ String
= \textquotedbl{}blue\textquotedbl{}, hasHoles:\ Boolean = false)}}~\\
\texttt{\textcolor{blue}{\footnotesize{}val sock = Shirt(hasHoles
= true)}}{\footnotesize \par}
\end{itemize}
\end{frame}

\begin{frame}{Tuples with one element and with zero elements}

\begin{itemize}
\item A tuple type expression \texttt{\textcolor{blue}{\footnotesize{}(Int,
String)}} is special syntax for parameterized type \texttt{\textcolor{blue}{\footnotesize{}Tuple2{[}Int,
String{]}}}{\footnotesize \par}
\item Case class with no parts is called a ``case object''
\item What are tuples with one element or with zero elements?
\begin{itemize}
\item There is no \texttt{\textcolor{blue}{\footnotesize{}Tuple0}} \textendash{}
it is a special type called \texttt{\textcolor{blue}{\footnotesize{}Unit}}{\footnotesize \par}
\end{itemize}
\end{itemize}
\begin{center}
\begin{tabular}{|c|c|}
\hline 
\textbf{Tuples} & \textbf{Case classes}\tabularnewline
\hline 
\hline 
\texttt{\textcolor{blue}{\footnotesize{}(123, \textquotedbl{}xyz\textquotedbl{}):\ Tuple2{[}Int,
String{]}}} & \texttt{\textcolor{blue}{\footnotesize{}case class A(x:\ Int, y:\ String)}}\tabularnewline
\hline 
\texttt{\textcolor{blue}{\footnotesize{}(123,):\ Tuple1{[}Int{]}}} & \texttt{\textcolor{blue}{\footnotesize{}case class B(z:\ Int)}}\tabularnewline
\hline 
\texttt{\textcolor{blue}{\footnotesize{}(): Unit}} & \texttt{\textcolor{blue}{\footnotesize{}case object C}}\tabularnewline
\hline 
\end{tabular}
\par\end{center}
\begin{itemize}
\item Case classes can have one or more type parameters: \\
\texttt{\textcolor{blue}{\footnotesize{}case class Pairs{[}A, B{]}(left:\ A,
right:\ B, count:\ Int)}}{\footnotesize \par}
\item The ``\texttt{\textcolor{blue}{\footnotesize{}Tuple}}'' types could
be defined by this code:\texttt{\textcolor{blue}{\footnotesize{}}}~\\
\texttt{\textcolor{blue}{\footnotesize{}case class Tuple2{[}A, B{]}(\_1:\ A,
\_2:\ B)}}{\footnotesize \par}
\end{itemize}
\end{frame}

\begin{frame}{Pattern-matching syntax for case classes}

Scala allows pattern matching in two places:
\begin{itemize}
\item \texttt{\textcolor{blue}{\footnotesize{}val }}\textcolor{blue}{\emph{\footnotesize{}pattern}}\texttt{\textcolor{blue}{\footnotesize{}
= ...}} (value assignment)
\item \texttt{\textcolor{blue}{\footnotesize{}case }}\textcolor{blue}{\emph{\footnotesize{}pattern}}\texttt{\textcolor{blue}{\footnotesize{}
}}\textcolor{blue}{\footnotesize{}$\Rightarrow$}\texttt{\textcolor{blue}{\footnotesize{}
...}} (partial function)
\end{itemize}
Examples with case classes:
\begin{itemize}
\item \texttt{\textcolor{blue}{\footnotesize{}val a = MySocks(10.5, \textquotedbl{}white\textquotedbl{})}}~\\
\texttt{\textcolor{blue}{\footnotesize{}val MySocks(x, y) = a}}{\footnotesize \par}
\item \texttt{\textcolor{blue}{\footnotesize{}val f:\ BagOfSocks$\Rightarrow$Int
= \{ case BagOfSocks(MySocks(s, c), z)$\Rightarrow$...\}}}{\footnotesize \par}
\item \texttt{\textcolor{blue}{\footnotesize{}def f(b:\ BagOfSocks):\ String
= b match \{ }}~\\
\texttt{\textcolor{blue}{\footnotesize{}\  \ case BagOfSocks(MySocks(s,
c), z) $\Rightarrow$ c}}~\\
\texttt{\textcolor{blue}{\footnotesize{}\}}}{\footnotesize \par}
\item Note: \texttt{\textcolor{blue}{\footnotesize{}s}}, \texttt{\textcolor{blue}{\footnotesize{}c}},
\texttt{\textcolor{blue}{\footnotesize{}z}} are defined as \textbf{pattern
variables} of correct types
\end{itemize}
\end{frame}

\begin{frame}{Disjunction types}

\begin{itemize}
\item Motivational examples:
\begin{itemize}
\item The roots of a quadratic equation can be either a pair, or one, or
none
\item Binary search gives either a found value and an index, or nothing
\item Computations that give a value or an error with a text message
\item Computer game states: several kinds of rooms, types players, etc.
\begin{itemize}
\item Each kind of room may have different sets of properties
\end{itemize}
\end{itemize}
\item We would like to be able to represent \emph{disjunctions} of sets
\begin{itemize}
\item A value that is \emph{either} \texttt{\textcolor{blue}{\footnotesize{}(Complex,
Complex)}} or \texttt{\textcolor{blue}{\footnotesize{}Complex}} or
empty \texttt{\textcolor{blue}{\footnotesize{}()}}{\footnotesize \par}
\item A value that is \emph{either} \texttt{\textcolor{blue}{\footnotesize{}(Int,
Int)}} or empty \texttt{\textcolor{blue}{\footnotesize{}()}}{\footnotesize \par}
\item A value that is \emph{either} an \texttt{\textcolor{blue}{\footnotesize{}Int}}
value or a \texttt{\textcolor{blue}{\footnotesize{}String}} error
message
\item A value that is one case class out of a number of case classes
\end{itemize}
\item Disjunction types represent such values as types
\end{itemize}
\end{frame}

\begin{frame}{Disjunction types: mathematical point of view}

\begin{itemize}
\item The type of the function's argument represents the function's \emph{domain}
\begin{itemize}
\item For example: $f(x)$ where $x\in\mathbb{R}$ 
\end{itemize}
\item We would like to be able to represent arbitrary \emph{disjoint} domains
\begin{itemize}
\item For example: $x$ is either a point on a line or a point on a surface
\end{itemize}
\item In functional programming, the disjoint domains are always \textbf{labeled}
\begin{itemize}
\item For example: $x\in\left(\text{left},\mathbb{R}\right)\cup\left(\text{right},\mathbb{R}^{2}\right)$
\item The disjoint union is always an \emph{exclusive-or} 
\item Given any such $x$, we can determine its ``side'' of the union
\item We can obtain the corresponding value in each case
\end{itemize}
\item In Scala, this type is denoted \texttt{\textcolor{blue}{\footnotesize{}Either{[}Double,
(Double, Double){]}}}{\footnotesize \par}
\item Pattern-matching is used to define expressions with such types
\end{itemize}
\end{frame}

\begin{frame}{Disjunction type: \texttt{Either{[}A, B{]}}}

Example: \texttt{\textcolor{blue}{\footnotesize{}Either{[}String,
Int{]}}} (may be used for error reporting)
\begin{itemize}
\item Represents a value that is \emph{either} a \texttt{\textcolor{blue}{\footnotesize{}String}}
or an \texttt{\textcolor{blue}{\footnotesize{}Int}} (but not both)
\item Example values: \texttt{\textcolor{blue}{\footnotesize{}Left(\textquotedbl{}blah\textquotedbl{})}}
or \texttt{\textcolor{blue}{\footnotesize{}Right(123)}}{\footnotesize \par}
\item Use pattern matching to distinguish ``left'' from ``right'':
\end{itemize}
\begin{lyxcode}
\textcolor{blue}{\footnotesize{}def~logError(x:~Either{[}String,~Int{]}):~Int~=~x~match~\{}{\footnotesize \par}

\textcolor{blue}{\footnotesize{}~~case~Left(error)~$\Rightarrow$~println(s\textquotedbl{}Got~error:~\$error\textquotedbl{});~-1}{\footnotesize \par}

\textcolor{blue}{\footnotesize{}~~case~Right(res)~$\Rightarrow$~res}{\footnotesize \par}

\textcolor{blue}{\footnotesize{}\}}\textsf{\textcolor{gray}{\footnotesize{}~//~Left(``blah'')~and~Right(123)~are~possible~values~of~type~Either{[}String,~Int{]}}}{\footnotesize \par}
\end{lyxcode}
\begin{itemize}
\item Now \texttt{\textcolor{blue}{\footnotesize{}logError(Right(123))}}
returns \texttt{\textcolor{blue}{\footnotesize{}123}} while \texttt{\textcolor{blue}{\footnotesize{}logError(Left(\textquotedbl{}bad
result\textquotedbl{}))}} prints the error and returns \texttt{\textcolor{blue}{\footnotesize{}-1}}{\footnotesize \par}
\item The \texttt{\textcolor{blue}{\footnotesize{}case}} expression chooses
among possible values of a given type
\begin{itemize}
\item Note the similarity with this code:
\end{itemize}
\end{itemize}
\begin{lyxcode}
\textcolor{blue}{\footnotesize{}def~f(x:~Int):~Int~=~x~match~\{}{\footnotesize \par}

\textcolor{blue}{\footnotesize{}~~case~0~$\Rightarrow$~println(s\textquotedbl{}error:~must~be~nonzero\textquotedbl{});~-1}{\footnotesize \par}

\textcolor{blue}{\footnotesize{}~~case~1~$\Rightarrow$~println(s\textquotedbl{}error:~must~be~greater~than~1\textquotedbl{});~-1}{\footnotesize \par}

\textcolor{blue}{\footnotesize{}~~case~res~$\Rightarrow$~res}{\footnotesize \par}

\textcolor{blue}{\footnotesize{}\}}\textsf{\textcolor{gray}{\footnotesize{}~//~0~and~1~are~possible~values~of~type~Int}}{\footnotesize \par}
\end{lyxcode}
\end{frame}

\begin{frame}{More general disjunction types: trait + case classes}

A future version of Scala 3 has a short syntax for disjunction types:
\begin{itemize}
\item \texttt{\textcolor{blue}{\footnotesize{}type MyIntOrStr = Int | String}}{\footnotesize \par}
\item more generally, \texttt{\textcolor{blue}{\footnotesize{}type MyType
= List{[}Int{]} | (Int, Boolean) | MySocks}}{\footnotesize \par}
\begin{itemize}
\item Some libraries (scalaz, cats, shapeless) also provide shorter syntax
\end{itemize}
\end{itemize}
For now, in Scala 2, we use the ``long syntax'':

(specify names for each case and for each part, use ``\texttt{\textcolor{blue}{\footnotesize{}trait}}''
/ ``\texttt{\textcolor{blue}{\footnotesize{}extends}}'')
\begin{lyxcode}
\textcolor{blue}{\footnotesize{}sealed~trait~MyType}{\footnotesize \par}

\textcolor{blue}{\footnotesize{}final~case~class~HaveListInt(x:~List{[}Int{]})~extends~MyType}{\footnotesize \par}

\textcolor{blue}{\footnotesize{}final~case~class~HaveIntBool(s:~Int,~b:~Boolean)~extends~MyType}{\footnotesize \par}

\textcolor{blue}{\footnotesize{}final~case~class~HaveSocks(socks:~MySocks)~extends~MyType}{\footnotesize \par}
\end{lyxcode}
Pattern-matching example:
\begin{lyxcode}
\textcolor{blue}{\footnotesize{}val~x:~MyType~=~if~(...)~HaveSocks(...)~else~HaveListInt(...)}{\footnotesize \par}

\textcolor{blue}{\footnotesize{}...~}\textsf{\textcolor{gray}{\footnotesize{}//~some~other~code~here}}{\footnotesize \par}

\textcolor{blue}{\footnotesize{}x~match~\{}{\footnotesize \par}

\textcolor{blue}{\footnotesize{}~~case~HaveListInt(lst)~$\Rightarrow$~...}{\footnotesize \par}

\textcolor{blue}{\footnotesize{}~~case~HaveIntBool(p,~q)~$\Rightarrow$~...}{\footnotesize \par}

\textcolor{blue}{\footnotesize{}~~case~HaveSocks(s)~$\Rightarrow$~...}{\footnotesize \par}

\textcolor{blue}{\footnotesize{}\}}{\footnotesize \par}
\end{lyxcode}
\end{frame}

\begin{frame}{The most often used disjunction type: \texttt{Option{[}T{]}}}

A simple implementation: 
\begin{lyxcode}
\textcolor{blue}{\footnotesize{}sealed~trait~Option{[}T{]}}{\footnotesize \par}

\textcolor{blue}{\footnotesize{}final~case~class~Some{[}T{]}(t:~T)~extends~Option{[}T{]}}{\footnotesize \par}

\textcolor{blue}{\footnotesize{}final~case~object~None~extends~Option{[}Nothing{]}}{\footnotesize \par}
\end{lyxcode}
Pattern-matching example:
\begin{lyxcode}
\textcolor{blue}{\footnotesize{}def~saveDivide(x:~Double,~y:~Double):~Option{[}Double{]}~=~\{}{\footnotesize \par}

\textcolor{blue}{\footnotesize{}~~if~(y~==~0)~None~else~Some(x~/~y)}{\footnotesize \par}

\textsf{\textcolor{gray}{\footnotesize{}//~Example~usage:}}{\footnotesize \par}

\textcolor{blue}{\footnotesize{}val~result~=~safeDivide(1.0,~q)~match~\{}{\footnotesize \par}

\textcolor{blue}{\footnotesize{}~~case~Some(x)~$\Rightarrow$~previousResult~{*}~x}{\footnotesize \par}

\textcolor{blue}{\footnotesize{}~~case~None~$\Rightarrow$~previousResult~}\textsf{\textcolor{gray}{\footnotesize{}//~provide~a~default~value}}{\footnotesize \par}

\textcolor{blue}{\footnotesize{}\}}{\footnotesize \par}
\end{lyxcode}
Many Scala library functions return an \texttt{\textcolor{blue}{\footnotesize{}Option{[}T{]}}}{\footnotesize \par}
\begin{itemize}
\item \texttt{\textcolor{blue}{\footnotesize{}find, headOption, reduceOption,
get}} (for\texttt{\textcolor{blue}{\footnotesize{} Map{[}K, V{]}}}),
\texttt{\textcolor{blue}{\footnotesize{}lift}} for \texttt{\textcolor{blue}{\footnotesize{}Array{[}T{]}}}{\footnotesize \par}
\begin{itemize}
\item Note: \texttt{\textcolor{blue}{\footnotesize{}Option{[}T{]}}} is ``collection-like'':
has \texttt{\textcolor{blue}{\footnotesize{}map}}, \texttt{\textcolor{blue}{\footnotesize{}flatMap}},
\texttt{\textcolor{blue}{\footnotesize{}filter}}, \texttt{\textcolor{blue}{\footnotesize{}exists}}...
\end{itemize}
\end{itemize}
\end{frame}

\begin{frame}{Worked examples}

\begin{itemize}
\item What problems can we solve now?
\begin{itemize}
\item Represent values from disjoint sets as a single type
\item Use such values in collections safely
\end{itemize}
\item Define a disjunction type \texttt{\textcolor{blue}{\footnotesize{}DayOfWeek}}
representing the seven days.
\item Modify \texttt{\textcolor{blue}{\footnotesize{}DayOfWeek}} so that
the values additionally represent a restaurant name and total amount
for Fridays and a wake-up time on Saturdays.
\item Define a disjunction type \texttt{\textcolor{blue}{\footnotesize{}RootsOfQuadratic}}
that represents real-valued roots of the equation $x^{2}+bx+c=0$
for arbitrary real $b$, $c$. (The cases of interest are: no real
roots; two equal roots; two unequal roots.) Implement \texttt{\textcolor{blue}{\footnotesize{}solve2:\ ((Double,
Double)) $\Rightarrow$ RootsOfQuadratic}}.
\item Define a function \texttt{\textcolor{blue}{\footnotesize{}rootAverage:\ RootsOfQuadratic
$\Rightarrow$ Option{[}Double{]}}} that computes the average value
of all real roots, returning \texttt{\textcolor{blue}{\footnotesize{}None}}
if the average is undefined.
\item Generate 100 random coefficients $b$, $c$ (uniformly distributed
between $-1$ and $1$) and compute the mean of \texttt{\textcolor{blue}{\footnotesize{}rootAverage}}
for them.
\item Implement \texttt{\textcolor{blue}{\footnotesize{}def f{[}A, B{]}:\ (Option{[}A{]},
Option{[}B{]}) $\Rightarrow$ Option{[}(A, B){]}}}{\footnotesize \par}
\end{itemize}
\end{frame}

\begin{frame}{Exercises II}

\begin{enumerate}
\item Define a disjunction type \texttt{\textcolor{blue}{\footnotesize{}CellState}}
representing the visual state of one cell in the ``\href{https://en.wikipedia.org/wiki/Minesweeper_(video_game)}{Minesweeper}''
game: A cell can be either closed, or display a bomb, or be open and
display the number of neighbor bombs.
\item Define a function from \texttt{\textcolor{blue}{\footnotesize{}Seq{[}Seq{[}CellState{]}{]}}}
to \texttt{\textcolor{blue}{\footnotesize{}Int}}, counting the total
number of cells with $0$ neighbor bombs shown.
\item Define a disjunction type \texttt{\textcolor{blue}{\footnotesize{}RootOfLinear}}
representing all possibilities for the solution of the equation $ax+b=0$
for arbitrary real $a$, $b$. (The cases of interest are: no roots;
one root; all $x$ are roots.) Implement the solution as a function
\texttt{\textcolor{blue}{\footnotesize{}solve1:\ ((Double, Double))
$\Rightarrow$ RootOfLinear}}.
\item Given a \texttt{\textcolor{blue}{\footnotesize{}Seq{[}(Double, Double){]}}}
containing pairs $\left(a,b\right)$ of the coefficients of $ax+b=0$,
use \texttt{\textcolor{blue}{\footnotesize{}solve1}} to produce a
\texttt{\textcolor{blue}{\footnotesize{}Seq{[}Double{]}}} containing
the roots of that equation when a unique root exists.
\item Define fully parametric functions having these types: 
\begin{enumerate}
\item \texttt{\textcolor{blue}{\footnotesize{}def f1{[}A, B{]}:\ Option{[}(A,
B){]} $\Rightarrow$ (Option{[}A{]}, Option{[}B{]})}}{\footnotesize \par}
\item \texttt{\textcolor{blue}{\footnotesize{}def f2{[}A, B{]}:\ Either{[}A,B{]}
$\Rightarrow$ (Option{[}A{]}, Option{[}B{]})}}{\footnotesize \par}
\item \texttt{\textcolor{blue}{\footnotesize{}def f3{[}A,B,C{]}:\ Either{[}A,
Either{[}B,C{]}{]} $\Rightarrow$ Either{[}Either{[}A,B{]}, C{]}}}{\footnotesize \par}
\end{enumerate}
\end{enumerate}
\end{frame}

\end{document}
